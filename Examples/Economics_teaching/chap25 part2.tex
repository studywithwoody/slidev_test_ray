\documentclass{beamer}
\usepackage{graphicx} % Required for inserting images
\usepackage{amsmath}
\usepackage[most]{tcolorbox}
\usepackage{lmodern}
\usepackage{mathabx}

\usetheme{Madrid} % 可選其他主題:e.g., Warsaw, Berkeley, etc.
\usecolortheme{default}
\setbeamertemplate{caption}[numbered]% Number float-like environments
% Customize the caption
\setbeamerfont{caption}{size=\footnotesize}
% \setbeamercolor{caption}{fg=blue}
% \setbeamercolor{caption name}{fg=red}

% 每章節開始時自動產生章節頁
\AtBeginSection[]
{
  \begin{frame}
    \frametitle{Table of Contents}
    \tableofcontents[currentsection]
  \end{frame}
}


\title{Mankiw's Principles of Economics}
\subtitle{chap 25: Production and Growth}
\author{Hsu Chun-Wei}
\date{July 2025}

\begin{document}

\maketitle

% 目錄頁
\begin{frame}
  \frametitle{Table of Contents}
  \tableofcontents
\end{frame}

%==========================================================
\section{25-3 Economic Growth and Public Policy}
%==========================================================
\subsection{Saving and Investment}
%==========================================================
\begin{frame}{Saving and Investment}

Because capital is a produced factor of production, a society can change the amount of capital it has.

\begin{tcolorbox}[colframe=blue!70!black, colback=white, title=\textbf{Saving and Investment}]
Saving and investment increase the stock of physical capital, which raises future productivity and output.
\end{tcolorbox}

\begin{itemize}
    \item Producing more capital goods today increases capital tomorrow.
    \item More capital allows workers to produce more goods and services.
    \item Higher productivity leads to higher living standards in the long run.
\end{itemize}

Encouraging saving and investment is one way governments can promote long-run economic growth.

\end{frame}

%==========================================================
\begin{frame}{The Trade-Off in Capital Accumulation}

Investment in capital is not a free lunch.

\begin{tcolorbox}[colframe=red!80!black, colback=white, title=\textbf{A Key Trade-Off}]
To invest more in capital, society must consume less today and save more of its current income.
\end{tcolorbox}

\begin{itemize}
    \item Resources are scarce.
    \item Producing capital goods requires fewer resources devoted to current consumption.
    \item Higher future consumption comes at the cost of lower current consumption.
\end{itemize}

This illustrates one of the \textit{Ten Principles of Economics}:  
People face trade-offs.

\end{frame}
%==========================================================
\subsection{Diminishing Returns and the Catch-Up Effect}
%==========================================================
\begin{frame}{Diminishing Returns to Capital}

\begin{center}
    \includegraphics[width=0.8\textwidth]{pictures/chap 25/T2.png}
\end{center}

\end{frame}
%==========================================================
\begin{frame}{Diminishing Returns to Capital}

Suppose a country increases its saving rate to accumulate more capital.

\begin{tcolorbox}[colframe=red!80!black, colback=white, title=\textbf{Diminishing Returns}]
As the stock of capital rises, the extra output produced by an additional unit of capital falls.
\end{tcolorbox}

\begin{itemize}
    \item When workers have little capital, extra capital raises productivity a lot.
    \item When workers already have much capital, extra capital raises productivity only slightly.
\end{itemize}

This idea holds other factors constant, such as technology and human capital.

\end{frame}

%==========================================================
\begin{frame}{Saving and Long-Run Growth}

An increase in the saving rate raises capital accumulation.

\begin{tcolorbox}[colframe=blue!70!black, colback=white, title=\textbf{Short Run vs Long Run}]
\begin{itemize}
    \item Higher saving leads to faster growth for a period of time.
    \item As capital accumulates, diminishing returns set in.
    \item Growth slows down over time.
\end{itemize}
\end{tcolorbox}

\begin{itemize}
    \item In the long run, higher saving leads to a higher level of output and income.
    \item It does not lead to a permanently higher growth rate.
\end{itemize}

Thus, capital accumulation raises living standards, but cannot sustain growth forever.

\end{frame}

%==========================================================
\begin{frame}{The Catch-Up Effect}

Diminishing returns have an important implication for economic growth.

\begin{tcolorbox}[colframe=red!80!black, colback=white, title=\textbf{Catch-Up Effect}]
Countries that start out relatively poor tend to grow faster than countries that start out rich.
\end{tcolorbox}

\begin{itemize}
    \item Poor countries have low capital per worker.
    \item Small increases in capital can greatly raise productivity.
    \item Rich countries already have high capital per worker.
\end{itemize}

International evidence shows that, holding other factors constant, poorer countries tend to grow faster than richer ones.

\end{frame}
%==========================================================
\subsection{Investment from Abroad}
%==========================================================
\begin{frame}{Investment from Abroad}

Domestic saving is not the only way for a country to invest in new capital.

\begin{tcolorbox}[colframe=blue!70!black, colback=white, title=\textbf{Investment from Abroad}]
Investment from abroad occurs when foreigners provide resources to finance capital accumulation in a country.
\end{tcolorbox}

Foreign investment increases:
\begin{itemize}
    \item The capital stock of the host country
    \item Productivity and output
    \item Wages in the long run
\end{itemize}

Thus, foreign investment is one way governments can promote economic growth.

\end{frame}

%==========================================================
\begin{frame}{Forms of Foreign Investment}

Investment from abroad takes two main forms.

\begin{tcolorbox}[colframe=red!80!black, colback=white, title=\textbf{Two Types of Foreign Investment}]
\begin{itemize}
    \item \textbf{Foreign Direct Investment (FDI)}:  
    Capital owned and operated by a foreign entity (e.g., building a factory).
    \item \textbf{Foreign Portfolio Investment (FPI)}:  
    Capital financed by foreigners but operated by domestic residents (e.g., buying stock).
\end{itemize}
\end{tcolorbox}

In both cases:
\begin{itemize}
    \item Foreign saving finances domestic investment.
    \item The host country’s capital stock increases.
\end{itemize}

\end{frame}

%==========================================================
\begin{frame}{GDP, GNP, and International Institutions}

Foreign investment does not affect all measures of prosperity in the same way.

\begin{tcolorbox}[colframe=blue!70!black, colback=white, title=\textbf{GDP vs GNP}]
\begin{itemize}
    \item \textbf{GDP}: income earned within a country by residents and nonresidents.
    \item \textbf{GNP}: income earned by residents of a country, at home and abroad.
\end{itemize}
\end{tcolorbox}

Because some profits flow back to foreign owners:
\begin{itemize}
    \item Foreign investment raises GDP more than it raises GNP in the host country.
\end{itemize}

\end{frame}

%==========================================================
\subsection{Education}
%==========================================================
\begin{frame}{Education and Economic Growth}

Education is an investment in human capital and is crucial for long-run economic growth.

\begin{tcolorbox}[colframe=blue!70!black, colback=white, title=\textbf{Education as Investment}]
Investment in education raises workers’ skills and productivity, leading to higher wages and higher living standards.
\end{tcolorbox}

\begin{itemize}
    \item Each additional year of schooling raises a worker’s productivity.
    \item The return to education is especially large in less developed countries.
    \item Education can be as important as physical capital for economic success.
\end{itemize}

Governments can raise living standards by providing good schools and encouraging education.

\end{frame}

%==========================================================
\begin{frame}{Costs and Benefits of Education}

Like all investments, education has an opportunity cost.

\begin{tcolorbox}[colframe=red!80!black, colback=white, title=\textbf{Opportunity Cost}]
When students are in school, they forgo wages they could have earned by working.
\end{tcolorbox}

In poor countries, children may leave school early because their labor is needed to support their families.

\begin{tcolorbox}[colframe=blue!70!black, colback=white, title=\textbf{Positive Externalities of Education}]
\begin{itemize}
    \item Educated workers generate new ideas.
    \item Knowledge can spill over to others.
    \item Social returns to education may exceed private returns.
\end{itemize}
\end{tcolorbox}

These external benefits help justify public subsidies for education.

\end{frame}

%==========================================================
\begin{frame}{Brain Drain}

One challenge facing poor countries is the brain drain.

\begin{tcolorbox}[colframe=red!80!black, colback=white, title=\textbf{Brain Drain}]
Brain drain is the emigration of highly educated workers from poor countries to rich countries.
\end{tcolorbox}

\begin{itemize}
    \item Educated workers seek higher wages abroad.
    \item Poor countries lose part of their human capital.
    \item The loss may reduce productivity and growth at home.
\end{itemize}

This creates a policy dilemma:
\begin{itemize}
    \item Studying abroad raises individual skills.
    \item But some workers may not return home.
\end{itemize}

Governments must balance the benefits and costs of international mobility of educated workers.

\end{frame}
%==========================================================
\subsection{Health and Nutrition}
%==========================================================
\begin{frame}{Health and Nutrition}

Human capital usually refers to education, but it also includes health and nutrition.

\begin{tcolorbox}[colframe=blue!70!black, colback=white, title=\textbf{Health as Human Capital}]
Investment in health and nutrition leads to a healthier population, which raises productivity and living standards.
\end{tcolorbox}

\begin{itemize}
    \item Healthier workers are more productive.
    \item Better nutrition improves physical and mental capacity.
    \item Health investment complements education and physical capital.
\end{itemize}

Thus, policies that improve health are one way to promote long-run economic growth.

\end{frame}

%==========================================================
\begin{frame}{Nutrition and Productivity}

\begin{tcolorbox}[colframe=red!80!black, colback=white, title=\textbf{Key Historical Insight}]
Better nutrition increases workers’ ability to perform physical labor and raise productivity.
\end{tcolorbox}

Evidence from history shows:
\begin{itemize}
    \item As nutrition improves, average height rises.
    \item Height is an indicator of early-life nutrition.
    \item Taller workers tend to earn higher wages.
\end{itemize}

These patterns suggest a strong link between health, nutrition, and productivity.

\end{frame}

%==========================================================
\begin{frame}{Health and Economic Development}

In many developing countries, poor health and malnutrition remain serious obstacles to growth.

\begin{tcolorbox}[colframe=red!80!black, colback=white, title=\textbf{A Vicious Circle}]
\begin{itemize}
    \item Poor countries are poor partly because their populations are unhealthy.
    \item Populations are unhealthy partly because countries are poor.
\end{itemize}
\end{tcolorbox}

However, this also creates the possibility of a virtuous circle.

\begin{tcolorbox}[colframe=blue!70!black, colback=white, title=\textbf{A Virtuous Circle}]
Economic growth improves health outcomes, which further raises productivity and promotes growth.
\end{tcolorbox}

Public policies that improve healthcare and nutrition can help economies escape poverty traps.

\end{frame}
%==========================================================
\begin{frame}{Does Food Aid Help or Hurt?}

Economic policies often have unintended consequences.

\begin{tcolorbox}[colframe=red!80!black, colback=white, title=\textbf{Central Question}]
Does food aid to developing countries reduce human suffering, or can it unintentionally increase armed conflict?
\end{tcolorbox}

Researchers studied the effects of U.S. food aid on civil conflict in developing countries.

\begin{tcolorbox}[colframe=blue!70!black, colback=white, title=\textbf{Key Finding}]
Increases in food aid are associated with a higher incidence, onset, and duration of armed civil conflict.
\end{tcolorbox}

The study examined food aid flows to 134 developing countries over several decades.

\end{frame}

%==========================================================
\begin{frame}{Unintended Consequences of Food Aid}

Food aid can be vulnerable to misuse in conflict-prone environments.

\begin{tcolorbox}[colframe=red!80!black, colback=white, title=\textbf{Possible Mechanisms}]
\begin{itemize}
    \item Armed groups may seize food aid to feed soldiers.
    \item Food aid convoys are valuable and easily captured.
    \item Aid may intensify existing ethnic or political tensions.
\end{itemize}
\end{tcolorbox}

The problem is especially severe in countries with:
\begin{itemize}
    \item Weak infrastructure (few roads).
    \item Limited government control over territory.
    \item Deep ethnic divisions.
\end{itemize}

\begin{tcolorbox}[colframe=blue!70!black, colback=white, title=\textbf{Policy Implication}]
Well-intentioned policies can fail without strong institutions and political stability.
\end{tcolorbox}

\end{frame}
%==========================================================
\subsection{Property Rights and Political Stability}
%==========================================================
\begin{frame}{Property Rights and Economic Growth}

Market economies rely on millions of transactions among firms and consumers.

\begin{tcolorbox}[colframe=blue!70!black, colback=white, title=\textbf{Role of Prices and Markets}]
Market prices coordinate economic activity by bringing supply and demand into balance.
\end{tcolorbox}

For markets to function efficiently, an economy must respect property rights.

\begin{tcolorbox}[colframe=red!80!black, colback=white, title=\textbf{Property Rights}]
Property rights are the ability of people to exercise authority over the resources they own.
\end{tcolorbox}

Without secure property rights, individuals and firms have little incentive to produce, invest, or trade.

\end{frame}

%==========================================================
\begin{frame}{Enforcing Property Rights}

Property rights require enforcement.

\begin{tcolorbox}[colframe=blue!70!black, colback=white, title=\textbf{Institutions That Enforce Property Rights}]
\begin{itemize}
    \item Courts that enforce contracts
    \item Legal systems that punish theft and fraud
    \item Honest and effective government officials
\end{itemize}
\end{tcolorbox}

In many poor countries:
\begin{itemize}
    \item Contracts are difficult to enforce.
    \item Corruption is widespread.
    \item Firms may be required to bribe officials.
\end{itemize}

\begin{tcolorbox}[colframe=red!80!black, colback=white, title=\textbf{Economic Consequences}]
Weak property rights discourage domestic saving and foreign investment.
\end{tcolorbox}

\end{frame}

%==========================================================
\begin{frame}{Political Stability and Prosperity}

Property rights are closely linked to political stability.

\begin{tcolorbox}[colframe=red!80!black, colback=white, title=\textbf{Political Instability}]
Revolutions, coups, and threats of confiscation reduce incentives to save and invest.
\end{tcolorbox}

\begin{itemize}
    \item Domestic residents fear losing their assets.
    \item Foreign investors avoid unstable countries.
    \item Capital accumulation slows down.
\end{itemize}

\begin{tcolorbox}[colframe=blue!70!black, colback=white, title=\textbf{Key Conclusion}]
Economic prosperity depends in part on political prosperity.
\end{tcolorbox}

Countries with secure property rights, honest governments, and stable political systems tend to enjoy higher standards of living.

\end{frame}
%==========================================================
\subsection{Free Trade}
%==========================================================
\begin{frame}{Free Trade and Economic Growth}

Some countries have pursued inward-oriented policies to promote growth.

\begin{tcolorbox}[colframe=red!80!black, colback=white, title=\textbf{Inward-Oriented Policies}]
Policies that restrict trade, such as tariffs and quotas, to protect domestic industries from foreign competition.
\end{tcolorbox}

These policies are often justified by the infant-industry argument.

However, most economists believe that countries are better off pursuing outward-oriented policies.

\begin{tcolorbox}[colframe=blue!70!black, colback=white, title=\textbf{Outward-Oriented Policies}]
Policies that integrate an economy into the global market through international trade.
\end{tcolorbox}

Countries that open to trade tend to experience faster economic growth.

\end{frame}

%==========================================================
\begin{frame}{Why Trade Promotes Growth}

International trade improves living standards by allowing countries to specialize.

\begin{tcolorbox}[colframe=blue!70!black, colback=white, title=\textbf{Trade as a Form of Technology}]
Trade allows a country to consume goods as if it had invented a new technology.
\end{tcolorbox}

\begin{itemize}
    \item Exports allow specialization in what a country does best.
    \item Imports provide access to goods and capital equipment.
    \item Eliminating trade restrictions can raise productivity.
\end{itemize}

Trade liberalization can generate growth similar to that from a major technological advance.

\end{frame}

%==========================================================
\begin{frame}{Geography and International Trade}

The extent of a country’s trade depends not only on policy but also on geography.

\begin{tcolorbox}[colframe=red!80!black, colback=white, title=\textbf{Geographic Constraints}]
Countries with easy access to seaports tend to trade more than landlocked countries.
\end{tcolorbox}

\begin{itemize}
    \item Coastal access lowers transportation costs.
    \item Trade is more difficult for landlocked countries.
    \item Geographic barriers can limit integration into global markets.
\end{itemize}

Access to international trade is an important determinant of productivity and income.Geography helps explain why some regions have persistently lower income levels.

\end{frame}

%==========================================================
\subsection{Research and Development}
%==========================================================
\begin{frame}{Research and Development}

The primary reason living standards are higher today than in the past is technological progress.

\begin{tcolorbox}[colframe=blue!70!black, colback=white, title=\textbf{Role of Technological Knowledge}]
Advances in technology improve the ability to produce goods and services and raise long-run living standards.
\end{tcolorbox}

\begin{itemize}
    \item Innovations increase productivity.
    \item Higher productivity leads to higher income.
    \item Economic growth depends crucially on new ideas.
\end{itemize}

Many technological advances come from private research, but there is also a role for public policy.

\end{frame}

%==========================================================
\begin{frame}{Knowledge as a Public Good}

Technological knowledge differs from ordinary goods.

\begin{tcolorbox}[colframe=red!80!black, colback=white, title=\textbf{Knowledge as a Public Good}]
Once an idea is discovered, it enters society’s pool of knowledge and others can use it at little cost.
\end{tcolorbox}

Because private markets may underinvest in research:
\begin{itemize}
    \item Governments support basic research.
    \item Governments provide grants and tax incentives.
\end{itemize}

\end{frame}

%==========================================================
\subsection{Population Growth}
%==========================================================
\begin{frame}{Population Growth}

Population growth has long been debated for its impact on economic development.

\begin{tcolorbox}[colframe=blue!70!black, colback=white, title=\textbf{Population Size}]
A larger population means a larger labor force and greater total production.
\end{tcolorbox}

At the same time:
\begin{itemize}
    \item A larger population also means more people consuming goods and services.
    \item Higher total output does not necessarily imply a higher standard of living.
\end{itemize}

Countries with both large and small populations can be found at all levels of economic development. Population growth interacts with other factors of production in complex ways.

\end{frame}

%==========================================================
\begin{frame}{Stretching Natural Resources}

An early and influential view on population growth was proposed by Thomas Robert Malthus.

\begin{tcolorbox}[colframe=red!80!black, colback=white, title=\textbf{Malthusian Theory}]
Malthus argued that population growth would outpace food production, condemning humanity to persistent poverty.
\end{tcolorbox}

According to Malthus:
\begin{itemize}
    \item Population tends to grow faster than the food supply.
    \item Poverty is restrained only by misery and vice.
    \item Efforts to alleviate poverty would worsen the problem.
\end{itemize}

His prediction suggested that rising population would inevitably strain natural resources.

\end{frame}

%==========================================================
\begin{frame}{Why Malthus Was Wrong}

History has not supported Malthus’s pessimistic forecast.

\begin{tcolorbox}[colframe=blue!70!black, colback=white, title=\textbf{What Actually Happened}]
\begin{itemize}
    \item World population increased dramatically.
    \item Living standards rose rather than fell.
    \item Hunger and malnutrition became less common on average.
\end{itemize}
\end{tcolorbox}

The key reason is technological progress:
\begin{itemize}
    \item Advances in agriculture increased productivity.
    \item Each worker can now feed far more people.
    \item Human ingenuity offset the strain from population growth.
\end{itemize}

Population growth does not doom societies to poverty when productivity continues to rise.

\end{frame}

%==========================================================
\begin{frame}{Diluting the Capital Stock}

Modern growth theory emphasizes how population growth affects capital accumulation.

\begin{tcolorbox}[colframe=red!80!black, colback=white, title=\textbf{Capital Dilution}]
Rapid population growth reduces capital per worker by spreading the capital stock more thinly.
\end{tcolorbox}

\begin{itemize}
    \item Faster population growth increases the number of workers.
    \item Capital must be shared among more workers.
    \item Capital per worker $(K/L)$ falls.
\end{itemize}

Lower capital per worker leads to lower productivity and lower GDP per worker.

\end{frame}

%==========================================================
\begin{frame}{Population Growth and Human Capital}

Capital dilution is especially severe for human capital.

\begin{tcolorbox}[colframe=blue!70!black, colback=white, title=\textbf{Human Capital Dilution}]
Countries with rapid population growth have many school-age children, placing heavy burdens on education systems.
\end{tcolorbox}

\begin{itemize}
    \item Educational resources must be spread across more students.
    \item Average educational attainment tends to be lower.
    \item Human capital per worker $(H/L)$ grows more slowly.
\end{itemize}

This makes it harder for workers to achieve high productivity levels.

\end{frame}

%==========================================================
\begin{frame}{Policies and Population Growth}

Some economists argue that reducing population growth can raise living standards.

\begin{tcolorbox}[colframe=red!80!black, colback=white, title=\textbf{Policy Approaches}]
\begin{itemize}
    \item Laws that limit family size.
    \item Programs that increase access to birth control.
    \item Policies that raise awareness of family planning.
\end{itemize}
\end{tcolorbox}
\end{frame}

%==========================================================
\begin{frame}{Policies and Population Growth}

Population growth also responds to incentives.

\begin{tcolorbox}[colframe=blue!70!black, colback=white, title=\textbf{People Respond to Incentives}]
When the opportunity cost of having children rises, families tend to choose smaller families.
\end{tcolorbox}

Expanding education and employment opportunities for women can reduce population growth and raise living standards.

\end{frame}

%==========================================================
\begin{frame}{Promoting Technological Progress}

Rapid population growth may reduce capital per worker, but it may also generate benefits.

\begin{tcolorbox}[colframe=blue!70!black, colback=white, title=\textbf{A Positive View of Population Growth}]
A larger population can promote technological progress by increasing the number of scientists, inventors, and engineers.
\end{tcolorbox}

\begin{itemize}
    \item More people means more potential innovators.
    \item Technological advances raise productivity for everyone.
    \item Population growth can act as an engine of economic progress.
\end{itemize}

This view contrasts with the idea that population growth necessarily lowers living standards.

\end{frame}

%==========================================================
\begin{frame}{Population Size and Innovation}

Some economists argue that technological progress occurs faster in larger populations.

\begin{tcolorbox}[colframe=red!80!black, colback=white, title=\textbf{Key Evidence}]
\begin{itemize}
    \item Over human history, growth rates rose as world population increased.
    \item Larger regions tended to experience faster technological development.
    \item Smaller and isolated populations advanced more slowly or even regressed.
\end{itemize}
\end{tcolorbox}

A large population increases the likelihood of discovering new ideas and technologies.

\end{frame}

%==========================================================
\section{Conclusion}
%==========================================================
\begin{frame}{The Production Function and Economic Growth}

Economic growth is explained using the production function:

\[
Y = A F(K, H, N)
\]

\begin{itemize}
    \item $Y$ : total output (GDP)
    \item $K$ : physical capital
    \item $H$ : human capital
    \item $N$ : labor (population)
    \item $A$ : technological knowledge
\end{itemize}

Assuming constant returns to scale:

\[
\frac{Y}{N} = A F\left(\frac{K}{N}, \frac{H}{N}, 1\right)
\]


\end{frame}

%==========================================================
\begin{frame}{Economic Growth: Theory Summary}

\renewcommand{\arraystretch}{1.4}
\begin{tabular}{|p{2.8cm}|p{4.2cm}|p{3.8cm}|}
\hline
\textbf{Factor} & \textbf{Effect on Growth} & \textbf{Link to Formula} \\
\hline
Physical Capital $(K)$ 
& Investment raises output but faces diminishing returns 
& $K/N \uparrow \Rightarrow Y/N \uparrow$ (level) \\
\hline
Human Capital $(H)$ 
& Education and training raise worker productivity 
& $H/N \uparrow \Rightarrow Y/N \uparrow$ \\
\hline
Population Growth $(N)$ 
& Rapid growth dilutes capital per worker 
& $N \uparrow \Rightarrow K/N \downarrow,\; H/N \downarrow$ \\
\hline
Population Growth (Innovation View) 
& Larger populations generate more ideas 
& $N \uparrow \Rightarrow A \uparrow$ \\
\hline
Technological Knowledge $(A)$ 
& Sustained source of long-run growth 
& $A \uparrow \Rightarrow Y/N \uparrow$ (permanently) \\
\hline
\end{tabular}

\end{frame}
%==========================================================
\begin{frame}{Economic Growth: Theory Summary}

\renewcommand{\arraystretch}{1.4}
\begin{tabular}{|p{2.8cm}|p{4.2cm}|p{3.8cm}|}
\hline
\textbf{Factor} & \textbf{Effect on Growth} & \textbf{Link to Formula} \\
\hline
Saving and Investment 
& Raises future productivity at a cost of current consumption 
& $K \uparrow \Rightarrow Y \uparrow$ \\
\hline
Institutions and Property Rights 
& Encourage investment and innovation 
& $K \uparrow,\; A \uparrow$ \\
\hline
Technological Knowledge $(A)$ 
& Sustained source of long-run growth 
& $A \uparrow \Rightarrow Y/N \uparrow$ (permanently) \\
\hline
Saving and Investment 
& Raises future productivity at a cost of current consumption 
& $K \uparrow \Rightarrow Y \uparrow$ \\
\hline
Institutions and Property Rights 
& Encourage investment and innovation 
& $K \uparrow,\; A \uparrow$ \\
\hline
\end{tabular}

\end{frame}
%==========================================================
%==========================================================
%==========================================================
%==========================================================
%==========================================================
%==========================================================
%==========================================================
\end{document}