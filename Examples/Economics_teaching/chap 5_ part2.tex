\documentclass{beamer}
\usepackage{graphicx} % Required for inserting images
\usepackage{amsmath}
\usepackage[most]{tcolorbox}
\usepackage{lmodern}
\usepackage{mathabx}

\usetheme{Madrid} % 可選其他主題:e.g., Warsaw, Berkeley, etc.
\usecolortheme{default}
\setbeamertemplate{caption}[numbered]% Number float-like environments
% Customize the caption
\setbeamerfont{caption}{size=\footnotesize}
% \setbeamercolor{caption}{fg=blue}
% \setbeamercolor{caption name}{fg=red}

% 每章節開始時自動產生章節頁
\AtBeginSection[]
{
  \begin{frame}
    \frametitle{Table of Contents}
    \tableofcontents[currentsection]
  \end{frame}
}


\title{Mankiw's Principles of Economics}
\subtitle{chap 5: Elasticity and
Its Application}
\author{Hsu Chun-Wei}
\date{July 2025}

\begin{document}

\maketitle

% 目錄頁
\begin{frame}
  \frametitle{Table of Contents}
  \tableofcontents
\end{frame}

%==========================================================
\section{The Applications of Supply, Demand,
and Elasticity}
%==========================================================
\begin{frame}{Can Good News for Farming Be Bad News for Farmers?}

Imagine a Kansas wheat farmer. A new hybrid of wheat is discovered that increases output per acre by 20\%. How does this affect farmers?

\begin{tcolorbox}[colframe=blue!70!black, colback=white, title=\textbf{Step 1: Which Curve Shifts?}]
The new hybrid increases productivity, allowing farmers to produce more wheat at any price.  
\textbf{The supply curve shifts to the right.}
\end{tcolorbox}

\begin{tcolorbox}[colframe=red!80!black, colback=white, title=\textbf{Step 2: Direction of the Shift}]
Supply increases from $S_1$ to $S_2$.  
Quantity rises (100 → 110), and price falls (from \$3 → \$2).  
Demand remains unchanged.
\end{tcolorbox}

This raises a key question:  
\textbf{Does producing more wheat make farmers better off or worse off?}

\end{frame}
%==========================================================
\begin{frame}{Frame Title}
    \begin{center}
        \includegraphics[width=0.7\textwidth]{pictures/chap5/T16.png}
    \end{center}
\end{frame}
%==========================================================
\begin{frame}{Why More Wheat Can Lower Farmers' Income}

Farmers' total revenue equals \(\text{Price} \times \text{Quantity}\).  
A technological improvement affects both components:

\begin{itemize}
    \item Quantity supplied increases (Q rises).
    \item Market price of wheat falls (P falls).
\end{itemize}

\begin{tcolorbox}[colframe=blue!70!black, colback=white, title=\textbf{Key: Demand for Wheat Is Inelastic}]
Because basic food products have few substitutes, demand for wheat is inelastic.  
Thus, when price falls, \textbf{total revenue falls}.
\end{tcolorbox}

Example from the text:
\begin{itemize}
    \item Price falls significantly, from \$300 to \$220.
    \item Quantity sold rises only slightly.
\end{itemize}

\textbf{Result:} The new hybrid reduces total revenue for farmers.

\end{frame}

%==========================================================
\begin{frame}{Long-Run Impact on Farmers}

If farmers are worse off, why do they use the new hybrid?  
Because each farmer takes the price as given and benefits individually by producing more.  
But when \textbf{all farmers adopt the new hybrid}, the increase in total supply drives the price downward, making all farmers worse off.

\begin{tcolorbox}[colframe=red!80!black, colback=white, title=\textbf{Historical Perspective}]
Over the past two centuries, advances in farm technology greatly increased food output.  
With inelastic food demand, higher supply led to falling farm revenues, pushing many workers out of farming.
\end{tcolorbox}
\end{frame}

%==========================================================
\begin{frame}{Policy Lessons}

This also explains certain farm policies:
\begin{itemize}
    \item Governments sometimes pay farmers \textbf{not} to plant crops.
    \item By reducing supply, prices rise and farm incomes increase.
\end{itemize}

\begin{tcolorbox}[colframe=blue!70!black, colback=white, title=\textbf{Important Insight}]
What is good for consumers (cheaper food) may not be good for farmers.  
Improved technology benefits society but can hurt farmers.
\end{tcolorbox}

\end{frame}
%==========================================================
\begin{frame}{Why Did OPEC Fail to Keep the Price of Oil High?}

In the 1970s, OPEC members attempted to raise the world price of oil by jointly reducing supply.  
This policy initially succeeded: oil prices rose dramatically in both the early and late 1970s.

However, OPEC struggled to maintain high prices. From 1982 to 1985, oil prices fell steadily, and by 1986 cooperation collapsed, causing a 45\% plunge in price. Prices later returned to 1970 levels and fluctuated widely in the 21st century due to changing global demand.

\end{frame}
%==========================================================
\begin{frame}{Why Did OPEC Fail to Keep the Price of Oil High?}
    \begin{tcolorbox}[colframe=blue!70!black, colback=white, title=\textbf{Short-Run Elasticities Explain Early Success}]
In the short run:
\begin{itemize}
    \item Supply of oil is inelastic — extraction capacity cannot change quickly.
    \item Demand for oil is inelastic — consumers cannot easily change consumption habits.
\end{itemize}
Because both curves are steep, a leftward shift in supply (from \(S_1\) to \(S_2\)) causes a \textbf{large increase in price}.  
Thus OPEC earned higher revenues even while selling less oil.
\end{tcolorbox}
\end{frame}
%==========================================================
\begin{frame}{Short Run vs. Long Run: Why OPEC Eventually Failed}
    \begin{center}
        \includegraphics[width=\textwidth]{pictures/chap5/T17.png}
    \end{center}
\end{frame}
%==========================================================
\begin{frame}{Short Run vs. Long Run: Why OPEC Eventually Failed}

The long run tells a different story. Over time:

\begin{itemize}
    \item Producers outside OPEC expand oil exploration and build new extraction capacity.
    \item Consumers adopt conservation methods and more efficient technologies.
\end{itemize}

Thus, both long-run supply and demand become \textbf{more elastic}.  
As panel (b) of Figure indicated, the same shift in supply from \(S_1\) to \(S_2\) leads to only a small increase in price.

\end{frame}

%==========================================================
\begin{frame}{Short Run vs. Long Run: Why OPEC Eventually Failed}

\begin{tcolorbox}[colframe=red!80!black, colback=white, title=\textbf{Key Insight: Cartel Power Weakens Over Time}]
\begin{itemize}
    \item In the short run: Inelastic curves → large price increases → high OPEC revenue.
    \item In the long run: Elastic curves → small price increases → reduced profitability.
\end{itemize}
OPEC learned that raising prices is easier in the short run than in the long run.
\end{tcolorbox}
\end{frame}
%==========================================================
\begin{frame}{Does Drug Interdiction Increase or Decrease Drug-Related Crime?}

Illegal drug use causes many social problems, including addiction and drug-related crime.  
The U.S. government spends billions each year on drug interdiction to reduce the flow of illegal drugs.

To analyze this policy, we consider:
\begin{itemize}
    \item whether the supply or demand curve shifts,
    \item the direction of the shift,
    \item the resulting changes in equilibrium price and quantity.
\end{itemize}

\begin{tcolorbox}[colframe=blue!70!black, colback=white, title=\textbf{Effect of Drug Interdiction}]
Interdiction targets \textbf{sellers}, not users.  
By raising the cost of smuggling and reducing available supply, the supply curve shifts left from \(S_1\) to \(S_2\).
\end{tcolorbox}

\end{frame}

%==========================================================
\begin{frame}{Drug Interdiction}

\begin{columns}

\column{0.5\textwidth}
Equilibrium outcomes:
\begin{itemize}
    \item Price rises from \(P_1\) to \(P_2\),
    \item Quantity falls from \(Q_1\) to \(Q_2\).
\end{itemize}

Thus, \textbf{drug use decreases}.

\column{0.5\textwidth}
\begin{center}
    \includegraphics[width=\textwidth]{pictures/chap5/t18.png}
\end{center}

\end{columns}

\end{frame}
%==========================================================
\begin{frame}{Why Drug Interdiction May Increase Crime}

A key question: What happens to the \textbf{total amount of money drug users pay}?  
This depends on the elasticity of demand for drugs.

\begin{tcolorbox}[colframe=red!80!black, colback=white, title=\textbf{Demand for Drugs Is Inelastic}]
Because addiction makes consumers unresponsive to price changes, the demand curve is steep.  
When price rises, total revenue \(P \times Q\) \textbf{increases}.
\end{tcolorbox}

Consequences:
\begin{itemize}
    \item Drug users now spend more to support their addiction.
    \item Many addicts may resort to theft or violent crime to obtain money.
\end{itemize}

\textbf{Thus, drug interdiction may reduce drug use but increase drug-related crime.} Some analysts argue for an alternative approach: reducing the \textbf{demand} for drugs.

\end{frame}

%==========================================================
\begin{frame}{Demand Reduction Through Drug Education}

\begin{columns}

\column{0.5\textwidth}
Successful drug education shifts the demand curve left from \(D_1\) to \(D_2\).
Results:
\begin{itemize}
    \item Equilibrium quantity falls from \(Q_1\) to \(Q_2\),
    \item Price falls from \(P_1\) to \(P_2\),
    \item Total revenue \(P \times Q\) also falls.
\end{itemize}

\column{0.5\textwidth}
\begin{center}
    \includegraphics[width=\textwidth]{pictures/chap5/T19.png}
\end{center}

    
\end{columns}

\end{frame}
%==========================================================
\begin{frame}{Demand Reduction Through Drug Education}

\begin{tcolorbox}[colframe=blue!70!black, colback=white, title=\textbf{Policy Insight}]
Unlike interdiction, demand reduction lowers both \textbf{drug use} and \textbf{drug-related crime}.
\end{tcolorbox}

Long-run considerations:
\begin{itemize}
    \item Demand may be more elastic over time.
    \item Interdiction may increase crime in the short run but decrease it in the long run.
\end{itemize}
\end{frame}
%==========================================================
%==========================================================
%==========================================================
%==========================================================
%==========================================================
%==========================================================
%==========================================================
%==========================================================
%==========================================================
%==========================================================
%==========================================================
\end{document}