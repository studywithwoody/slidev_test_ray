\documentclass{beamer}
\usepackage{graphicx} % Required for inserting images
\usepackage{amsmath}
\usepackage[most]{tcolorbox}
\usepackage{lmodern}
\usepackage{mathabx}

\usetheme{Madrid} % 可選其他主題:e.g., Warsaw, Berkeley, etc.
\usecolortheme{default}
\setbeamertemplate{caption}[numbered]% Number float-like environments
% Customize the caption
\setbeamerfont{caption}{size=\footnotesize}
% \setbeamercolor{caption}{fg=blue}
% \setbeamercolor{caption name}{fg=red}

% 每章節開始時自動產生章節頁
\AtBeginSection[]
{
  \begin{frame}
    \frametitle{Table of Contents}
    \tableofcontents[currentsection]
  \end{frame}
}


\title{Mankiw's Principles of Economics}
\subtitle{chap26: Saving, Investment, and the Financial System}
\author{Hsu Chun-Wei}
\date{July 2025}

\begin{document}

\maketitle

% 目錄頁
\begin{frame}
  \frametitle{Table of Contents}
  \tableofcontents
\end{frame}

%==========================================================
\begin{frame}{Why Do We Need the Financial System?}

Imagine you have just graduated from college with a degree in economics and decide to start your own business—an economic forecasting firm. Before earning any revenue, you must first invest in capital goods such as computers, desks, chairs, and office equipment.

\vspace{0.5em}

These items are forms of \textbf{capital} that the firm uses to produce and sell services. However, most entrepreneurs do not have enough personal savings to finance these investments on their own.

\vspace{0.5em}

There are two main options:
\begin{itemize}
    \item Use personal savings.
    \item Obtain funds from others who are willing to provide financial resources.
\end{itemize}

\end{frame}

%==========================================================
\begin{frame}{Saving, Investment, and the Financial System}

Entrepreneurs often finance their investments by borrowing money or by offering a share of future profits to investors. In both cases, the investment is funded by \textbf{someone else's saving}.

\begin{tcolorbox}[colframe=blue!70!black, colback=white, title=\textbf{Definition: Financial System}]
The financial system consists of institutions that help match one person's saving with another person's investment.
\end{tcolorbox}

Saving and investment are crucial for long-run economic growth:
\begin{itemize}
    \item Higher saving allows more investment in capital.
    \item More capital increases productivity.
    \item Higher productivity raises living standards.
\end{itemize}

At any point in time:
\begin{itemize}
    \item Some people want to save part of their income.
    \item Others want to borrow to invest in new businesses.
\end{itemize}

\end{frame}

%==========================================================
\begin{frame}{What This Chapter Examines}

This chapter explains how the financial system coordinates saving and investment in the economy.

\begin{tcolorbox}[colframe=green!70!black, colback=white, title=\textbf{Chapter Overview}]
\begin{itemize}
    \item Financial institutions in the economy.
    \item The relationship between the financial system, saving, and investment.
    \item A model of supply and demand for funds in financial markets.
\end{itemize}
\end{tcolorbox}

In this model:
\begin{itemize}
    \item The \textbf{interest rate} is the price that balances saving and investment.
    \item Government policies can affect interest rates and resource allocation.
\end{itemize}

\end{frame}

%==========================================================
\section{26-1 Financial Institutions in the U.S. Economy}
%==========================================================

\begin{frame}{Financial Institutions in the U.S. Economy}

At the broadest level, the financial system moves the economy's scarce resources from \textbf{savers} to \textbf{borrowers}.

\begin{tcolorbox}[colframe=red!80!black, colback=white, title=\textbf{Key Idea}]
The financial system channels funds from people who spend less than they earn (savers) to people who spend more than they earn (borrowers).
\end{tcolorbox}

\begin{itemize}
    \item \textbf{Savers} provide funds with the expectation of earning interest in the future.
    \item \textbf{Borrowers} demand funds to finance purchases such as education, housing, or starting a business.
    \item Both sides agree to exchange money today for repayment with interest later.
\end{itemize}

\end{frame}

%==========================================================
\begin{frame}{Why Do Savers and Borrowers Use the Financial System?}

People participate in the financial system for different reasons.

\begin{tcolorbox}[colframe=blue!70!black, colback=white, title=\textbf{Motivations}]
\begin{itemize}
    \item Savers want to prepare for future needs such as retirement or education expenses.
    \item Borrowers want to invest in large purchases or income-generating activities.
\end{itemize}
\end{tcolorbox}

The financial system allows:
\begin{itemize}
    \item Savers to earn a return on their funds.
    \item Borrowers to access resources before they have earned enough income.
\end{itemize}

\end{frame}

%==========================================================
\begin{frame}{Types of Financial Institutions}

The financial system is composed of institutions that help coordinate savers and borrowers.

\begin{tcolorbox}[colframe=green!70!black, colback=white, title=\textbf{Two Main Categories}]
Financial institutions can be divided into:
\begin{itemize}
    \item \textbf{Financial Markets}
    \item \textbf{Financial Intermediaries}
\end{itemize}
\end{tcolorbox}

In the following sections, we examine each category and how it helps allocate resources in the economy.

\end{frame}

%==========================================================
\subsection{Financial Markets}
%==========================================================
\begin{frame}{Financial Markets}

Financial markets are institutions through which people who want to save can directly supply funds to people who want to borrow.

\begin{tcolorbox}[colframe=red!80!black, colback=white, title=\textbf{Key Definition}]
Financial markets allow savers and borrowers to interact directly without a financial intermediary.
\end{tcolorbox}

The two most important financial markets in the economy are:
\begin{itemize}
    \item The \textbf{bond market}
    \item The \textbf{stock market}
\end{itemize}

\end{frame}

%==========================================================
\begin{frame}{The Bond Market}

When a company or government wants to borrow money, it can do so by selling bonds to the public.

\begin{tcolorbox}[colframe=blue!70!black, colback=white, title=\textbf{Bond}]
A bond is a certificate of indebtedness
\end{tcolorbox}

A bond specifies:
\begin{itemize}
    \item The \textbf{principal} (the amount borrowed)
    \item The \textbf{interest rate}
    \item The \textbf{date of maturity} (when the loan is repaid)
\end{itemize}

Bond buyers:
\begin{itemize}
    \item Receive periodic interest payments
    \item Can hold the bond until maturity or sell it earlier
\end{itemize}

\end{frame}

%==========================================================
\begin{frame}{Bond Characteristics: Term and Credit Risk}

Bonds differ in several important ways.

\begin{tcolorbox}[colframe=green!70!black, colback=white, title=\textbf{1. Term}]
The term of a bond is the length of time until it matures.
\end{tcolorbox}

\begin{itemize}
    \item Short-term bonds mature in a few months or years.
    \item Long-term bonds may mature in 30 years or more.
    \item Long-term bonds are riskier and usually pay higher interest rates.
\end{itemize}

\begin{tcolorbox}[colframe=orange!80!black, colback=white, title=\textbf{2. Credit Risk}]
Credit risk is the probability that the borrower will fail to repay interest or principal.
\end{tcolorbox}

\begin{itemize}
    \item Failure to repay is called \textbf{default}.
    \item Higher credit risk $\Rightarrow$ higher interest rate.
    \item U.S. government bonds are considered very safe.
\end{itemize}

\end{frame}

%==========================================================
\begin{frame}{Bond Characteristics: Tax Treatment}

\begin{tcolorbox}[colframe=purple!70!black, colback=white, title=\textbf{3. Tax Treatment}]
Tax treatment refers to how the tax laws treat the interest earned on a bond.
\end{tcolorbox}

\begin{itemize}
    \item Interest on most bonds is taxable income.
    \item \textbf{Municipal bonds} are issued by state and local governments.
    \item Interest on municipal bonds is exempt from federal income tax.
\end{itemize}

Because of this tax advantage:
\begin{itemize}
    \item Municipal bonds usually pay lower interest rates.
\end{itemize}

\end{frame}

%==========================================================
\begin{frame}{The Stock Market}

Another way for a firm to raise funds is to sell stock.

\begin{tcolorbox}[colframe=red!80!black, colback=white, title=\textbf{Stock}]
Stock represents ownership in a firm and a claim on the profits the firm earns.
\end{tcolorbox}

\begin{itemize}
    \item Each share represents partial ownership of the company.
    \item Stockholders benefit when the firm is profitable.
    \item Profits may be paid as dividends or reflected in higher stock prices.
\end{itemize}

\end{frame}

%==========================================================
\begin{frame}{Equity Finance vs Debt Finance}

Firms can raise funds in two main ways.

\begin{tcolorbox}[colframe=blue!70!black, colback=white, title=\textbf{Types of Finance}]
\begin{itemize}
    \item \textbf{Equity finance}: selling stock
    \item \textbf{Debt finance}: selling bonds
\end{itemize}
\end{tcolorbox}

Key differences:
\begin{itemize}
    \item Stockholders are \textbf{owners} of the firm.
    \item Bondholders are \textbf{creditors} of the firm.
    \item Bondholders are paid before stockholders if the firm faces financial trouble.
\end{itemize}

\end{frame}

%==========================================================
\begin{frame}{Risk and Return of Stocks}

Stocks and bonds differ in both risk and return.

\begin{tcolorbox}[colframe=orange!80!black, colback=white, title=\textbf{Risk and Return}]
Compared to bonds, stocks offer higher risk and potentially higher return.
\end{tcolorbox}

\begin{itemize}
    \item Stockholders benefit more when firms are highly profitable.
    \item Stockholders are last in line if the firm goes bankrupt.
\end{itemize}

After stocks are issued:
\begin{itemize}
    \item Shares trade on organized stock exchanges.
    \item The firm receives no money when shares trade in secondary markets.
\end{itemize}

\end{frame}

%==========================================================
\begin{frame}{Stock Prices}

Stock prices are determined by supply and demand.

\begin{tcolorbox}[colframe=green!70!black, colback=white, title=\textbf{Stock Prices}]
The demand for a stock reflects people's expectations of a firm's future profitability.
\end{tcolorbox}

\begin{itemize}
    \item Optimism about future profits raises stock prices.
    \item Pessimism about future profits lowers stock prices.
\end{itemize}

\end{frame}

%==========================================================
\begin{frame}{Stock Indexes}

\begin{tcolorbox}[colframe=purple!70!black, colback=white, title=\textbf{Stock Index}]
A stock index measures the average price of a group of stocks.
\end{tcolorbox}

Examples:
\begin{itemize}
    \item Dow Jones Industrial Average
    \item Standard \& Poor's 500 Index
\end{itemize}

Stock indexes are often used as indicators of future economic conditions.

\end{frame}
%==========================================================
\subsection{Key Numbers for Stock Watchers}
%==========================================================
\begin{frame}{Key Numbers for Stock Watchers}

When following the stock of a company, investors focus on several key numbers.

\begin{tcolorbox}[colframe=red!80!black, colback=white, title=\textbf{Key Idea}]
Stock prices and related indicators summarize information about a firm's performance and future prospects.
\end{tcolorbox}

The three most important numbers are:
\begin{itemize}
    \item Price
    \item Dividend
    \item Price-Earnings Ratio (P/E)
\end{itemize}

\end{frame}

%==========================================================
\begin{frame}{Price and Dividend}

\begin{tcolorbox}[colframe=blue!70!black, colback=white, title=\textbf{Price}]
The price of a stock is the amount at which a share trades in the market.
\end{tcolorbox}

\begin{itemize}
    \item Reflects supply and demand for the stock.
    \item Responds quickly to new information.
    \item News services may report last price, previous close, high, and low.
\end{itemize}

\begin{tcolorbox}[colframe=green!70!black, colback=white, title=\textbf{Dividend}]
A dividend is the portion of a firm's profits paid to stockholders.
\end{tcolorbox}

\begin{itemize}
    \item Profits not paid out are called \textbf{retained earnings}.
    \item Dividend yield = dividend as a percentage of stock price.
\end{itemize}

\end{frame}

%==========================================================
\begin{frame}{Price-Earnings Ratio (P/E)}

\begin{tcolorbox}[colframe=purple!70!black, colback=white, title=\textbf{Price-Earnings Ratio}]
The P/E ratio equals the price of a stock divided by earnings per share.
\end{tcolorbox}

\begin{itemize}
    \item Earnings per share = total earnings / number of shares outstanding.
    \item A higher P/E often reflects higher expected future earnings.
    \item A lower P/E may indicate lower expected growth or undervaluation.
\end{itemize}

Historically, the typical P/E ratio is about 15.

\begin{tcolorbox}[colframe=orange!80!black, colback=white, title=\textbf{Interpretation}]
Stock prices reflect expectations about future profitability.
\end{tcolorbox}

\end{frame}
%==========================================================
\subsection{Financial Intermediaries}
%==========================================================
\begin{frame}{Financial Intermediaries}

Financial intermediaries are institutions through which savers can \textbf{indirectly} provide funds to borrowers.

\begin{tcolorbox}[colframe=red!80!black, colback=white, title=\textbf{Key Definition}]
Financial intermediaries stand between savers and borrowers and help channel funds in the economy.
\end{tcolorbox}

The two most important financial intermediaries are:
\begin{itemize}
    \item Banks
    \item Mutual funds
\end{itemize}

\end{frame}

%==========================================================
\begin{frame}{Why Do Small Businesses Rely on Banks?}

Large corporations can raise funds by selling stocks and bonds, but small businesses usually cannot.

\begin{tcolorbox}[colframe=blue!70!black, colback=white, title=\textbf{Key Idea}]
Small businesses typically finance expansion by borrowing from banks.
\end{tcolorbox}

\begin{itemize}
    \item Stocks and bonds are harder to sell for small or unfamiliar firms.
    \item Banks specialize in evaluating and lending to local borrowers.
\end{itemize}

\end{frame}

%==========================================================
\begin{frame}{Banks as Financial Intermediaries}

Banks are the most familiar financial intermediaries.

\begin{tcolorbox}[colframe=green!70!black, colback=white, title=\textbf{Primary Role of Banks}]
Banks take deposits from savers and use those deposits to make loans to borrowers.
\end{tcolorbox}

\begin{itemize}
    \item Banks pay interest on deposits.
    \item Banks charge higher interest on loans.
    \item The difference covers costs and generates profit.
\end{itemize}

\end{frame}

%==========================================================
\begin{frame}{Banks and the Medium of Exchange}

Banks play a second important role in the economy.

\begin{tcolorbox}[colframe=orange!80!black, colback=white, title=\textbf{Medium of Exchange}]
A medium of exchange is an item people use to easily engage in transactions.
\end{tcolorbox}

\begin{itemize}
    \item Bank deposits can be accessed by checks and debit cards.
    \item This makes bank deposits convenient for daily transactions.
    \item This role distinguishes banks from other financial institutions.
\end{itemize}

\end{frame}

%==========================================================
\begin{frame}{Mutual Funds}

A mutual fund is a financial intermediary that pools money from many investors.

\begin{tcolorbox}[colframe=red!80!black, colback=white, title=\textbf{Mutual Fund}]
A mutual fund sells shares to the public and uses the proceeds to buy a diversified portfolio of stocks, bonds, or both.
\end{tcolorbox}

\begin{itemize}
    \item Shareholders share both the risk and return of the portfolio.
    \item Gains occur if the portfolio rises in value.
    \item Losses occur if the portfolio falls in value.
\end{itemize}

\end{frame}

%==========================================================
\begin{frame}{Mutual Funds and Diversification}

The primary advantage of mutual funds is diversification.

\begin{tcolorbox}[colframe=blue!70!black, colback=white, title=\textbf{Diversification}]
Diversification reduces risk by spreading investments across many assets.
\end{tcolorbox}

\begin{itemize}
    \item Individual stocks and bonds are risky on their own.
    \item A diversified portfolio reduces exposure to any single firm.
    \item Mutual funds make diversification possible with small amounts of money.
\end{itemize}

\end{frame}

%==========================================================
\begin{frame}{Professional Management}

Another claimed advantage of mutual funds is professional management.

\begin{tcolorbox}[colframe=orange!80!black, colback=white, title=\textbf{Professional Management}]
Mutual funds give investors access to professional money managers.
\end{tcolorbox}

\begin{itemize}
    \item Managers research firms and select promising investments.
    \item Funds charge fees, usually between 0.5\% and 2.0\% per year.
\end{itemize}

However, many economists are skeptical:
\begin{itemize}
    \item Stock prices already reflect available information.
    \item It is difficult to consistently "beat the market."
\end{itemize}

\end{frame}

%==========================================================
\begin{frame}{Index Funds}

Some mutual funds follow a passive investment strategy.

\begin{tcolorbox}[colframe=green!70!black, colback=white, title=\textbf{Index Fund}]
An index fund buys all the stocks in a given stock index.
\end{tcolorbox}

\begin{itemize}
    \item Index funds trade very infrequently.
    \item Lower management and transaction costs.
    \item Often outperform actively managed mutual funds on average.
\end{itemize}

\end{frame}

%==========================================================
\section{26-2 Saving and Investment in the National Income Accounts}
%==========================================================
\begin{frame}{Saving and Investment in the National Income Accounts}

Events in the financial system are central to understanding developments in the overall economy.

\begin{tcolorbox}[colframe=red!80!black, colback=white, title=\textbf{Key Idea}]
Financial institutions coordinate saving and investment, which are key determinants of long-run economic growth.
\end{tcolorbox}

\begin{itemize}
    \item Saving and investment affect GDP and living standards.
    \item Financial markets help allocate resources to productive uses.
    \item Macroeconomists study how events and policies affect saving and investment.
\end{itemize}

\end{frame}

%==========================================================
\begin{frame}{Accounting Identities and National Income}

This section focuses on accounting rather than individual behavior.

\begin{tcolorbox}[colframe=blue!70!black, colback=white, title=\textbf{Accounting}]
Accounting describes how economic variables are defined and added up.
\end{tcolorbox}

\begin{itemize}
    \item National income accounting summarizes the economy as a whole.
    \item It includes GDP and related statistics.
    \item An \textbf{identity} is an equation that must be true by definition.
\end{itemize}

These identities clarify how saving and investment are related in the macroeconomy.

\end{frame}

%==========================================================
\begin{frame}{GDP and Its Components}

GDP is both total income and total expenditure in the economy.

\begin{tcolorbox}[colframe=red!80!black, colback=white, title=\textbf{GDP Identity}]
GDP ($Y$) is divided into four components:
\[
Y = C + I + G + NX
\]
\end{tcolorbox}

\begin{itemize}
    \item $C$: Consumption
    \item $I$: Investment
    \item $G$: Government purchases
    \item $NX$: Net exports
\end{itemize}

This equation is an \textbf{identity} and must always hold by definition.

\end{frame}

%==========================================================
\begin{frame}{Closed Economy Assumption}

To simplify the analysis, we assume a closed economy.

\begin{tcolorbox}[colframe=blue!70!black, colback=white, title=\textbf{Closed Economy}]
A closed economy does not engage in international trade or international borrowing and lending.
\end{tcolorbox}

\begin{itemize}
    \item Imports and exports are zero.
    \item Net exports ($NX$) = 0.
\end{itemize}

Therefore, the GDP identity becomes:
\[
Y = C + I + G
\]

\end{frame}
    
%==========================================================
\begin{frame}{From GDP to Saving}

Starting from the closed-economy identity:
\[
Y = C + I + G
\]

Subtract consumption and government purchases from both sides:
\[
Y - C - G = I
\]

\begin{tcolorbox}[colframe=green!70!black, colback=white, title=\textbf{National Saving}]
National saving ($S$) is the income remaining after paying for consumption and government purchases.
\end{tcolorbox}

Thus:
\[
S = I
\]

\end{frame}

%==========================================================
\begin{frame}{Saving Equals Investment}

\begin{tcolorbox}[colframe=orange!80!black, colback=white, title=\textbf{Key Result}]
For the economy as a whole, saving must equal investment.
\end{tcolorbox}

\begin{itemize}
    \item This is an accounting identity, not a behavioral assumption.
    \item It holds because of how the variables are defined.
    \item Financial markets coordinate saving and investment.
\end{itemize}

\end{frame}

%==========================================================
\begin{frame}{Private and Public Saving}

Let $T$ denote taxes minus transfer payments.

National saving can be written as:
\[
S = Y - C - G
\]

or equivalently:
\[
S = (Y - T - C) + (T - G)
\]

\begin{tcolorbox}[colframe=purple!70!black, colback=white, title=\textbf{Components of Saving}]
\begin{itemize}
    \item Private saving = $Y - T - C$
    \item Public saving = $T - G$
\end{itemize}
\end{tcolorbox}

\end{frame}

%==========================================================
\begin{frame}{Budget Balance and Financial Markets}

\begin{itemize}
    \item If $T > G$, the government runs a \textbf{budget surplus}.
    \item Public saving ($T - G$) is positive.
    \item If $T < G$, the government runs a \textbf{budget deficit}.
    \item Public saving is negative.
\end{itemize}

\begin{tcolorbox}[colframe=red!70!black, colback=white, title=\textbf{Big Picture}]
Financial markets and intermediaries take national saving and direct it toward national investment.
\end{tcolorbox}

\end{frame}

%==========================================================
\begin{frame}{The Meaning of Saving and Investment}

The terms \textit{saving} and \textit{investment} are often used casually and interchangeably.

\begin{tcolorbox}[colframe=red!80!black, colback=white, title=\textbf{Key Distinction}]
Macroeconomists use the terms saving and investment carefully and distinctly.
\end{tcolorbox}

\begin{itemize}
    \item Everyday language ≠ economic definitions
    \item National income accounting uses precise meanings
\end{itemize}

\end{frame}

%==========================================================
\begin{frame}{Saving in Macroeconomics}

Consider an individual whose income exceeds consumption.

\begin{tcolorbox}[colframe=blue!70!black, colback=white, title=\textbf{Saving}]
Saving is income that is not spent on consumption or government purchases.
\end{tcolorbox}

\begin{itemize}
    \item Depositing money in a bank is saving.
    \item Buying stocks or bonds is also saving.
    \item These actions add to the nation’s saving.
\end{itemize}

Even if individuals think they are "investing," economists classify these actions as saving.

\end{frame}

%==========================================================
\begin{frame}{Investment in Macroeconomics}

In macroeconomics, investment has a specific meaning.

\begin{tcolorbox}[colframe=green!70!black, colback=white, title=\textbf{Investment}]
Investment refers to the purchase of new capital goods.
\end{tcolorbox}

Examples:
\begin{itemize}
    \item Firms building new factories or buying equipment
    \item Households purchasing new homes
\end{itemize}

Financial transactions such as buying stocks or bonds are \textbf{not} investment in the macroeconomic sense.

\end{frame}

%==========================================================
\begin{frame}{Saving Equals Investment}

\begin{tcolorbox}[colframe=orange!80!black, colback=white, title=\textbf{Key Result}]
For the economy as a whole, saving equals investment ($S = I$).
\end{tcolorbox}

\begin{itemize}
    \item This does not have to be true for each individual.
    \item Some households save more than they invest.
    \item Others invest more than they save and borrow the difference.
\end{itemize}

Banks and financial institutions allow one person’s saving to finance another person’s investment.

\end{frame}

%==========================================================
%==========================================================
%==========================================================
%==========================================================
%==========================================================
%==========================================================
%==========================================================
%==========================================================



\end{document}