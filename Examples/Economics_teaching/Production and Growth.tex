\documentclass{beamer}
\usepackage{graphicx} % Required for inserting images
\usepackage{amsmath}
\usepackage[most]{tcolorbox}
\usepackage{lmodern}
\usepackage{mathabx}

\usetheme{Madrid} % 可選其他主題:e.g., Warsaw, Berkeley, etc.
\usecolortheme{default}
\setbeamertemplate{caption}[numbered]% Number float-like environments
% Customize the caption
\setbeamerfont{caption}{size=\footnotesize}
% \setbeamercolor{caption}{fg=blue}
% \setbeamercolor{caption name}{fg=red}

% 每章節開始時自動產生章節頁
\AtBeginSection[]
{
  \begin{frame}
    \frametitle{Table of Contents}
    \tableofcontents[currentsection]
  \end{frame}
}


\title{Mankiw's Principles of Economics}
\subtitle{chap 25: Production and Growth}
\author{Hsu Chun-Wei}
\date{July 2025}

\begin{document}

\maketitle

% 目錄頁
\begin{frame}
  \frametitle{Table of Contents}
  \tableofcontents
\end{frame}
%==========================================================
\begin{frame}{Living Standards and Economic Growth}

There is tremendous variation in the standard of living around the world.  
Average income in rich countries can be more than ten times higher than in poor countries, and these income differences translate into large differences in quality of life.

\begin{tcolorbox}[colframe=red!80!black, colback=white, title=\textbf{Key Observation: Income and Quality of Life}]
Higher income is associated with better nutrition, safer housing, improved healthcare, longer life expectancy, and greater access to goods such as cars, phones, and televisions.
\end{tcolorbox}

\end{frame}
%==========================================================
\begin{frame}{Living Standards and Economic Growth}

Even within a single country, living standards change over time.  
In the United States, real GDP per person has grown by about 2 percent per year over the past century.

\begin{tcolorbox}[colframe=blue!70!black, colback=white, title=\textbf{Why Small Growth Rates Matter}]
\begin{itemize}
    \item A 2\% annual growth rate doubles income roughly every 35 years.
    \item Long-run growth explains why each generation enjoys higher living standards than previous generations.
\end{itemize}
\end{tcolorbox}

\end{frame}

%==========================================================
\begin{frame}{Divergent Growth Experiences and Key Questions}

Economic growth rates vary substantially across countries.  
Some East Asian economies have experienced rapid growth, while others have seen little or no improvement in income levels.

\begin{tcolorbox}[colframe=red!80!black, colback=white, title=\textbf{Contrasting Growth Experiences}]
\begin{itemize}
    \item Some economies grow at 7\% per year, doubling income every 10 years.
    \item Rapid growth can transform a country from poor to rich within one generation.
    \item In contrast, some countries have experienced stagnant or declining income levels.
\end{itemize}
\end{tcolorbox}
\end{frame}

%==========================================================
\begin{frame}{Divergent Growth Experiences and Key Questions}

These differences raise fundamental questions in macroeconomics:
\begin{itemize}
    \item Why do living standards differ so much across countries?
    \item How can rich countries maintain high standards of living?
    \item What policies can help poor countries achieve faster growth?
\end{itemize}

\begin{tcolorbox}[colframe=blue!70!black, colback=white, title=\textbf{Focus of This Chapter}]
\begin{itemize}
    \item Real GDP per person as a measure of economic prosperity
    \item Productivity: output produced per hour of work
    \item The relationship between productivity and economic policy
\end{itemize}
\end{tcolorbox}

\end{frame}
%==========================================================
\section{25-1 Economic Growth around the World}
%==========================================================
\begin{frame}{Economic Growth around the World}

To study long-run economic growth, economists compare real GDP per person across countries and over long periods of time.

\begin{tcolorbox}[colframe=red!80!black, colback=white, title=\textbf{Real GDP per Person}]
Real GDP per person measures average income and is a key indicator of a country’s standard of living.
\end{tcolorbox}

International data show enormous differences in living standards:
\begin{itemize}
    \item Average income in the United States is several times higher than in China.
    \item Average income in the United States is more than ten times higher than in India.
    \item The poorest countries today have income levels similar to rich countries many decades ago.
\end{itemize}

\end{frame}
%==========================================================
\begin{frame}{Economic Growth around the World}

\begin{center}
    \includegraphics[width=0.8\textwidth]{pictures/chap 25/T1.png}
\end{center}

\end{frame}
%==========================================================
\begin{frame}{Growth Rates and Long-Run Trends}

The growth rate of real GDP per person measures how rapidly living standards improve over time.

\begin{tcolorbox}[colframe=blue!70!black, colback=white, title=\textbf{Understanding Growth Rates}]
\begin{itemize}
    \item Growth rates represent long-run averages, not year-by-year changes.
    \item Short-run fluctuations are ignored to reveal long-run trends.
    \item Even small differences in growth rates lead to large income gaps over time.
\end{itemize}
\end{tcolorbox}

For example:
\begin{itemize}
    \item A growth rate of about 1.8\% per year can raise income more than tenfold over a century.
    \item Countries with growth above 2\% per year experience dramatic improvements in living standards.
\end{itemize}

\end{frame}

%==========================================================
\begin{frame}{Changing Income Rankings across Countries}

Countries are ranked very differently today than they were a century ago due to differences in growth rates.

\begin{tcolorbox}[colframe=red!80!black, colback=white, title=\textbf{Key Patterns from the Data}]
\begin{itemize}
    \item Some countries experienced rapid growth and moved from poor to rich.
    \item Others grew slowly and fell behind in relative income rankings.
    \item High income today does not guarantee high income forever.
\end{itemize}
\end{tcolorbox}

These facts lead to fundamental macroeconomic questions:
\begin{itemize}
    \item Why do some countries grow faster than others?
    \item Why do some nations catch up while others remain poor?
\end{itemize}

These questions motivate the study of long-run economic growth.

\end{frame}

%==========================================================
\section{25-2 Productivity: Its Role and Determinants}
%==========================================================
\begin{frame}{Productivity and Living Standards}

Why do living standards vary so much around the world?  
In one word, the answer is \textbf{productivity}.

\begin{tcolorbox}[colframe=red!80!black, colback=white, title=\textbf{Productivity}]
Productivity is the quantity of goods and services produced from each unit of labor input.
\end{tcolorbox}

Countries with higher productivity:
\begin{itemize}
    \item Produce more goods and services.
    \item Earn higher income.
    \item Enjoy higher standards of living.
\end{itemize}

Understanding differences in productivity is key to understanding differences in income and economic growth.

\end{frame}

%==========================================================
\begin{frame}{Why Productivity Is So Important}

To understand productivity, consider a simple economy based on \textit{Robinson Crusoe}.

Crusoe lives alone and produces everything he consumes:
\begin{itemize}
    \item He catches fish.
    \item He grows vegetables.
    \item He makes clothing.
\end{itemize}

\begin{tcolorbox}[colframe=blue!70!black, colback=white, title=\textbf{Key Insight from Crusoe’s Economy}]
Crusoe’s standard of living depends entirely on his productivity.
\end{tcolorbox}

\begin{itemize}
    \item Higher productivity means more goods or more leisure.
    \item Better methods raise productivity and improve well-being.
\end{itemize}

\end{frame}

%==========================================================
\begin{frame}{Productivity and Nations}

What is true for Robinson Crusoe is also true for entire economies.

\begin{tcolorbox}[colframe=red!80!black, colback=white, title=\textbf{GDP and Productivity}]
\begin{itemize}
    \item GDP measures total income and total output.
    \item An economy’s income equals its production.
    \item Higher productivity leads to higher GDP per person.
\end{itemize}
\end{tcolorbox}

Countries with higher productivity:
\begin{itemize}
    \item Produce more goods and services per worker.
    \item Enjoy higher living standards.
    \item Experience faster long-run growth.
\end{itemize}

This reflects one of the \textit{Ten Principles of Economics}:  
A country’s standard of living depends on its ability to produce goods and services.

\end{frame}

%==========================================================
\begin{frame}{How Productivity Is Determined}

Productivity depends on many factors, not only for Robinson Crusoe but also for modern economies.

\begin{tcolorbox}[colframe=red!80!black, colback=white, title=\textbf{Four Determinants of Productivity}]
\begin{itemize}
    \item Physical capital
    \item Human capital
    \item Natural resources
    \item Technological knowledge
\end{itemize}
\end{tcolorbox}

In Crusoe’s economy:
\begin{itemize}
    \item More fishing poles increase output.
    \item Better skills improve efficiency.
    \item Abundant fish raise potential production.
    \item Better fishing techniques boost productivity.
\end{itemize}

Each of these factors has a direct counterpart in real-world economies.

\end{frame}

%==========================================================
\begin{frame}{Physical Capital per Worker}

Workers are more productive when they have more and better tools to work with.

\begin{tcolorbox}[colframe=blue!70!black, colback=white, title=\textbf{Physical Capital}]
Physical capital is the stock of equipment and structures used to produce goods and services.
\end{tcolorbox}

Examples of physical capital:
\begin{itemize}
    \item Tools and machinery
    \item Equipment and factories
    \item Infrastructure such as roads and buildings
\end{itemize}
\end{frame}

%==========================================================
\begin{frame}{Physical Capital per Worker}


\begin{tcolorbox}[colframe=red!80!black, colback=white, title=\textbf{Key Insight}]
Physical capital is a produced factor of production:  
it is itself the result of past production.
\end{tcolorbox}

An economy with more capital per worker can produce more output per worker and enjoy a higher standard of living.

\end{frame}

%==========================================================
\begin{frame}{Human Capital per Worker}

A second determinant of productivity is human capital.

\begin{tcolorbox}[colframe=blue!70!black, colback=white, title=\textbf{Human Capital}]
Human capital is the knowledge and skills that workers acquire through education, training, and experience.
\end{tcolorbox}

Human capital includes:
\begin{itemize}
    \item Education in schools and universities
    \item Job training and work experience
    \item Skills developed over a lifetime
\end{itemize}

\end{frame}

%==========================================================
\begin{frame}{Human Capital per Worker}

\begin{tcolorbox}[colframe=red!80!black, colback=white, title=\textbf{Key Insight}]
Like physical capital, human capital is a produced factor of production.
\end{tcolorbox}

Investment in human capital raises productivity and improves a nation’s standard of living.

\end{frame}

%==========================================================
\begin{frame}{Natural Resources per Worker}

A third determinant of productivity is natural resources.

\begin{tcolorbox}[colframe=blue!70!black, colback=white, title=\textbf{Natural Resources}]
Natural resources are inputs into production provided by nature, such as land, rivers, forests, and mineral deposits.
\end{tcolorbox}

Types of natural resources:
\begin{itemize}
    \item Renewable resources (e.g., forests)
    \item Nonrenewable resources (e.g., oil)
\end{itemize}

\end{frame}

%==========================================================
\begin{frame}{Natural Resources per Worker}

Differences in natural resources help explain differences in living standards across countries.

\begin{tcolorbox}[colframe=red!80!black, colback=white, title=\textbf{Important Qualification}]
Natural resources are not necessary for high productivity.  
Countries can achieve high living standards through trade and other determinants of productivity.
\end{tcolorbox}

\end{frame}

%==========================================================
\begin{frame}{Technological Knowledge}

A fourth determinant of productivity is technological knowledge.

\begin{tcolorbox}[colframe=blue!70!black, colback=white, title=\textbf{Technological Knowledge}]
Technological knowledge is society’s understanding of the best ways to produce goods and services.
\end{tcolorbox}

Technological progress allows:
\begin{itemize}
    \item The same number of workers to produce more output.
    \item Fewer workers to produce the same output.
    \item Labor to shift from one sector to another.
\end{itemize}

For example, advances in farming technology allow a small share of the population to produce enough food for the entire economy, freeing labor for other goods and services.

\end{frame}

%==========================================================
\begin{frame}{Technology and Human Capital}

Technological knowledge takes many forms.

\begin{tcolorbox}[colframe=red!80!black, colback=white, title=\textbf{Forms of Technological Knowledge}]
\begin{itemize}
    \item Common knowledge that spreads quickly.
    \item Proprietary knowledge known only to its creator.
    \item Temporarily protected knowledge through patents.
\end{itemize}
\end{tcolorbox}

\end{frame}

%==========================================================
\begin{frame}{Technology and Human Capital}

It is important to distinguish between technological knowledge and human capital.

\begin{tcolorbox}[colframe=blue!70!black, colback=white, title=\textbf{Key Distinction}]
\begin{itemize}
    \item Technological knowledge is society’s understanding of how the world works.
    \item Human capital is the resources devoted to transmitting this knowledge to workers.
\end{itemize}
\end{tcolorbox}

Workers’ productivity depends on both.

\end{frame}

%==========================================================
\begin{frame}{The Production Function}

Economists use a production function to describe the relationship between inputs and output.

\begin{tcolorbox}[colframe=blue!70!black, colback=white, title=\textbf{Production Function}]
\[
Y = A F(L, K, H, N)
\]
\end{tcolorbox}

where:
\begin{itemize}
    \item $Y$ = output
    \item $L$ = labor
    \item $K$ = physical capital
    \item $H$ = human capital
    \item $N$ = natural resources
    \item $A$ = level of technology
\end{itemize}

As technology improves, $A$ rises, allowing the economy to produce more output from the same inputs.

\end{frame}

%==========================================================
\begin{frame}{Constant Returns to Scale}

Many production functions exhibit constant returns to scale.

\begin{tcolorbox}[colframe=red!80!black, colback=white, title=\textbf{Constant Returns to Scale}]
If all inputs are multiplied by the same factor, output increases by the same factor.
\[
xY = A F(xL, xK, xH, xN)
\]
\end{tcolorbox}

Dividing both sides by $L$ gives:
\[
\frac{Y}{L} = A F\left(1, \frac{K}{L}, \frac{H}{L}, \frac{N}{L}\right)
\]
\end{frame}

%==========================================================
\begin{frame}{Constant Returns to Scale}

\[
\frac{Y}{L} = A F\left(1, \frac{K}{L}, \frac{H}{L}, \frac{N}{L}\right)
\]

\begin{tcolorbox}[colframe=blue!70!black, colback=white, title=\textbf{Key Insight}]
Output per worker depends on:
\begin{itemize}
    \item Physical capital per worker $(K/L)$
    \item Human capital per worker $(H/L)$
    \item Natural resources per worker $(N/L)$
    \item Technology $(A)$
\end{itemize}
\end{tcolorbox}

\end{frame}
%==========================================================
%==========================================================

\end{document}