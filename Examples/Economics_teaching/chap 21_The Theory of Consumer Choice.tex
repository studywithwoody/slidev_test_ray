\documentclass{beamer}
\usepackage{graphicx} % Required for inserting images
\usepackage{amsmath}
\usepackage[most]{tcolorbox}
\usepackage{lmodern}
\usepackage{mathabx}

\usetheme{Madrid} % 可選其他主題:e.g., Warsaw, Berkeley, etc.
\usecolortheme{default}
\setbeamertemplate{caption}[numbered]% Number float-like environments
% Customize the caption
\setbeamerfont{caption}{size=\footnotesize}
% \setbeamercolor{caption}{fg=blue}
% \setbeamercolor{caption name}{fg=red}

% 每章節開始時自動產生章節頁
\AtBeginSection[]
{
  \begin{frame}
    \frametitle{Table of Contents}
    \tableofcontents[currentsection]
  \end{frame}
}


\title{Mankiw's Principles of Economics}
\subtitle{chap 21: The Theory of Consumer Choice}
\author{Hsu Chun-Wei}
\date{July 2025}

\begin{document}

\maketitle

% 目錄頁
\begin{frame}
  \frametitle{Table of Contents}
  \tableofcontents
\end{frame}

%==========================================================
\section{21-1 The Budget Constraint: What the Consumer Can Afford}
%==========================================================
\begin{frame}{The Budget Constraint: What the Consumer Can Afford}

Most consumers would like to consume more goods than they actually do.
Their choices are limited by income.

\begin{tcolorbox}[colframe=red!80!black, colback=white, title=\textbf{Budget Constraint}]
A budget constraint shows the combinations of goods
that a consumer can afford given:
\begin{itemize}
    \item Income
    \item Prices of goods
\end{itemize}
\end{tcolorbox}

\end{frame}

%==========================================================
\begin{frame}{The Budget Constraint: What the Consumer Can Afford}

To simplify the analysis, consider a consumer who buys only two goods:
\textbf{pizza} and \textbf{Pepsi}.

Suppose:
\begin{itemize}
    \item Income = \$1{,}000
    \item Price of pizza = \$10
    \item Price of Pepsi = \$2
\end{itemize}

All consumption bundles that cost exactly \$1{,}000 lie on the budget constraint.

\end{frame}
%==========================================================
\begin{frame}{The Budget Constraint: What the Consumer Can Afford}

\begin{center}
    \includegraphics[width=0.8\textwidth]{pictures/chap21/T1.png}
\end{center}

\end{frame}

%==========================================================
\begin{frame}{Slope of the Budget Constraint}

The budget constraint shows the trade-off the consumer faces
between two goods.

\begin{tcolorbox}[colframe=red!80!black, colback=white, title=\textbf{Slope of the Budget Constraint}]
The slope of the budget constraint equals the
\textbf{relative price of the two goods}.
\end{tcolorbox}

In this example:
\[
\text{Slope} = \frac{\text{Price of pizza}}{\text{Price of Pepsi}} = \frac{10}{2} = 5
\]

This means that the opportunity cost of one pizza
is \textbf{5 liters of Pepsi}.

\end{frame}

%==========================================================
\begin{frame}{Slope of the Budget Constraint}

\begin{center}
    \includegraphics[width=0.4\textwidth]{pictures/chap21/T2.png}
\end{center}
The slope of the budget constraint measures
how much of one good the consumer must give up
to obtain one more unit of the other good.

\end{frame}

%==========================================================
\section{21-2 Preferences: What the Consumer Wants}
%==========================================================
\begin{frame}{Preferences: What the Consumer Wants}

The budget constraint shows what a consumer \textbf{can afford}.
However, consumer choice also depends on what the consumer \textbf{wants}.

\begin{tcolorbox}[colframe=red!80!black, colback=white, title=\textbf{Preferences}]
A consumer’s preferences describe how she ranks different
consumption bundles of goods.
\end{tcolorbox}

To analyze preferences, we again simplify the problem by assuming
that the consumer chooses between only two goods:
\textbf{pizza} and \textbf{Pepsi}.

\vspace{0.5em}

Preferences, together with the budget constraint,
determine the consumer’s optimal choice.

\end{frame}

%==========================================================
\begin{frame}{Representing Preferences with Indifference Curves}

We represent a consumer’s preferences using \textbf{indifference curves}.

\begin{tcolorbox}[colframe=red!80!black, colback=white, title=\textbf{Indifference Curve}]
An indifference curve shows all combinations of goods
that make the consumer equally satisfied.
\end{tcolorbox}

Along a given indifference curve, the consumer is willing to trade
one good for the other while remaining equally happy.
\end{frame}

%==========================================================
\begin{frame}{Representing Preferences with Indifference Curves}

\begin{center}
    \includegraphics[width=0.8\textwidth]{pictures/chap21/T3.png}
\end{center}

\end{frame}

%==========================================================
\begin{frame}{Representing Preferences with Indifference Curves}

\begin{tcolorbox}[colframe=blue!70!black, colback=white, title=\textbf{Marginal Rate of Substitution (MRS)}]
The slope of an indifference curve equals the \textbf{marginal rate of substitution (MRS)}:
the rate at which the consumer is willing to give up one good
to obtain more of the other.
\end{tcolorbox}

Because indifference curves are bowed inward,
the marginal rate of substitution diminishes as consumption changes.

Higher indifference curves represent higher levels of satisfaction
and are preferred to lower ones.

\end{frame}

%==========================================================
\begin{frame}{Four Properties of Indifference Curves}

Because indifference curves represent consumer preferences,
they share several common properties.

\begin{tcolorbox}[colframe=red!80!black, colback=white, title=\textbf{Property 1: Higher Is Better}]
Higher indifference curves are preferred to lower ones.
Consumers prefer more of both goods to less.
\end{tcolorbox}

\begin{tcolorbox}[colframe=blue!70!black, colback=white, title=\textbf{Property 2: Downward Sloping}]
Indifference curves slope downward.
If the consumer gives up some of one good,
she must receive more of the other to remain equally satisfied.
\end{tcolorbox}

\end{frame}

%==========================================================
\begin{frame}{Four Properties of Indifference Curves}


\begin{tcolorbox}[colframe=red!80!black, colback=white, title=\textbf{Property 3: Do Not Cross}]
Indifference curves cannot cross,
because crossing would violate the assumption
that consumers prefer more of both goods to less.
\end{tcolorbox}

\begin{tcolorbox}[colframe=red!80!black, colback=white, title=\textbf{Why Indifference Curves Cannot Cross}]
If two indifference curves crossed,
the consumer would be indifferent between two bundles
even though one bundle has more of both goods.
This contradicts the assumption that more is preferred to less.
\end{tcolorbox}

\end{frame}

%==========================================================
\begin{frame}{Four Properties of Indifference Curves}

\begin{center}
    \includegraphics[width=0.8\textwidth]{pictures/chap21/T4.png}
\end{center}

\end{frame}

%==========================================================
\begin{frame}{Four Properties of Indifference Curves}

\begin{tcolorbox}[colframe=blue!70!black, colback=white, title=\textbf{Property 4: Bowed Inward}]
Indifference curves are bowed inward,
reflecting a diminishing marginal rate of substitution (MRS).
\end{tcolorbox}

\begin{tcolorbox}[colframe=blue!70!black, colback=white, title=\textbf{Why Indifference Curves Are Bowed Inward}]
The slope of an indifference curve is the marginal rate of substitution (MRS).
Because consumers are more willing to give up a good
when they have a lot of it and less willing when they have little,
the MRS diminishes as consumption changes.
\end{tcolorbox}

\textbf{Key insight:}  
The shape of indifference curves reflects consumers’
willingness to substitute one good for another.

\end{frame}

%==========================================================
\begin{frame}{Four Properties of Indifference Curves}
    \begin{center}
        \includegraphics[width=0.8\textwidth]{pictures/chap21/T5.png}
    \end{center}
\end{frame}
%==========================================================
\begin{frame}{Two Extreme Examples: Perfect Substitutes}

When two goods are easy to substitute for each other,
the consumer is willing to trade them at a constant rate.

\begin{tcolorbox}[colframe=red!80!black, colback=white, title=\textbf{Perfect Substitutes}]
Two goods are perfect substitutes if the consumer
is always willing to trade one good for the other
at a fixed rate.
\end{tcolorbox}

\textbf{Example:} Nickels and dimes.  
The consumer cares only about total monetary value.

\begin{itemize}
    \item The marginal rate of substitution (MRS) is constant.
    \item Indifference curves are straight lines.
\end{itemize}

\begin{tcolorbox}[colframe=blue!70!black, colback=white, title=\textbf{Key Insight}]
A constant MRS implies straight-line indifference curves.
\end{tcolorbox}

\end{frame}
%==========================================================
\begin{frame}{Two Extreme Examples: Perfect Substitutes}
\begin{center}
    \includegraphics[width=0.7\textwidth]{pictures/chap21/T6.png}
\end{center}
\end{frame}
%==========================================================
\begin{frame}{Two Extreme Examples: Perfect Complements}

When two goods are useful only when consumed together,
the consumer values them in fixed proportions.

\begin{tcolorbox}[colframe=red!80!black, colback=white, title=\textbf{Perfect Complements}]
Two goods are perfect complements if the consumer
wants to consume them in fixed ratios.
\end{tcolorbox}

\textbf{Example:} Left shoes and right shoes.

\begin{itemize}
    \item Extra units of one good have no value without the other.
    \item Indifference curves are right angles.
\end{itemize}

\begin{tcolorbox}[colframe=blue!70!black, colback=white, title=\textbf{Key Insight}]
Right-angle indifference curves reflect zero substitutability.
\end{tcolorbox}

\end{frame}
%==========================================================
\begin{frame}{Two Extreme Examples: Perfect Complements}
\begin{center}
    \includegraphics[width=0.7\textwidth]{pictures/chap21/T7.png}
\end{center}
\end{frame}
%==========================================================
\section{21-3 Optimization: What the Consumer Chooses}
%==========================================================
\begin{frame}{Optimization: What the Consumer Chooses}

Consumer choice combines two elements:
\begin{itemize}
    \item The \textbf{budget constraint} (what the consumer can afford)
    \item The \textbf{indifference curves} (what the consumer wants)
\end{itemize}

\begin{tcolorbox}[colframe=red!80!black, colback=white, title=\textbf{Consumer's Goal}]
The consumer chooses the combination of goods
that lies on the \textbf{highest possible indifference curve}
while remaining on or below the budget constraint.
\end{tcolorbox}

The optimal choice is the point where the consumer achieves
the greatest satisfaction given her limited income.

\end{frame}
%==========================================================
\begin{frame}{Optimization: What the Consumer Chooses}
\begin{center}
    \includegraphics[width=0.8\textwidth]{pictures/chap21/T8.png}
\end{center}
\end{frame}
%==========================================================
\begin{frame}{The Consumer's Optimal Choice}

At the optimal consumption bundle,
the budget constraint is tangent to an indifference curve.

\begin{tcolorbox}[colframe=red!80!black, colback=white, title=\textbf{Tangency Condition}]
At the optimum:
\[
\text{MRS} = \frac{P_{\text{pizza}}}{P_{\text{Pepsi}}}
\]
\end{tcolorbox}

This condition means:
\begin{itemize}
    \item The rate at which the consumer is willing to trade goods
    \item Equals the rate at which the market allows goods to be traded
\end{itemize}

\begin{tcolorbox}[colframe=blue!70!black, colback=white, title=\textbf{Key Insight}]
If MRS were greater or smaller than the relative price,
the consumer could rearrange consumption to reach a higher indifference curve.
\end{tcolorbox}

\end{frame}

%==========================================================
\begin{frame}{MRS and Relative Prices}

The relative price measures the rate at which the \textbf{market}
is willing to trade one good for another.

\vspace{0.5em}

The marginal rate of substitution measures the rate at which the \textbf{consumer}
is willing to trade one good for another.

\begin{tcolorbox}[colframe=blue!70!black, colback=white, title=\textbf{Key Interpretation}]
At the consumer’s optimum, the consumer’s valuation of the two goods
equals the market’s valuation.
\end{tcolorbox}

As a result, market prices reflect the value that consumers place on goods.

\end{frame}

%==========================================================
\begin{frame}{Utility: An Alternative Way to Describe Preferences}

So far, we have described consumer preferences using \textbf{indifference curves}.
Another common approach is to use the concept of \textbf{utility}.

\begin{tcolorbox}[colframe=red!80!black, colback=white, title=\textbf{Utility}]
Utility is an abstract measure of the satisfaction or happiness
that a consumer receives from a bundle of goods.
Bundles on higher indifference curves provide higher utility.
\end{tcolorbox}

The \textbf{marginal utility} of a good is the increase in utility
from consuming one additional unit of that good.
Most goods exhibit \textbf{diminishing marginal utility}.

\end{frame}

%==========================================================
\begin{frame}{Utility: An Alternative Way to Describe Preferences}


\begin{tcolorbox}[colframe=blue!70!black, colback=white, title=\textbf{Utility-Based Optimization}]
At the consumer’s optimum:
\[
\text{MRS} = \frac{MU_x}{MU_y} = \frac{P_x}{P_y}
\]
which can be rewritten as:
\[
\frac{MU_x}{P_x} = \frac{MU_y}{P_y}
\]
\end{tcolorbox}

\textbf{Interpretation:}  
The consumer allocates spending so that the marginal utility per dollar
is equalized across goods.

\end{frame}

%==========================================================
\begin{frame}{How Changes in Income Affect Consumer Choice}

When a consumer’s income changes, the budget constraint changes.

\begin{tcolorbox}[colframe=red!80!black, colback=white, title=\textbf{Income Increase}]
An increase in income shifts the budget constraint outward.
Because prices do not change, the slope of the budget constraint remains the same.
\end{tcolorbox}

As income rises:
\begin{itemize}
    \item The consumer can afford more of both goods.
    \item The optimal choice moves to a higher indifference curve.
\end{itemize}

The change in consumption depends on whether the goods are
\textbf{normal goods} or \textbf{inferior goods}.

\end{frame}

%==========================================================
\begin{frame}{Normal Goods}

\begin{tcolorbox}[colframe=blue!70!black, colback=white, title=\textbf{Normal Good}]
A good is a normal good if the consumer buys more of it
when income increases.
\end{tcolorbox}

\begin{center}
    \includegraphics[width=0.6\textwidth]{pictures/chap21/T9.png}
\end{center}

\end{frame}

%==========================================================
\begin{frame}{Inferior Goods}

\begin{tcolorbox}[colframe=red!80!black, colback=white, title=\textbf{Inferior Good}]
A good is an inferior good if the consumer buys less of it
when income increases.
\end{tcolorbox}

\begin{center}
    \includegraphics[width=0.6\textwidth]{pictures/chap21/T10.png}
\end{center}

\end{frame}

%==========================================================
\begin{frame}{Normal Goods and Inferior Goods}

\begin{itemize}
    \item Most goods are normal goods.
    \item Some goods (e.g.\ bus rides) can be inferior goods.
\end{itemize}

\textbf{Key point:}  
Income changes affect consumption by shifting the budget constraint,
while preferences remain unchanged.

\end{frame}

%==========================================================
\begin{frame}{How Changes in Prices Affect Consumer Choice}

A change in the price of a good alters the consumer’s budget constraint.

\begin{tcolorbox}[colframe=red!80!black, colback=white, title=\textbf{Price Decrease}]
A fall in the price of a good rotates the budget constraint outward.
The intercept on the axis of the cheaper good moves outward,
while the other intercept remains unchanged.
\end{tcolorbox}

Because prices change:
\begin{itemize}
    \item The budget constraint becomes flatter or steeper.
    \item The relative price of the two goods changes.
\end{itemize}

As a result, the consumer’s optimal choice moves to a new point
on a different indifference curve.

\end{frame}
%==========================================================

\begin{frame}{Substitution Effect and Income Effect}

A change in price affects consumption through two channels.

\begin{tcolorbox}[colframe=blue!70!black, colback=white, title=\textbf{Substitution Effect}]
When the price of a good falls, the consumer substitutes toward
the cheaper good and away from the relatively more expensive good.
\end{tcolorbox}

\begin{tcolorbox}[colframe=red!80!black, colback=white, title=\textbf{Income Effect}]
A price decrease increases the consumer’s real income,
allowing her to buy more or less of a good depending on
whether it is normal or inferior.
\end{tcolorbox}

\textbf{Key result:}
\begin{itemize}
    \item Substitution effect always increases consumption of the cheaper good.
    \item Income effect depends on whether the good is normal or inferior.
\end{itemize}

\end{frame}
%==========================================================
%==========================================================
\begin{frame}{Total Effect of a Price Decrease}

Consider a fall in the price of Pepsi.

\begin{tcolorbox}[colframe=red!80!black, colback=white, title=\textbf{Pepsi (the cheaper good)}]
\begin{itemize}
    \item Income effect: buy more Pepsi.
    \item Substitution effect: buy more Pepsi.
    \item Total effect: Pepsi consumption increases.
\end{itemize}
\end{tcolorbox}

\begin{tcolorbox}[colframe=blue!70!black, colback=white, title=\textbf{Pizza (the other good)}]
\begin{itemize}
    \item Income effect: buy more pizza (normal good).
    \item Substitution effect: buy less pizza.
    \item Total effect: ambiguous.
\end{itemize}
\end{tcolorbox}

\textbf{Conclusion:}  
The total effect of a price change depends on how
income and substitution effects interact.

\end{frame}
%==========================================================
\begin{frame}{Summary}
\begin{center}
    \includegraphics[width=\textwidth]{pictures/chap21/T12.png}
\end{center}
\end{frame}
%==========================================================
\begin{frame}{Decomposing a Price Change}

\begin{center}
    \includegraphics[width=0.8\textwidth]{pictures/chap21/T11.png}
\end{center}

\end{frame}

%==========================================================
\begin{frame}{Decomposing a Price Change}

A change in the price of a good can be decomposed
into a substitution effect and an income effect.

\begin{tcolorbox}[colframe=red!80!black, colback=white, title=\textbf{Substitution Effect}]
The substitution effect is the change in consumption
that results from a change in relative prices,
holding utility constant.
\end{tcolorbox}

\begin{tcolorbox}[colframe=blue!70!black, colback=white, title=\textbf{Income Effect}]
The income effect is the change in consumption
that results from moving to a higher or lower indifference curve,
holding relative prices constant.
\end{tcolorbox}

\begin{itemize}
    \item Movement from point A to point B shows the substitution effect.
    \item Movement from point B to point C shows the income effect.
\end{itemize}

\textbf{Point B is hypothetical}:  
The consumer never actually chooses it,
but it helps isolate the two effects.

\end{frame}
%==========================================================
\begin{frame}{Deriving the Demand Curve}

A demand curve shows the quantity demanded of a good
at each possible price.

\begin{tcolorbox}[colframe=red!80!black, colback=white, title=\textbf{Key Idea}]
A consumer’s demand curve summarizes
the optimal choices that result from
\begin{itemize}
    \item the budget constraint, and
    \item the consumer’s preferences.
\end{itemize}
\end{tcolorbox}

When the price of a good changes:
\begin{itemize}
    \item the budget constraint changes,
    \item the consumer chooses a new optimal bundle,
    \item the quantity demanded changes.
\end{itemize}

Plotting the optimal quantity demanded at each price
traces out the demand curve.

\end{frame}
%==========================================================
\begin{frame}{Deriving the Demand Curve}
\begin{center}
    \includegraphics[width=\textwidth]{pictures/chap21/T13.png}
\end{center}
\end{frame}
%==========================================================
\begin{frame}{An Example: The Demand Curve for Pepsi}

Suppose the price of Pepsi falls from \$2 to \$1 per liter.

\begin{itemize}
    \item The budget constraint rotates outward.
    \item Due to substitution and income effects,
          the consumer buys more Pepsi.
\end{itemize}

\begin{tcolorbox}[colframe=blue!70!black, colback=white, title=\textbf{From Choice to Demand}]
At a price of \$2, the consumer buys 250 liters of Pepsi.\\
At a price of \$1, the consumer buys 750 liters of Pepsi.
\end{tcolorbox}

Each price–quantity pair represents one point on the demand curve.
Connecting these points gives the consumer’s demand curve for Pepsi.

\end{frame}
%==========================================================
\section{21-4 Two Applications}
%==========================================================
\begin{frame}{Do All Demand Curves Slope Downward?}

Normally, when the price of a good rises,
the quantity demanded falls.
This regularity is known as the \textbf{law of demand}.

\begin{tcolorbox}[colframe=blue!70!black, colback=white, title=\textbf{Why Demand Slopes Downward}]
When price increases:
\begin{itemize}
    \item Substitution effect: consumers buy less of the good.
    \item Income effect: consumers feel poorer.
\end{itemize}
\end{tcolorbox}

For most goods, these two effects reinforce each other,
leading to a downward-sloping demand curve.

\end{frame}

%==========================================================
\begin{frame}{Giffen Goods}

In rare cases, demand curves can slope upward.

\begin{tcolorbox}[colframe=red!80!black, colback=white, title=\textbf{Giffen Good}]
A Giffen good is an \textbf{inferior good}
for which the \textbf{income effect dominates}
the substitution effect.
\end{tcolorbox}

When the price of a Giffen good rises:
\begin{itemize}
    \item Substitution effect: buy less of the good.
    \item Income effect: consumer becomes poorer and buys more of the inferior good.
\end{itemize}

If the income effect is strong enough,
the consumer buys \textbf{more} of the good at a higher price.

\end{frame}
%==========================================================
\begin{frame}{Giffen Goods}
\begin{center}
    \includegraphics[width=0.8\textwidth]{pictures/chap21/T14.png}
\end{center}
\end{frame}
%==========================================================
\begin{frame}{Case Study: The Search for Giffen Goods}

Economists have long debated whether Giffen goods
exist in the real world.

\begin{tcolorbox}[colframe=red!80!black, colback=white, title=\textbf{Historical Evidence}]
Some historians suggest that potatoes were a Giffen good
during the 19th-century Irish potato famine.
As the price of potatoes rose, poor households reduced
their consumption of meat and bought more potatoes,
the staple food.
\end{tcolorbox}

\begin{tcolorbox}[colframe=blue!70!black, colback=white, title=\textbf{Modern Empirical Evidence}]
Robert Jensen and Nolan Miller conducted a field experiment
in rural China.
They found strong evidence of Giffen behavior for rice
among very poor households:
lowering the price of rice reduced rice consumption,
while raising the price increased it.
\end{tcolorbox}

\textbf{Conclusion:}  
Giffen goods can exist, but they are extremely rare.

\end{frame}

%==========================================================
\begin{frame}{How Do Wages Affect Labor Supply?}

Individuals allocate their limited time between \textbf{leisure} and \textbf{work}. 
Working provides income for \textbf{consumption}, while leisure provides direct satisfaction.
The key trade-off is between \textbf{leisure and consumption}.

\begin{tcolorbox}[colframe=red!80!black, colback=white, title=\textbf{Time Allocation Framework}]
\begin{itemize}
    \item Time is limited (e.g., 100 hours per week).
    \item Each hour of work earns a wage and increases consumption.
    \item Each hour of leisure reduces labor income.
\end{itemize}
\end{tcolorbox}
\end{frame}
%==========================================================
\begin{frame}{How Do Wages Affect Labor Supply?}
\begin{center}
    \includegraphics[width=0.8\textwidth]{pictures/chap21/T17.png}
\end{center}
\end{frame}
%==========================================================
\begin{frame}{How Do Wages Affect Labor Supply?}

The wage represents the \textbf{opportunity cost of leisure}.  
A higher wage makes leisure more expensive in terms of forgone consumption.

\begin{tcolorbox}[colframe=blue!70!black, colback=white, title=\textbf{Budget Constraint}]
\begin{itemize}
    \item Vertical intercept: maximum consumption (all time spent working).
    \item Horizontal intercept: maximum leisure (no work, no consumption).
    \item A higher wage rotates the budget constraint outward and makes it steeper.
\end{itemize}
\end{tcolorbox}

\end{frame}

%==========================================================
%==========================================================
\begin{frame}{Wage Changes and Labor Supply}

When the wage increases, two opposing forces affect labor supply.

\begin{tcolorbox}[colframe=red!80!black, colback=white, title=\textbf{Substitution Effect}]
\begin{itemize}
    \item Leisure becomes more expensive relative to consumption.
    \item Individuals substitute away from leisure toward work.
    \item Tends to increase hours of labor supplied.
\end{itemize}
\end{tcolorbox}

\begin{tcolorbox}[colframe=blue!70!black, colback=white, title=\textbf{Income Effect}]
\begin{itemize}
    \item Higher wages increase overall well-being.
    \item If leisure is a normal good, individuals demand more leisure.
    \item Tends to reduce hours of labor supplied.
\end{itemize}
\end{tcolorbox}

\end{frame}
%==========================================================
\begin{frame}{Wage Changes and Labor Supply}

If the substitution effect dominates, labor supply slopes upward.

\begin{center}
    \includegraphics[width=\textwidth]{pictures/chap21/T15.png}
\end{center}

\end{frame}
%==========================================================
\begin{frame}{Wage Changes and Labor Supply}
If the income effect dominates, labor supply slopes backward.

\begin{center}
    \includegraphics[width=\textwidth]{pictures/chap21/T16.png}
\end{center}

\end{frame}
%==========================================================
\begin{frame}{Income Effects on Labor Supply: Historical Evidence}

Over long periods of time, evidence suggests that labor supply may slope backward.
Historically, as wages have risen, people have chosen to work fewer hours.

\begin{tcolorbox}[colframe=red!80!black, colback=white, title=\textbf{Historical Pattern}]
\begin{itemize}
    \item In the past, six-day workweeks were common.
    \item Today, five-day workweeks are the norm.
    \item Average real wages have increased over time.
\end{itemize}
\end{tcolorbox}

\end{frame}
%==========================================================
\begin{frame}{Income Effects on Labor Supply: Historical Evidence}


Advances in technology raise worker productivity, increasing the demand for labor.
This raises equilibrium wages and the reward for working.

\begin{tcolorbox}[colframe=blue!70!black, colback=white, title=\textbf{Economic Explanation}]
\begin{itemize}
    \item Higher wages increase income.
    \item Workers become better off.
    \item Many choose to enjoy more leisure rather than work more.
\end{itemize}
\end{tcolorbox}

\begin{tcolorbox}[colframe=green!70!black, colback=white, title=\textbf{Key Insight}]
Over time, the \textbf{income effect of higher wages dominates the substitution effect}.
\end{tcolorbox}

\end{frame}

%==========================================================
\begin{frame}{Income Effect on Labor Supply: Lottery Evidence}

Lottery winners provide a unique opportunity to isolate the income effect on labor supply.

\begin{tcolorbox}[colframe=red!80!black, colback=white, title=\textbf{Why Lottery Winners Matter}]
\begin{itemize}
    \item Income increases sharply.
    \item Wages do not change.
    \item The slope of the budget constraint remains the same.
\end{itemize}
\end{tcolorbox}

\end{frame}

%==========================================================
\begin{frame}{Income Effect on Labor Supply: Lottery Evidence}


Because wages are unchanged, there is \textbf{no substitution effect}.
Any change in labor supply reflects only the income effect.

\begin{tcolorbox}[colframe=blue!70!black, colback=white, title=\textbf{Empirical Findings}]
\begin{itemize}
    \item Winners of over \$50,000:
          \begin{itemize}
              \item About 25\% quit working within one year.
              \item About 9\% reduce their working hours.
          \end{itemize}
    \item Winners of over \$1 million:
          \begin{itemize}
              \item Nearly 40\% stop working.
          \end{itemize}
\end{itemize}
\end{tcolorbox}

\begin{tcolorbox}[colframe=green!70!black, colback=white, title=\textbf{Conclusion}]
Large income increases can substantially reduce labor supply.
\end{tcolorbox}

\end{frame}


%==========================================================
%==========================================================
%==========================================================
%==========================================================
%==========================================================
%==========================================================
%==========================================================

\end{document}