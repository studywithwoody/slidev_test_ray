\documentclass{beamer}
\usepackage{graphicx} % Required for inserting images
\usepackage{amsmath}
\usepackage[most]{tcolorbox}
\usepackage{lmodern}
\usepackage{mathabx}

\usetheme{Madrid} % 可選其他主題:e.g., Warsaw, Berkeley, etc.
\usecolortheme{default}
\setbeamertemplate{caption}[numbered]% Number float-like environments
% Customize the caption
\setbeamerfont{caption}{size=\footnotesize}
% \setbeamercolor{caption}{fg=blue}
% \setbeamercolor{caption name}{fg=red}

\title{Mankiw's Principles of Economics}
\subtitle{chap 4: The Market Forces of Supply and Demand}
\author{Hsu Chun-Wei}
\date{July 2025}

\begin{document}

\maketitle
%======================================================================
\section{Demand}
%======================================================================
\begin{frame}{Markets}

    \textbf{Key Terms:}
    \begin{itemize}
        \item \textbf{Market}: a group of buyers and sellers of a particular good or service.
        \item Buyers collectively determine \textbf{demand}; sellers collectively determine \textbf{supply}.
    \end{itemize}

    \begin{tcolorbox}[colframe=blue!70!black, colback=white, title=\textbf{What Is a Market?}]
    Markets vary in structure.  
    \begin{itemize}
        \item Some are highly organized (e.g., agricultural commodity markets with auctioneers).
        \item Others are decentralized and informal (e.g., local ice-cream market).
    \end{itemize}
    \end{tcolorbox}
    
\end{frame}

%======================================================================
\begin{frame}{What Is Competition?}

\begin{tcolorbox}[colframe=blue!70!black, colback=white, title=\textbf{Competitive Market}]
A competitive market has many buyers and sellers, each having a negligible impact on the market price.
\end{tcolorbox}

\begin{itemize}
    \item Buyers know there are multiple sellers offering similar products.
    \item Sellers know buyers can switch to other sellers.
    \item No single buyer or seller determines the price of the product.
\end{itemize}

\begin{tcolorbox}[colframe=red!80!black, colback=white, title=\textbf{Price and Quantity Determination}]
Price and quantity are determined collectively by all buyers and sellers as they interact in the marketplace.
\end{tcolorbox}

\end{frame}

%======================================================================
\begin{frame}{Perfect Competition and Market Types}

\begin{tcolorbox}[colframe=blue!70!black, colback=white, title=\textbf{Perfectly Competitive Markets}]
Characteristics:
\begin{itemize}
    \item Goods offered for sale are identical.
    \item Buyers and sellers are so numerous that no individual can influence the market price.
\end{itemize}
Participants are \textbf{price takers}.
\end{tcolorbox}

Examples:
\begin{itemize}
    \item Wheat market: many farmers, millions of consumers.
\end{itemize}

\textbf{Insight:}  
Perfect competition is a useful simplification; many lessons still apply broadly to real-world markets.

\end{frame}

%======================================================================
\begin{frame}{The Demand Curve: Price and Quantity Demanded}

\begin{itemize}
    \item Price plays a central role in determining quantity demanded.
    \item When the price of ice cream rises, buyers purchase fewer cones; when price falls, buyers purchase more.
    \item This inverse relationship is known as the \textbf{law of demand}.
\end{itemize}

\begin{tcolorbox}[colframe=blue!70!black, colback=white, title=\textbf{Quantity Demanded}]
The quantity demanded of a good is the amount buyers are willing and able to purchase at a given price.
\end{tcolorbox}

\begin{tcolorbox}[colframe=red!80!black, colback=white, title=\textbf{Law of Demand}]
Other things being equal, when the price of a good rises, quantity demanded falls;  
when the price falls, quantity demanded rises.
\end{tcolorbox}

\end{frame}

%======================================================================
\begin{frame}{Demand Schedule}

\begin{columns}[T]

%---------------- Left column: table explanation ----------------%
\begin{column}{0.5\textwidth}
\begin{tcolorbox}[colframe=blue!70!black, colback=white, title=\textbf{Demand Schedule}]
A demand schedule shows the quantity demanded at each price.
\end{tcolorbox}

Example (Catherine's monthly ice-cream consumption):
\begin{itemize}
    \item Higher price → fewer cones demanded.
    \item At \$3.00: 0 cones; at \$0.50: 10 cones.
    \item Reflects the law of demand.
\end{itemize}

\end{column}

%---------------- Right column: image ----------------%
\begin{column}{0.5\textwidth}
    \begin{figure}
        \centering
        \includegraphics[width=0.9\textwidth]{pictures/chap4/T8.png}
    \end{figure}
\end{column}

\end{columns}

\end{frame}

%======================================================================
\begin{frame}{Demand Curve}

\begin{tcolorbox}[colframe=red!80!black, colback=white, title=\textbf{Demand Curve}]
Graphical representation of the demand schedule.  
Slopes downward: lower price → greater quantity demanded.
\end{tcolorbox}

\begin{figure}
    \centering
    \includegraphics[width=0.5\textwidth]{pictures/chap4/T9.png}
\end{figure}

\end{frame}

%======================================================================
\begin{frame}{Why the Demand Curve Slopes Downward}

\begin{tcolorbox}[colframe=blue!70!black, colback=white, title=\textbf{Demand Curve Basics}]
The demand curve shows the relationship between price (vertical axis) and quantity demanded (horizontal axis).
\end{tcolorbox}

\begin{itemize}
    \item As price decreases, buyers purchase more units of the good.
    \item As price increases, buyers reduce consumption.
    \item This inverse relationship results in a downward-sloping demand curve.
\end{itemize}

\begin{tcolorbox}[colframe=red!80!black, colback=white, title=\textbf{Key Insight}]
The downward slope reflects the law of demand:  
\textit{other things equal, lower price leads to higher quantity demanded}.
\end{tcolorbox}

\end{frame}

%======================================================================
\begin{frame}{Market Demand versus Individual Demand}

\begin{tcolorbox}[colframe=blue!70!black, colback=white, title=\textbf{Market Demand}]
Market demand is the sum of all individual demands for a good at each price.
\end{tcolorbox}

\begin{itemize}
    \item Individual demand curves show how much one consumer buys at various prices.
    \item To obtain market demand, we \textbf{horizontally sum} the quantities demanded by all individuals at each price.
    \item Example: Catherine and Nicholas each have their own demand schedule for ice cream.
\end{itemize}

\begin{tcolorbox}[colframe=red!80!black, colback=white, title=\textbf{Interpretation}]
The market demand curve shows how total quantity demanded varies with the price of the good, holding all other factors constant.
\end{tcolorbox}

\end{frame}

%======================================================================
\begin{frame}{Summing Individual Demands to Obtain Market Demand}

\begin{tcolorbox}[colframe=blue!70!black, colback=white, title=\textbf{How Market Demand Is Derived}]
\begin{itemize}
    \item Add individual quantities at each price.
    \item Example: At \$2.00,
    \begin{itemize}
        \item Catherine demands 4 cones.
        \item Nicholas demands 3 cones.
        \item Market demand = 7 cones.
    \end{itemize}
\end{itemize}
\end{tcolorbox}

\begin{figure}
    \centering
    \includegraphics[width=\textwidth]{pictures/chap4/T10.png}
    \caption*{\small Market Demand as the Sum of Individual Demands}
\end{figure}

\end{frame}

%======================================================================
\begin{frame}{Shifts in the Demand Curve}

\begin{tcolorbox}[colframe=blue!70!black, colback=white, title=\textbf{Why Demand Shifts}]
The demand curve holds other factors constant, but real-world conditions change over time. When these changes affect quantity demanded at every price, the demand curve shifts.
\end{tcolorbox}

\begin{itemize}
    \item A positive change (e.g., health benefits of ice cream) increases quantity demanded at all prices → demand shifts right.
    \item A negative change decreases quantity demanded at all prices → demand shifts left.
    \item A shift means buyers now purchase more or less at the same price.
\end{itemize}

\end{frame}

%======================================================================
\begin{frame}{Graphical Representation of Demand Shifts}

\begin{tcolorbox}[colframe=red!80!black, colback=white, title=\textbf{Increase vs. Decrease in Demand}]
\begin{itemize}
    \item \textbf{Increase in demand}: curve shifts right.
    \item \textbf{Decrease in demand}: curve shifts left.
\end{itemize}
\end{tcolorbox}


\begin{figure}
    \centering
    \includegraphics[width= 0.6\textwidth]{pictures/chap4/T11.png}
\end{figure}


\end{frame}

%======================================================================
\begin{frame}{Income and Types of Goods}

\begin{tcolorbox}[colframe=blue!70!black, colback=white, title=\textbf{Income and Demand}]
A change in income affects quantity demanded for many goods.
\end{tcolorbox}

\begin{itemize}
    \item When income falls, consumers have less to spend and reduce consumption of many goods.
\end{itemize}

\begin{tcolorbox}[colframe=red!80!black, colback=white, title=\textbf{Types of Goods}]
\begin{itemize}
    \item \textbf{Normal Good}: demand falls when income falls; demand rises when income rises.
    \item \textbf{Inferior Good}: demand rises when income falls (e.g., bus rides as a cheaper alternative).
\end{itemize}
\end{tcolorbox}

\end{frame}

%======================================================================
\begin{frame}{Prices of Related Goods}

\begin{tcolorbox}[colframe=blue!70!black, colback=white, title=\textbf{Substitutes}]
When a fall in the price of one good reduces the demand for another good, the two goods are called \textbf{substitutes}.
\end{tcolorbox}

Examples:
\begin{itemize}
    \item Ice cream and frozen yogurt
    \item Hot dogs and hamburgers
    \item Sweaters and sweatshirts
    \item Movie tickets and DVD rentals
\end{itemize}

\end{frame}

%======================================================================
\begin{frame}{Prices of Related Goods}

\begin{tcolorbox}[colframe=red!80!black, colback=white, title=\textbf{Complements}]
When a fall in the price of one good raises the demand for another good, the two goods are called \textbf{complements}.
\end{tcolorbox}

Examples:
\begin{itemize}
    \item Ice cream and hot fudge
    \item Automobiles and gasoline
    \item Computers and software
    \item Peanut butter and jelly
\end{itemize}
\end{frame}
%======================================================================
\begin{frame}{Other Determinants of Demand}

\textbf{Tastes}

A major determinant of demand. If you like a good more, you buy more of it. Economists do not explain tastes but study how demand changes when tastes change.

\vspace{1em}
\textbf{Expectations}

Expectations about future income or prices influence present demand. 
\begin{itemize}
    \item Expect higher income → demand increases today.
    \item Expect lower future price → less willing to buy today.
\end{itemize}

\vspace{1em}
\textbf{Number of Buyers}

Market demand depends on how many buyers participate.  
More buyers → higher market demand at every price.

\end{frame}

%======================================================================
\begin{frame}{Conclusion}
\begin{figure}
    \centering
    \includegraphics[width= 0.9\textwidth]{pictures/chap4/T12.png}
\end{figure}
\end{frame}
%======================================================================
\begin{frame}{Case Study: Reducing the Quantity of Smoking Demanded}

Public policymakers often want to reduce smoking because of adverse health effects. 
There are two main policy approaches:

\begin{tcolorbox}[colframe=blue!70!black, colback=white, title=\textbf{1. Shift the Demand Curve Left}]
Policies such as public service announcements, health warnings, and bans on advertising aim to reduce the quantity demanded at every price.  
If effective, the demand curve for cigarettes shifts to the left.
\end{tcolorbox}

\begin{tcolorbox}[colframe=red!80!black, colback=white, title=\textbf{2. Increase the Price of Cigarettes}]
If the government raises cigarette taxes, the higher price reduces the quantity smoked.  
This is a \textbf{movement along} the demand curve, not a shift.
\end{tcolorbox}

\end{frame}

%======================================================================
\begin{frame}{Shifts vs. Movements in Cigarette Demand}

\begin{figure}
    \centering
    \includegraphics[width=\textwidth]{pictures/chap4/T13.png}
    \caption{\small A shift in demand vs. a movement along the demand curve}
\end{figure}

\end{frame}

%======================================================================
\begin{frame}{Effects of Cigarette Taxes and Related Goods}

\textbf{Effects of Cigarette Taxes}

Economists study how smoking responds to price changes:
\begin{itemize}
    \item A 10\% increase in cigarette prices reduces quantity demanded by about 4\%.
    \item Teenagers are more sensitive: a 10\% increase leads to a 12\% drop in teen smoking.
\end{itemize}

\vspace{1em}

\textbf{Cigarettes and Illicit Drugs}

How cigarette prices affect marijuana use is debated:
\begin{itemize}
    \item Some argue tobacco and marijuana are substitutes → higher cigarette prices increase marijuana use.
    \item Others argue tobacco is a “gateway drug,” suggesting complementarity.
    \item Data suggest cigarettes and marijuana may be \textbf{complements}, not substitutes.
\end{itemize}

\end{frame}
%======================================================================
\section{Supply}
%======================================================================
\begin{frame}{The Supply Curve: Price and Quantity Supplied}

The \textbf{quantity supplied} of a good or service is the amount that sellers are willing 
and able to sell. Many factors influence quantity supplied, but price plays a central role.

\vspace{0.6em}

When the price of ice cream is high, selling becomes profitable, so producers supply more.  
When the price is low, supplying becomes less profitable, and quantity supplied decreases.  
At very low prices, some sellers may even shut down production entirely.

\begin{tcolorbox}[colframe=orange!85!black, colback=white, title=\textbf{Law of Supply}]
Other things being equal, the quantity supplied of a good increases when its price rises,  
and decreases when its price falls.
\end{tcolorbox}

This positive relationship between price and quantity supplied forms the basis of the 
\textbf{supply curve}.

\end{frame}

%======================================================================
\begin{frame}{Supply Schedule}

The \textbf{supply schedule} lists the quantity of ice-cream cones supplied at each price.  
A higher price leads to a higher quantity supplied, illustrating the law of supply.

\begin{tcolorbox}[colframe=blue!70!black, colback=white, title=\textbf{Key Idea: Upward-Sloping Supply Curve}]
\begin{itemize}
    \item Higher prices make production more profitable.
    \item Sellers are willing to supply more as price increases.
    \item The supply curve slopes upward as a result.
\end{itemize}
\end{tcolorbox}

\begin{center}
\includegraphics[width=0.3\textwidth]{pictures/chap4/T14.png}
\end{center}

\end{frame}

%======================================================================
\begin{frame}{Supply Curve}
    \begin{center}
    \includegraphics[width=0.8\textwidth]{pictures/chap4/T15.png}
    \end{center}
\end{frame}
%======================================================================
\begin{frame}{Market Supply versus Individual Supply}

Just as market demand is the sum of all buyers' demands, \textbf{market supply} is the sum 
of the quantities supplied by all sellers. Each seller has an individual supply schedule, 
which shows the quantity supplied at each price.

\begin{tcolorbox}[colframe=orange!85!black, colback=white, title=\textbf{Key Idea: Market Supply}]
\begin{itemize}
    \item Market supply is the \textbf{horizontal sum} of all individual supply curves.
    \item At each price, add the quantity supplied by each producer.
    \item The resulting curve shows how total quantity supplied varies with price.
\end{itemize}
\end{tcolorbox}

\end{frame}

%======================================================================
\begin{frame}{Market Supply as the Sum of Individual Supplies}
For example, in the ice-cream market, Ben and Jerry each have their own supply schedules.  
At any given price, the market supply equals the quantity Ben supplies plus the quantity 
Jerry supplies.
\begin{center}
    \includegraphics[width=0.8\textwidth]{pictures/chap4/T16.png}
\end{center}
\end{frame}
%======================================================================
\begin{frame}{Market Supply as the Sum of Individual Supplies}

\begin{tcolorbox}[colframe=blue!70!black, colback=white, title=\textbf{Horizontal Summation}]
To obtain the market supply curve:
\begin{itemize}
    \item Take each price level.
    \item Add the individual quantities supplied at that price.
    \item Plot the total on the horizontal axis.
\end{itemize}
\end{tcolorbox}

\begin{center}
\includegraphics[width=\textwidth]{pictures/chap4/T17.png}
\end{center}

\end{frame}

%======================================================================
\begin{frame}{Shifts in the Supply Curve}

Because the supply curve is drawn holding other factors constant, changes in these 
underlying factors cause the supply curve to shift.

\begin{tcolorbox}[colframe=orange!85!black, colback=white, title=\textbf{Increase vs. Decrease in Supply}]
\begin{itemize}
    \item \textbf{Increase in supply}: supply curve shifts right; sellers supply more at every price.
    \item \textbf{Decrease in supply}: supply curve shift+s left; sellers supply less at every price.
\end{itemize}
\end{tcolorbox}

For example, if the price of sugar 
falls, the cost of producing ice cream decreases. At any given price, sellers are willing 
to supply more, shifting the supply curve to the right.

\end{frame}

%======================================================================
\begin{frame}{Visualizing Shifts in the Supply Curve}

\begin{itemize}
    \item A rightward shift represents an \textbf{increase in supply}.
    \item A leftward shift represents a \textbf{decrease in supply}.
\end{itemize}

\begin{center}
\includegraphics[width=0.80\textwidth]{pictures/chap4/T18.png}
\end{center}

\end{frame}

%==========================================================
\begin{frame}{Determinants of Supply}

Several factors other than price can shift the supply curve. These factors change firms’ 
costs or incentives and therefore alter the quantity supplied at every price.

\begin{tcolorbox}[colframe=purple!80!black, colback=white, title=\textbf{Key Determinants of Supply}]
\begin{itemize}
    \item \textbf{Input Prices}  
    Higher input prices make production less profitable, decreasing supply. Lower input 
    prices reduce costs and increase supply.

    \item \textbf{Technology}  
    Advances in technology reduce production costs, raising supply.

    \item \textbf{Expectations}  
    If firms expect higher prices in the future, they may reduce current supply to store 
    more for later sale.

    \item \textbf{Number of Sellers}  
    More sellers increase market supply; fewer sellers reduce it.
\end{itemize}
\end{tcolorbox}

\end{frame}
%==========================================================
\begin{frame}{Conclusion}
    \begin{center}
        \includegraphics[width=0.8\textwidth]{pictures/chap4/T18.png}
    \end{center}
\end{frame}
%==========================================================
\section{Market equilibrium}
%==========================================================
\begin{frame}{Market Equilibrium}

Figure 8 shows the market supply and demand curves together. There is one point where 
the two curves intersect—this point is called the \textbf{equilibrium}. The price at this 
point is the \textbf{equilibrium price}, and the quantity is the \textbf{equilibrium quantity}. 

\begin{tcolorbox}[colframe=blue!70!black, colback=white, title=\textbf{What Is Equilibrium?}]
At the equilibrium price, the quantity that buyers are willing and able to buy exactly 
equals the quantity that sellers are willing and able to sell.  
This price is sometimes called the \textbf{market-clearing price}.
\end{tcolorbox}

Markets naturally move toward equilibrium because buyers and sellers respond to incentives.  
If the price is not at equilibrium, market forces push it toward the equilibrium price.

\end{frame}

%==========================================================
\begin{frame}{The Equilibrium of Supply and Demand}

The equilibrium is found at the intersection of the supply and demand curves.  
At the equilibrium price of \$2.00, 7 ice-cream cones are supplied and 7 are demanded.

\begin{center}
\includegraphics[width=0.82\textwidth]{pictures/chap4/T20.png}
\end{center}

\end{frame}

%==========================================================
\begin{frame}{Surplus: Price Above Equilibrium}

When the market price is above the equilibrium price, quantity supplied exceeds 
quantity demanded. For example, at a price of \$2.50, sellers supply 10 cones while 
buyers demand only 4 cones. The result is a \textbf{surplus}, also called 
\textit{excess supply}.

\begin{tcolorbox}[colframe=red!70!black, colback=white, title=\textbf{What Happens During a Surplus?}]
\begin{itemize}
    \item Sellers cannot sell all they want at the current price.
    \item Inventories rise.
    \item Sellers cut prices to increase sales.
    \item Price continues to fall until it reaches equilibrium.
\end{itemize}
\end{tcolorbox}

\end{frame}

%==========================================================
\begin{frame}{Surplus: Price Above Equilibrium}
    \begin{center}
\includegraphics[width=0.80\textwidth]{pictures/chap4/T21.png}
\end{center}
\end{frame}
%==========================================================
\begin{frame}{Shortage: Price Below Equilibrium}

When the market price is below the equilibrium price, quantity demanded exceeds 
quantity supplied. At a price of \$1.50, buyers demand 10 cones while sellers supply 
only 4 cones. This situation creates a \textbf{shortage}, also called 
\textit{excess demand}.

\begin{tcolorbox}[colframe=blue!70!black, colback=white, title=\textbf{What Happens During a Shortage?}]
\begin{itemize}
    \item Buyers cannot purchase all they want at the current price.
    \item Long lines and rationing may occur.
    \item Sellers raise prices to reduce excess demand.
    \item Price rises until it reaches equilibrium.
\end{itemize}
\end{tcolorbox}


\end{frame}

%==========================================================
\begin{frame}{Shortage: Price Below Equilibrium}
    \begin{center}
\includegraphics[width=0.80\textwidth]{pictures/chap4/T22.png}
\end{center}
\end{frame}
%==========================================================
\begin{frame}{Price Adjustment and the Return to Equilibrium}

Whether prices start too high (surplus) or too low (shortage), the actions of buyers 
and sellers naturally push the market toward the equilibrium price. Once the market 
reaches equilibrium, there is no upward or downward pressure on price.

\begin{tcolorbox}[colframe=green!70!black, colback=white, title=\textbf{Law of Supply and Demand}]
The price of any good adjusts to bring the quantity supplied and the quantity 
demanded into balance.
\end{tcolorbox}

Disequilibrium conditions are temporary because market forces continually move prices 
toward equilibrium.

\end{frame}
%==========================================================

%==========================================================
%==========================================================
%==========================================================
%==========================================================
%==========================================================
%==========================================================
%==========================================================
%==========================================================
%==========================================================
%==========================================================

\end{document}