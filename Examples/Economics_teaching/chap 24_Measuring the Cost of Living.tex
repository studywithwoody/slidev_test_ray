\documentclass{beamer}
\usepackage{graphicx} % Required for inserting images
\usepackage{amsmath}
\usepackage[most]{tcolorbox}
\usepackage{lmodern}
\usepackage{mathabx}

\usetheme{Madrid} % 可選其他主題:e.g., Warsaw, Berkeley, etc.
\usecolortheme{default}
\setbeamertemplate{caption}[numbered]% Number float-like environments
% Customize the caption
\setbeamerfont{caption}{size=\footnotesize}
% \setbeamercolor{caption}{fg=blue}
% \setbeamercolor{caption name}{fg=red}

% 每章節開始時自動產生章節頁
\AtBeginSection[]
{
  \begin{frame}
    \frametitle{Table of Contents}
    \tableofcontents[currentsection]
  \end{frame}
}


\title{Mankiw's Principles of Economics}
\subtitle{chap 24: Measuring the Cost of Living}
\author{Hsu Chun-Wei}
\date{July 2025}

\begin{document}

\maketitle

% 目錄頁
\begin{frame}
  \frametitle{Table of Contents}
  \tableofcontents
\end{frame}

%==========================================================
\section{24-1 The Consumer Price Index}
%==========================================================
\begin{frame}{24-1 The Consumer Price Index (CPI)}

The \textbf{Consumer Price Index (CPI)} measures the overall cost of the goods and services bought by a typical consumer.

It is calculated and reported monthly by the \textbf{Bureau of Labor Statistics (BLS)} and is widely used to track changes in the cost of living.

\begin{tcolorbox}[colframe=red!80!black, colback=white, title=\textbf{Key Idea: What CPI Measures}]
\begin{itemize}
    \item CPI measures the cost of a fixed basket of goods and services.
    \item It reflects how prices change over time for a typical consumer.
    \item A rising CPI indicates inflation.
\end{itemize}
\end{tcolorbox}

CPI is often compared with the \textbf{GDP deflator}, another measure of the overall price level, but CPI focuses specifically on consumer purchases.

\end{frame}

%==========================================================
\begin{frame}{How the CPI Is Calculated}

To understand how CPI is constructed, consider a simplified economy with only two goods: hot dogs and hamburgers.

\begin{center}
    \includegraphics[width=0.75\textwidth]{pictures/chap24/T1.png}
\end{center}

\end{frame}
%==========================================================
\begin{frame}{What Is in the CPI's Basket?}

When constructing the CPI, the \textbf{Bureau of Labor Statistics (BLS)} includes the goods and services typically purchased by consumers.

Each category is \textbf{weighted} according to how much consumers spend on it, which the BLS calls its \textbf{relative importance}.

\begin{tcolorbox}[colframe=blue!80!black, colback=white, title=\textbf{Key Idea: CPI Basket}]
\begin{itemize}
    \item CPI is based on a basket of goods and services.
    \item Categories are weighted by consumer spending.
\end{itemize}
\end{tcolorbox}

\begin{itemize}
    \item A price increase in a high-weight category raises CPI more.
    \item A price increase in a low-weight category has smaller impact.
    \item CPI reflects \textbf{consumer spending patterns}, not importance to individuals.
\end{itemize}

\end{frame}
%==========================================================
\begin{frame}{What Is in the CPI's Basket?}
\begin{center}
    \includegraphics[width=0.8\textwidth]{pictures/chap24/T2.png}
\end{center}

\end{frame}
%==========================================================
\begin{frame}{How the CPI Is Calculated}

\begin{tcolorbox}[colframe=blue!70!black, colback=white, title=\textbf{Five Steps to Calculate CPI}]
\begin{enumerate}
    \item \textbf{Fix the basket}: Determine the typical consumer's basket of goods.
    \item \textbf{Find the prices}: Record prices of each good in different years.
    \item \textbf{Compute the basket's cost}: Calculate total cost using the fixed basket.
    \item \textbf{Choose a base year and compute CPI}:
    \[
    CPI = \frac{\text{Price of basket in current year}}{\text{Price of basket in base year}} \times 100
    \]
    \item \textbf{Compute the inflation rate}.
\end{enumerate}
\end{tcolorbox}

The basket remains fixed so that CPI isolates price changes from quantity changes.

\end{frame}

%==========================================================
\begin{frame}{Base Year and Inflation Rate}

The \textbf{base year} is the benchmark year against which other years are compared.
The CPI is always \textbf{100 in the base year}.

\begin{tcolorbox}[colframe=red!70!black, colback=white, title=\textbf{Inflation Rate Formula}]
\[
\text{Inflation Rate in Year 2}
=
\frac{CPI_{\text{Year 2}} - CPI_{\text{Year 1}}}{CPI_{\text{Year 1}}}
\times 100
\]
\end{tcolorbox}

\begin{itemize}
    \item Inflation measures the percentage change in the CPI.
    \item A higher inflation rate means prices are rising faster.
\end{itemize}

In real data, CPI is reported monthly and often appears in news reports.

\end{frame}

%==========================================================
\begin{frame}{CPI and Other Price Indexes}

In addition to CPI, the BLS also calculates other price indexes.

\begin{tcolorbox}[colframe=blue!80!black, colback=white, title=\textbf{Producer Price Index (PPI)}]
\begin{itemize}
    \item PPI measures the cost of goods and services bought by firms.
    \item Firms often pass higher costs to consumers.
    \item Therefore, changes in PPI may help predict changes in CPI.
\end{itemize}
\end{tcolorbox}

\end{frame}

%==========================================================
\begin{frame}{Problems in Measuring the Cost of Living}

The goal of the CPI is to measure changes in the cost of living.
However, CPI is an \textbf{imperfect measure} and tends to overstate inflation.

\begin{tcolorbox}[colframe=red!80!black, colback=white, title=\textbf{Problem 1: Substitution Bias}]
\begin{itemize}
    \item CPI uses a \textbf{fixed basket of goods}.
    \item Consumers substitute toward relatively cheaper goods when prices change.
    \item CPI ignores substitution and therefore \textbf{overstates the increase in the cost of living}.
\end{itemize}
\end{tcolorbox}

\textbf{Example:}  
If apples become more expensive and pears become cheaper, consumers buy fewer apples and more pears.  
CPI assumes consumers keep buying the same basket, exaggerating inflation.

\end{frame}

%==========================================================
\begin{frame}{Problem 2: Introduction of New Goods}

When new goods are introduced, consumers gain more choices and higher well-being.

\begin{tcolorbox}[colframe=blue!80!black, colback=white, title=\textbf{New Goods Bias}]
\begin{itemize}
    \item New goods increase consumer choice and make each dollar more valuable.
    \item CPI does not immediately reflect the benefits of new goods.
    \item As a result, CPI \textbf{overstates the cost of living}.
\end{itemize}
\end{tcolorbox}

\textbf{Example:}  
The introduction of the iPod made it easier to listen to music.
Consumers were better off even if prices did not fall, but CPI did not decrease.

\end{frame}

%==========================================================
\begin{frame}{Problem 3: Unmeasured Quality Change}

Quality changes are difficult to measure accurately.

\begin{tcolorbox}[colframe=red!70!black, colback=white, title=\textbf{Quality Change Bias}]
\begin{itemize}
    \item If quality improves while prices stay the same, CPI overstates inflation.
    \item If quality deteriorates, CPI understates inflation.
    \item BLS attempts to adjust for quality, but measurement is imperfect.
\end{itemize}
\end{tcolorbox}

Despite improvements, many economists believe CPI still slightly overstates inflation.

\begin{itemize}
    \item CPI is used to adjust Social Security and government benefits.
    \item Small measurement errors can have large policy effects.
\end{itemize}

\end{frame}

%==========================================================
\begin{frame}{GDP Deflator vs Consumer Price Index}

Both the \textbf{GDP deflator} and the \textbf{Consumer Price Index (CPI)} measure changes in the overall price level.

\begin{tcolorbox}[colframe=blue!80!black, colback=white, title=\textbf{Two Price Indexes}]
\begin{itemize}
    \item \textbf{GDP Deflator}:
    \[
    \frac{\text{Nominal GDP}}{\text{Real GDP}} \times 100
    \]
    \item \textbf{CPI}: Measures the cost of a typical consumer’s basket of goods and services.
\end{itemize}
\end{tcolorbox}

Usually, CPI inflation and GDP deflator inflation move together, but important differences can cause them to diverge.

\end{frame}

%==========================================================
\begin{frame}{Difference 1: What Goods Are Included?}

The most important difference is \textbf{what each index measures}.

\begin{tcolorbox}[colframe=red!80!black, colback=white, title=\textbf{Coverage Difference}]
\begin{itemize}
    \item \textbf{GDP Deflator}:
    \begin{itemize}
        \item Includes all goods and services \textbf{produced domestically}.
        \item Excludes imports.
    \end{itemize}
    \item \textbf{CPI}:
    \begin{itemize}
        \item Includes goods and services \textbf{bought by consumers}.
        \item Includes imports.
    \end{itemize}
\end{itemize}
\end{tcolorbox}

\textbf{Examples:}
\begin{itemize}
    \item A U.S.-made airplane sold to the military affects GDP deflator, not CPI.
    \item An imported car bought by consumers affects CPI, not GDP deflator.
\end{itemize}

\end{frame}

%==========================================================
\begin{frame}{Difference 2: Fixed vs Changing Basket}

The second difference concerns how prices are weighted.

\begin{tcolorbox}[colframe=blue!70!black, colback=white, title=\textbf{Basket Difference}]
\begin{itemize}
    \item \textbf{CPI}:
    \begin{itemize}
        \item Uses a \textbf{fixed basket} of goods and services.
        \item Basket changes infrequently.
    \end{itemize}
    \item \textbf{GDP Deflator}:
    \begin{itemize}
        \item Uses a \textbf{changing basket}.
        \item Reflects current production automatically.
    \end{itemize}
\end{itemize}
\end{tcolorbox}

When prices change unevenly, these different weighting methods can cause CPI and GDP deflator inflation to diverge.

\end{frame}
%==========================================================
\begin{frame}{Historical Comparison of Inflation Measures}
\begin{center}
    \includegraphics[width=0.9\textwidth]{pictures/chap24/T3.png}
\end{center}
\end{frame}
%==========================================================
\begin{frame}{Historical Comparison of Inflation Measures}

Historically, CPI and the GDP deflator usually tell a similar story.

\begin{itemize}
    \item 1970s: Both measures show high inflation.
    \item Late 1970s–early 1980s: CPI rose faster due to sharp increases in oil prices.
    \item 1980s–2000s: Both measures show relatively low inflation.
\end{itemize}

Divergence between CPI and GDP deflator is the exception, not the rule.

\end{frame}
%==========================================================
\section{24-2 Correcting Economic Variables for the Effects of Inflation}
%==========================================================
\begin{frame}{Correcting Economic Variables for Inflation}

The purpose of measuring the overall price level is to allow comparisons of dollar figures from different points in time.

\begin{tcolorbox}[colframe=blue!80!black, colback=white, title=\textbf{Why Inflation Adjustment Matters}]
\begin{itemize}
    \item A dollar today does not have the same purchasing power as a dollar in the past.
    \item Inflation distorts comparisons of income, prices, and output over time.
    \item Price indexes allow us to make meaningful comparisons.
\end{itemize}
\end{tcolorbox}

Once we know how price indexes are calculated, we can use them to compare past and present dollar values.

\end{frame}

%==========================================================
\begin{frame}{Using Price Indexes to Compare Dollar Values}

Economists use price indexes, such as the CPI, to adjust nominal values for inflation.

\begin{tcolorbox}[colframe=red!80!black, colback=white, title=\textbf{Key Question}]
\begin{itemize}
    \item How do we convert a dollar figure from the past into today's dollars?
    \item How do we compare real purchasing power across years?
\end{itemize}
\end{tcolorbox}

In this section, we learn how to use price indexes to correct economic variables for the effects of inflation.

\end{frame}

%==========================================================
\begin{frame}{Dollar Figures from Different Times}

A dollar in the past does not have the same purchasing power as a dollar today.

\begin{tcolorbox}[colframe=blue!80!black, colback=white, title=\textbf{Key Question}]
\begin{itemize}
    \item Was Babe Ruth's salary of \$80,000 in 1931 high or low by today's standards?
    \item To answer this, we must adjust for inflation.
\end{itemize}
\end{tcolorbox}

To compare dollar amounts across time, we convert past dollars into today's dollars using a price index such as the CPI.

\end{frame}

%==========================================================
\begin{frame}{Converting Dollars across Time}

The formula for converting a dollar figure from year $T$ into today's dollars is:

\begin{tcolorbox}[colframe=red!80!black, colback=white, title=\textbf{Inflation Adjustment Formula}]
\[
\text{Amount in today's dollars}
=
\text{Amount in year } T \times
\frac{\text{Price level today}}{\text{Price level in year } T}
\]
\end{tcolorbox}

A price index such as the \textbf{Consumer Price Index (CPI)} measures the price level and determines the size of the inflation correction.

\end{frame}

%==========================================================
\begin{frame}{Example: Babe Ruth's Salary}

Government statistics show:
\begin{itemize}
    \item CPI in 1931 = 15.2
    \item CPI in 2012 = 229.5
\end{itemize}

\begin{tcolorbox}[colframe=blue!70!black, colback=white, title=\textbf{Calculation}]
\[
\text{Salary in 2012 dollars}
=
\$80{,}000 \times \frac{229.5}{15.2}
=
\$1{,}207{,}894
\]
\end{tcolorbox}

Babe Ruth’s 1931 salary is equivalent to over \$1.2 million in 2012 dollars.

\end{frame}

%==========================================================
\begin{frame}{Another Example: President Hoover}

President Hoover earned \$75,000 in 1931.

\begin{tcolorbox}[colframe=red!70!black, colback=white, title=\textbf{Inflation-Adjusted Salary}]
\[
\$75{,}000 \times \frac{229.5}{15.2}
=
\$1{,}132{,}401
\]
\end{tcolorbox}

In 2012 dollars, Hoover’s salary was well above President Obama’s \$400,000 salary.

\end{frame}

%==========================================================
\begin{frame}{Indexation}

Price indexes are often used to correct for the effects of inflation automatically.

\begin{tcolorbox}[colframe=blue!80!black, colback=white, title=\textbf{Definition: Indexation}]
\begin{itemize}
    \item A dollar amount is \textbf{indexed for inflation} if it is automatically adjusted for changes in the price level.
    \item Indexation is typically written into laws or contracts.
\end{itemize}
\end{tcolorbox}

Indexation allows dollar figures from different times to maintain their real purchasing power.

\end{frame}

%==========================================================
\begin{frame}{Indexation in Labor Contracts: COLA}

Many long-term labor contracts include indexation of wages to inflation.

\begin{tcolorbox}[colframe=red!80!black, colback=white, title=\textbf{Cost-of-Living Allowance (COLA)}]
\begin{itemize}
    \item COLA automatically raises wages when the CPI rises.
    \item It protects workers from losing purchasing power due to inflation.
    \item Wages indexed to CPI rise with the cost of living.
\end{itemize}
\end{tcolorbox}

COLA clauses are commonly used in contracts between firms and labor unions.

\end{frame}

%==========================================================
\begin{frame}{Indexation in Government Policy}

Indexation is also widely used in government programs.

\begin{tcolorbox}[colframe=blue!70!black, colback=white, title=\textbf{Examples of Indexation}]
\begin{itemize}
    \item \textbf{Social Security benefits} are adjusted annually for inflation.
    \item \textbf{Income tax brackets} are indexed to inflation.
\end{itemize}
\end{tcolorbox}

However, not all parts of the tax system are indexed for inflation, which can create distortions.
These issues are discussed further when analyzing the costs of inflation.

\end{frame}

%==========================================================
\begin{frame}{Real and Nominal Interest Rates}

Interest rates compare money at different points in time.
Therefore, inflation must be taken into account.

\begin{tcolorbox}[colframe=blue!80!black, colback=white, title=\textbf{Key Insight}]
\begin{itemize}
    \item Earning interest increases the number of dollars you have.
    \item What matters is how much those dollars can buy.
    \item Inflation changes the value of future dollars.
\end{itemize}
\end{tcolorbox}

To understand gains from saving or costs of borrowing, we must correct interest rates for inflation.

\end{frame}

%==========================================================
\begin{frame}{Example: Sally Saver}

Suppose Sally deposits \$1{,}000 in a bank account earning a \textbf{10\% nominal interest rate}.

\begin{itemize}
    \item After one year, she withdraws \$1{,}100.
    \item Nominally, Sally has \$100 more than before.
\end{itemize}

\begin{tcolorbox}[colframe=red!80!black, colback=white, title=\textbf{Key Question}]
Is Sally actually richer in terms of purchasing power?
\end{tcolorbox}

The answer depends on what happens to prices during the year.

\end{frame}

%==========================================================
\begin{frame}{Inflation and Purchasing Power}

Assume Sally spends all her money on DVDs.

\begin{tcolorbox}[colframe=blue!70!black, colback=white, title=\textbf{Outcomes under Different Price Changes}]
\begin{itemize}
    \item \textbf{0\% inflation}: Purchasing power rises by 10\%.
    \item \textbf{6\% inflation}: Purchasing power rises by about 4\%.
    \item \textbf{10\% inflation}: Purchasing power is unchanged.
    \item \textbf{12\% inflation}: Purchasing power falls by about 2\%.
    \item \textbf{Deflation}: Purchasing power rises more than the interest rate.
\end{itemize}
\end{tcolorbox}

Higher inflation reduces the real gain from saving.
If inflation exceeds the interest rate, purchasing power falls.

\end{frame}

%==========================================================
\begin{frame}{Nominal vs Real Interest Rates}

Economists distinguish between two types of interest rates.

\begin{tcolorbox}[colframe=red!80!black, colback=white, title=\textbf{Definitions}]
\begin{itemize}
    \item \textbf{Nominal interest rate}: Measures the change in dollar amounts.
    \item \textbf{Real interest rate}: Measures the change in purchasing power.
\end{itemize}
\end{tcolorbox}

The relationship between nominal interest rate, inflation, and real interest rate is approximately:

\[
\text{Real interest rate}
=
\text{Nominal interest rate}
-
\text{Inflation rate}
\]

The real interest rate tells us how fast purchasing power rises over time.

\end{frame}

%==========================================================
\begin{frame}{Case Study: Interest Rates in the U.S. Economy}

\begin{center}
    \includegraphics[width=\textwidth]{pictures/chap24/T4.png}
\end{center}

\end{frame}

%==========================================================
\begin{frame}{Case Study: Interest Rates in the U.S. Economy}

This figure shows \textbf{nominal and real interest rates} in the U.S. economy since 1965.

\begin{tcolorbox}[colframe=blue!80!black, colback=white, title=\textbf{How the Data Are Defined}]
\begin{itemize}
    \item Nominal interest rate: rate on three-month Treasury bills.
    \item Inflation rate: percentage change in the CPI.
    \item Real interest rate:
    \[
    \text{Real interest rate} = \text{Nominal interest rate} - \text{Inflation rate}
    \]
\end{itemize}
\end{tcolorbox}

In most years, the nominal interest rate exceeds the real interest rate because inflation is usually positive.

\end{frame}

%==========================================================
\begin{frame}{Key Patterns in Real and Nominal Interest Rates}

Because inflation varies over time, real and nominal interest rates do not always move together.

\begin{tcolorbox}[colframe=red!80!black, colback=white, title=\textbf{Historical Insights}]
\begin{itemize}
    \item \textbf{1970s}:
    \begin{itemize}
        \item Nominal interest rates were high.
        \item Inflation was even higher.
        \item Real interest rates were low or negative.
    \end{itemize}
    \item \textbf{Late 1990s}:
    \begin{itemize}
        \item Nominal interest rates were lower.
        \item Inflation was very low.
        \item Real interest rates were relatively high.
    \end{itemize}
\end{itemize}
\end{tcolorbox}

\textbf{Deflation}: When prices fall, the real interest rate exceeds the nominal interest rate.

\end{frame}
%==========================================================
\section{24-3 Conclusion}
%==========================================================
\begin{frame}{Conclusion: Key Takeaways}

\begin{tcolorbox}[colframe=blue!80!black, colback=white, title=\textbf{What We Learned}]
\begin{itemize}
    \item Inflation reduces the purchasing power of money.
    \item Price indexes measure changes in the overall price level.
    \item CPI and the GDP deflator allow comparisons across time.
    \item Economic variables must be corrected for inflation.
\end{itemize}
\end{tcolorbox}

Price indexes help us better understand how the economy changes over time.

\end{frame}

%==========================================================
%==========================================================

\end{document}