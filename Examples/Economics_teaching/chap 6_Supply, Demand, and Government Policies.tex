\documentclass{beamer}
\usepackage{graphicx} % Required for inserting images
\usepackage{amsmath}
\usepackage[most]{tcolorbox}
\usepackage{lmodern}
\usepackage{mathabx}

\usetheme{Madrid} % 可選其他主題:e.g., Warsaw, Berkeley, etc.
\usecolortheme{default}
\setbeamertemplate{caption}[numbered]% Number float-like environments
% Customize the caption
\setbeamerfont{caption}{size=\footnotesize}
% \setbeamercolor{caption}{fg=blue}
% \setbeamercolor{caption name}{fg=red}

% 每章節開始時自動產生章節頁
\AtBeginSection[]
{
  \begin{frame}
    \frametitle{Table of Contents}
    \tableofcontents[currentsection]
  \end{frame}
}


\title{Mankiw's Principles of Economics}
\subtitle{chap 6: Supply, Demand, and Government Policies}
\author{Hsu Chun-Wei}
\date{July 2025}

\begin{document}

\maketitle

% 目錄頁
\begin{frame}
  \frametitle{Table of Contents}
  \tableofcontents
\end{frame}

%==========================================================
\section{Controls on Prices}
%==========================================================
\begin{frame}{Price Controls: The Ice Cream Market Example}

Consider a competitive market for ice cream with no government regulation.
In equilibrium, the price adjusts to balance supply and demand, so the quantity buyers want to purchase equals the quantity sellers want to sell.
Suppose the equilibrium price is \(\$3\) per cone.

\begin{tcolorbox}[colframe=blue!70!black, colback=white, title=\textbf{Equilibrium Outcome}]
At the equilibrium price, the market clears:
\begin{itemize}
    \item Buyers purchase exactly what sellers produce.
    \item No shortages or surpluses exist.
\end{itemize}
\end{tcolorbox}

Some groups are dissatisfied with this outcome.
Ice-cream consumers argue that \(\$3\) is too high, while ice-cream producers argue that \(\$3\) is too low and reduces their income.
Both groups lobby the government to intervene by controlling prices.

\end{frame}

%==========================================================
\begin{frame}{Price Controls}
\begin{tcolorbox}[colframe=red!80!black, colback=white, title=\textbf{Price Controls}]
\begin{itemize}
    \item \textbf{Price Ceiling}: a legal maximum price, imposed to help buyers.
    \item \textbf{Price Floor}: a legal minimum price, imposed to help sellers.
\end{itemize}
\end{tcolorbox}

These policies prevent prices from adjusting freely and lead to different market outcomes.
\end{frame}
%==========================================================
\begin{frame}{How Price Ceilings Affect Market Outcomes}

When the government imposes a \textbf{price ceiling}, two outcomes are possible,
depending on whether the ceiling is above or below the equilibrium price.

\begin{columns}

\column{0.5\textwidth}
\textbf{Case 1: Price Ceiling Is Not Binding}
\begin{itemize}
    \item Example: Price ceiling = \$4, equilibrium price = \$3.
    \item The market price is below the ceiling.
    \item The ceiling has \textbf{no effect} on price or quantity.
    \item The market reaches equilibrium normally.
\end{itemize}

\column{0.5\textwidth}
\begin{center}
    \includegraphics[width=\textwidth]{pictures/chap 6/T1.png}
\end{center}

\end{columns}

\end{frame}

%==========================================================
\begin{frame}{How Price Ceilings Affect Market Outcomes}

\begin{columns}

\column{0.5\textwidth}
\textbf{Case 2: Price Ceiling Is Binding}
\begin{itemize}
    \item Example:
    Price ceiling = \$2, equilibrium price = \$3.
    \item The ceiling is a \textbf{binding constraint}.
    \item The market price cannot rise to equilibrium.
    \item Quantity demanded exceeds quantity supplied.
\end{itemize}

\column{0.5\textwidth}
\begin{center}
    \includegraphics[width=\textwidth]{pictures/chap 6/T2.png}
\end{center}
\end{columns}

\end{frame}
%==========================================================
\begin{frame}{Shortages and Rationing under a Binding Price Ceiling}

When a binding price ceiling is imposed, the market price equals the ceiling.
At this lower price, quantity demanded exceeds quantity supplied.

\begin{tcolorbox}[colframe=red!80!black, colback=white, title=\textbf{Market Outcome}]
\begin{itemize}
    \item Quantity demanded: 125 cones
    \item Quantity supplied: 75 cones
    \item \textbf{Shortage: 50 cones}
\end{itemize}
\end{tcolorbox}

Because prices can no longer ration goods, other rationing mechanisms emerge.

\end{frame}

%==========================================================
\begin{frame}{Shortages and Rationing under a Binding Price Ceiling}
\begin{tcolorbox}[colframe=blue!70!black, colback=white, title=\textbf{Rationing Mechanisms}]
\begin{itemize}
    \item Long waiting lines
    \item First-come, first-served
    \item Seller favoritism or discrimination
\end{itemize}
\end{tcolorbox}

These outcomes are often inefficient and potentially unfair.
In contrast, free markets ration goods efficiently through prices.
\end{frame}
%==========================================================
\begin{frame}{Case Study: Lines at the Gas Pump}

In 1973, the Organization of Petroleum Exporting Countries (OPEC) raised the price of crude oil.
Because crude oil is a key input in gasoline production, higher oil prices increased production costs and reduced the supply of gasoline.

\begin{tcolorbox}[colframe=blue!70!black, colback=white, title=\textbf{Initial Situation}]
\begin{itemize}
    \item Before the oil price increase, the equilibrium price of gasoline was below the price ceiling.
    \item The price ceiling was \textbf{not binding}.
    \item The gasoline market functioned normally.
\end{itemize}
\end{tcolorbox}
Many people blamed OPEC for the long lines at gas stations.
However, economists argue that government price controls played a central role.
\end{frame}

%==========================================================
\begin{frame}{Case Study: Lines at the Gas Pump}

\begin{tcolorbox}[colframe=red!80!black, colback=white, title=\textbf{Key Question}]
If oil prices rose, why didn’t gasoline prices adjust to clear the market?
\end{tcolorbox}
\begin{center}
    \includegraphics[width=\textwidth]{pictures/chap 6/T3.png}
\end{center}
\end{frame}
%==========================================================
\begin{frame}{Price Ceilings, Supply Shocks, and Shortages}

When the price of crude oil increased, the supply of gasoline shifted left from \(S_1\) to \(S_2\).

\textbf{What Went Wrong?}
\begin{itemize}
    \item In an unregulated market, the price would rise from \(P_1\) to \(P_2\).
    \item The price ceiling prevented the price from increasing.
    \item The price ceiling became \textbf{binding}.
\end{itemize}

At the regulated price:
\begin{itemize}
    \item Quantity demanded = \(Q_D\)
    \item Quantity supplied = \(Q_S\)
    \item \textbf{Shortage = \(Q_D - Q_S\)}
\end{itemize}
\begin{tcolorbox}[colframe=blue!70!black, colback=white, title=\textbf{Lesson from the Gasoline Market}]
When the government imposes a binding price ceiling during a negative supply shock,
severe shortages and long waiting lines can result.
\end{tcolorbox}
\end{frame}

%==========================================================
\begin{frame}{How Price Floors Affect Market Outcomes}

A \textbf{price floor} is a legal minimum price set by the government.
Like price ceilings, price floors attempt to control market outcomes.
\begin{columns}

\column{0.55\textwidth}
\begin{tcolorbox}[colframe=blue!70!black, colback=white, title=\textbf{Case1:Non-Binding Price Floor}]
\begin{itemize}
    \item Example:\\
    Price floor = \$2,\\
    equilibrium price = \$3.
    \item The equilibrium price is above the floor.
    \item The price floor has \textbf{no effect}.
    \item The market reaches equilibrium naturally.
\end{itemize}
\end{tcolorbox}

\column{0.45\textwidth}
\begin{center}
    \includegraphics[width=\textwidth]{pictures/chap 6/T4.png}
\end{center}
\end{columns}

\end{frame}
%==========================================================
\begin{frame}{How Price Floors Affect Market Outcomes}

\begin{columns}

\column{0.5\textwidth}
\begin{tcolorbox}[colframe=red!80!black, colback=white, title=\textbf{Case2: Binding Price Floor}]
\begin{itemize}
    \item Example:\\
    Price floor = \$4,\\
    equilibrium price = \$3.
    \item The floor is a \textbf{binding constraint}.
    \item The market price cannot fall to equilibrium.
    \item Quantity supplied exceeds quantity demanded.
\end{itemize}
\end{tcolorbox}

\column{0.5\textwidth}
\begin{center}
    \includegraphics[width=\textwidth]{pictures/chap 6/T5.png}
\end{center}
\end{columns}
\end{frame}

%==========================================================
\begin{frame}{Surpluses under a Binding Price Floor}

When a binding price floor is imposed, the market price equals the floor.

\begin{tcolorbox}[colframe=red!80!black, colback=white, title=\textbf{Market Outcome}]
\begin{itemize}
    \item Quantity supplied: 120 cones
    \item Quantity demanded: 80 cones
    \item \textbf{Surplus: 40 cones}
\end{itemize}
\end{tcolorbox}

At the price floor, some sellers are unable to sell their goods.
As a result, alternative rationing mechanisms emerge.

\begin{tcolorbox}[colframe=blue!70!black, colback=white, title=\textbf{Rationing under a Price Floor}]
\begin{itemize}
    \item Sellers compete for buyers
    \item Favoritism or personal connections may matter
    \item Goods may not go to buyers who value them most
\end{itemize}
\end{tcolorbox}

\end{frame}

%==========================================================
\begin{frame}{The Minimum Wage}

The \textbf{minimum wage} is a legal minimum price for labor.
It is an important example of a \textbf{price floor}.

\begin{tcolorbox}[colframe=blue!70!black, colback=white, title=\textbf{Definition and Purpose}]
\begin{itemize}
    \item Minimum-wage laws set the lowest wage employers may legally pay.
    \item First introduced in the United States in 1938.
    \item Intended to ensure workers a minimally adequate standard of living.
\end{itemize}
\end{tcolorbox}

In the labor market:
\begin{itemize}
    \item Workers determine the supply of labor.
    \item Firms determine the demand for labor.
    \item Without intervention, wages adjust to balance supply and demand.
\end{itemize}

\end{frame}
%==========================================================
\begin{frame}{The Minimum Wage}
\begin{center}
    \includegraphics[width=\textwidth]{pictures/chap 6/T6.png}
\end{center}
\end{frame}
%==========================================================
\begin{frame}{Minimum Wage and Unemployment}

If the minimum wage is set \textbf{above} the equilibrium wage,
it becomes a \textbf{binding price floor}.

\begin{tcolorbox}[colframe=red!80!black, colback=white, title=\textbf{Market Outcome}]
\begin{itemize}
    \item Wage cannot fall to equilibrium.
    \item Quantity of labor supplied exceeds quantity demanded.
    \item The result is \textbf{unemployment}.
\end{itemize}
\end{tcolorbox}

The minimum wage:
\begin{itemize}
    \item Raises incomes for workers who keep their jobs.
    \item Reduces job opportunities for some workers.
\end{itemize}

\end{frame}
%==========================================================
\begin{frame}{Evidence and Debate over the Minimum Wage}

\begin{tcolorbox}[colframe=blue!70!black, colback=white, title=\textbf{Who Is Most Affected?}]
The minimum wage is most likely to be binding for:
\begin{itemize}
    \item Low-skilled workers
    \item Inexperienced workers
    \item Teenagers
\end{itemize}
\end{tcolorbox}

Studies of the teenage labor market find that:
\begin{itemize}
    \item A 10\% increase in the minimum wage
    \item Reduces teenage employment by about 1--3\%.
\end{itemize}
\end{frame}

%==========================================================
\begin{frame}{Evidence and Debate over the Minimum Wage}
    \begin{tcolorbox}[colframe=blue!70!black, colback=white, title=\textbf{Why the Effect Is Limited}]
\begin{itemize}
    \item Not all workers earn the minimum wage.
    \item Enforcement is imperfect.
    \item Some wages are already above the minimum.
\end{itemize}
\end{tcolorbox}

\begin{tcolorbox}[colframe=red!80!black, colback=white, title=\textbf{The Ongoing Debate}]
\begin{itemize}
    \item \textbf{Supporters}: Raise incomes of the working poor.
    \item \textbf{Opponents}: Cause unemployment and are poorly targeted.
\end{itemize}
\end{tcolorbox}

Economists remain divided on whether the benefits outweigh the costs.

\end{frame}
%==========================================================
\section{Taxes}
%==========================================================
\begin{frame}{How Taxes on Sellers Affect Market Outcomes}

Suppose the government levies a \(\$0.50\) tax on sellers of ice-cream cones.
To analyze the effect, we follow three steps.

\textbf{Step 1: Which Curve Is Affected?}
\begin{itemize}
    \item The tax is levied on sellers, not buyers.
    \item Quantity demanded at each price is unchanged.
    \item The \textbf{demand curve does not shift}.
    \item The tax makes selling ice cream less profitable.
\end{itemize}

\textbf{Step 2: Direction and Size of the Shift}
\begin{itemize}
    \item The tax raises the cost of producing and selling ice cream.
    \item Quantity supplied falls at every price.
    \item The \textbf{supply curve shifts left (upward)}.
    \item The shift is exactly equal to the size of the tax (\(\$0.50\)).
\end{itemize}

\end{frame}
%==========================================================
\begin{frame}{How Taxes on Sellers Affect Market Outcomes}
    \begin{center}
        \includegraphics[width=0.85\textwidth]{pictures/chap 6/T7.png}
    \end{center}
\end{frame}
%==========================================================
\begin{frame}{Taxes on Sellers: New Equilibrium and Incidence}

After the supply curve shifts, the market reaches a new equilibrium.

\textbf{Step 3: New Equilibrium}
\begin{itemize}
    \item Price paid by buyers rises from \$3.00 to \$3.30.
    \item Quantity sold falls from 100 to 90 cones.
    \item The size of the market shrinks.
\end{itemize}

Although sellers send the entire tax to the government,
buyers and sellers share the burden.

\begin{tcolorbox}[colframe=blue!70!black, colback=white, title=\textbf{Tax Incidence}]
\begin{itemize}
    \item Buyers pay \$0.30 more per cone.
    \item Sellers receive \$2.80 after tax (\$3.30 -\$0.50).
    \item Both buyers and sellers are worse off.
\end{itemize}
\end{tcolorbox}

\begin{itemize}
    \item Taxes discourage market activity.
    \item Taxes reduce the quantity traded.
\end{itemize}
\end{frame}
%==========================================================
\begin{frame}{How Taxes on Buyers Affect Market Outcomes}

Suppose the government levies a \(\$0.50\) tax on buyers of ice-cream cones.
To analyze the effect, we again follow three steps.

\textbf{Step 1: Which Curve Is Affected?}
\begin{itemize}
    \item The tax is levied on buyers, not sellers.
    \item Sellers face the same incentives at any given price.
    \item The \textbf{supply curve does not change}.
    \item Buyers must now pay a tax in addition to the market price.
\end{itemize}

\textbf{Step 2: Direction and Size of the Shift}
\begin{itemize}
    \item The tax makes buying ice cream less attractive.
    \item Quantity demanded falls at every price.
    \item The \textbf{demand curve shifts left (downward)}.
    \item The shift equals the size of the tax (\(\$0.50\)).
\end{itemize}

\end{frame}
%==========================================================
\begin{frame}{How Taxes on Buyers Affect Market Outcomes}
    \begin{center}
        \includegraphics[width=0.85\textwidth]{pictures/chap 6/T8.png}
    \end{center}
\end{frame}
%==========================================================
\begin{frame}{Taxes on Buyers: New Equilibrium and Implications}

After the demand curve shifts, the market reaches a new equilibrium.

\textbf{Step 3: New Equilibrium}
\begin{itemize}
    \item Market price falls from \$3.00 to \$2.80.
    \item Quantity sold falls from 100 to 90 cones.
    \item The size of the market shrinks.
\end{itemize}

Although sellers receive a lower market price,
buyers pay a higher \textbf{effective price}.

\begin{tcolorbox}[colframe=blue!70!black, colback=white, title=\textbf{Tax Incidence}]
\begin{itemize}
    \item Sellers receive \$2.80 per cone.
    \item Buyers pay \$3.30 including the tax.
    \item Buyers and sellers share the burden of the tax.
\end{itemize}
\end{tcolorbox}

\end{frame}
%==========================================================
\begin{frame}{Can Congress Distribute the Burden of a Payroll Tax?}

A \textbf{payroll tax} is a tax on wages paid to workers.
An important example in the United States is the \textbf{FICA tax},
which funds Social Security and Medicare.

\begin{tcolorbox}[colframe=blue!70!black, colback=white, title=\textbf{Key Facts}]
\begin{itemize}
    \item Payroll taxes are levied on labor income.
    \item In 2013, the total FICA tax was 15.3\% of earnings.
    \item By law, half is paid by firms and half by workers.
\end{itemize}
\end{tcolorbox}

This legal division raises an important question:
Who actually bears the burden of the payroll tax?

\end{frame}
%==========================================================
\begin{frame}{Payroll Tax}
A payroll tax can be analyzed as a tax on labor,
where the price is the wage.
    \begin{center}
        \includegraphics[width=0.8\textwidth]{pictures/chap 6/T9.png}
    \end{center}
\end{frame}
%==========================================================
\begin{frame}{Payroll Taxes and Tax Incidence}

\begin{itemize}
    \item The tax creates a \textbf{wedge} between:
    \begin{itemize}
        \item the wage firms pay, and
        \item the wage workers receive.
    \end{itemize}
    \item Wages paid by firms rise.
    \item Wages received by workers fall.
\end{itemize}

\begin{tcolorbox}[colframe=blue!70!black, colback=white, title=\textbf{Key Lesson of Tax Incidence}]
\begin{itemize}
    \item Workers and firms share the burden of the tax.
    \item The division of the burden does \textbf{not} depend on how the law assigns the tax.
    \item The same outcome would occur if the tax were levied entirely on workers or firms.
\end{itemize}
\end{tcolorbox}

Lawmakers cannot easily dictate who bears the burden of a tax; market forces determine tax incidence.


\end{frame}

%==========================================================
\begin{frame}{Elasticity and Tax Incidence}

When a tax is imposed, buyers and sellers share the burden.
The key question is how that burden is divided.

\begin{tcolorbox}[colframe=blue!70!black, colback=white, title=\textbf{What Determines Tax Incidence?}]
\begin{itemize}
    \item The division of the tax burden does \textbf{not} depend on
    whether the tax is levied on buyers or sellers.
    \item It depends on the \textbf{relative elasticity} of supply and demand.
\end{itemize}
\end{tcolorbox}

\begin{tcolorbox}[colframe=red!80!black, colback=white, title=\textbf{Two Key Cases}]
\begin{itemize}
    \item Elastic supply, inelastic demand  
    \(\rightarrow\) Buyers bear most of the tax.
    \item Inelastic supply, elastic demand  
    \(\rightarrow\) Sellers bear most of the tax.
\end{itemize}
\end{tcolorbox}

\end{frame}
%==========================================================
\begin{frame}{Elastic supply, inelastic demand}
    \begin{center}
        \includegraphics[width=0.8\textwidth]{pictures/chap 6/T10.png}
    \end{center}
\end{frame}
%==========================================================
\begin{frame}{Inelastic supply, elastic demand}
    \begin{center}
        \includegraphics[width=0.8\textwidth]{pictures/chap 6/T11.png}
    \end{center}
\end{frame}
%==========================================================
\begin{frame}{Who Bears More of the Tax Burden?}

The two panels lead to a general lesson.

\vspace{0.5em}
\textbf{A tax burden falls more heavily on the side of the market that is less elastic.}
\vspace{0.5em}

\begin{tcolorbox}[colframe=blue!70!black, colback=white, title=\textbf{Economic Intuition}]
\begin{itemize}
    \item Elasticity measures how easily buyers or sellers can leave the market.
    \item The side with fewer alternatives is less willing to leave.
    \item That side must bear more of the tax burden.
\end{itemize}
\end{tcolorbox}

\begin{itemize}
    \item In labor markets, labor supply is often less elastic than labor demand.
    \item As a result, workers tend to bear most of the burden of payroll taxes.
\end{itemize}

\end{frame}
%==========================================================
\begin{frame}{Case Study: Who Pays the Luxury Tax?}

In 1990, Congress adopted a luxury tax on items such as yachts,
private airplanes, jewelry, and expensive cars.

\begin{tcolorbox}[colframe=blue!70!black, colback=white, title=\textbf{Policy Goal}]
\begin{itemize}
    \item Raise revenue from those who could most easily afford to pay.
    \item Tax goods primarily consumed by the rich.
    \item Make the tax system more progressive.
\end{itemize}
\end{tcolorbox}

At first glance, taxing luxury goods seemed like a logical way
to tax wealthy consumers.

However, once supply and demand responded,
the outcome was very different from what lawmakers intended.

\end{frame}

%==========================================================
\begin{frame}{Elasticity Explains the Outcome of the Luxury Tax}

\begin{tcolorbox}[colframe=red!80!black, colback=white, title=\textbf{Market Elasticities}]
\begin{itemize}
    \item Demand for yachts is \textbf{elastic}.
    \item Wealthy buyers have many good alternatives.
    \item Supply of yachts is relatively \textbf{inelastic} in the short run.
    \item Yacht builders and workers have fewer alternatives.
\end{itemize}
\end{tcolorbox}

\begin{tcolorbox}[colframe=blue!70!black, colback=white, title=\textbf{Tax Incidence Result}]
\begin{itemize}
    \item Most of the tax burden falls on producers.
    \item Firms receive a much lower price.
    \item Workers in luxury industries suffer income losses.
\end{itemize}
\end{tcolorbox}

As a result, the burden of the luxury tax fell more on workers
than on wealthy consumers.
Congress repealed most of the luxury tax in 1993.

\end{frame}

%==========================================================
%==========================================================
%==========================================================
%==========================================================
%==========================================================
%==========================================================
%==========================================================
%==========================================================
%==========================================================
%==========================================================
%==========================================================
\end{document}