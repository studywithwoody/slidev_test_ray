\documentclass{beamer}
\usepackage{graphicx} % Required for inserting images
\usepackage{amsmath}
\usepackage[most]{tcolorbox}
\usepackage{lmodern}
\usepackage{mathabx}

\usetheme{Madrid} % 可選其他主題:e.g., Warsaw, Berkeley, etc.
\usecolortheme{default}
\setbeamertemplate{caption}[numbered]% Number float-like environments
% Customize the caption
\setbeamerfont{caption}{size=\footnotesize}
% \setbeamercolor{caption}{fg=blue}
% \setbeamercolor{caption name}{fg=red}

\title{Mankiw's Principles of Economics}
\subtitle{chap 4: The Market Forces of Supply and Demand}
\author{Hsu Chun-Wei}
\date{July 2025}

\begin{document}

\maketitle
%==========================================================
\section{Changes in Equilibrium}
%==========================================================
\begin{frame}{Three Steps to Analyzing Changes in Equilibrium}

The equilibrium price and quantity depend on the positions of the supply and demand 
curves. When an event shifts one of these curves, the market reaches a new equilibrium 
with a new price and quantity.

To analyze how a particular event affects equilibrium, we follow three systematic steps:

\begin{tcolorbox}[colframe=green!60!black, colback=white, title=\textbf{Three-Step Method}]
\begin{enumerate}
    \item Decide whether the event shifts the \textbf{supply curve}, the \textbf{demand curve}, or both.
    \item Determine the \textbf{direction} of the shift (rightward or leftward).
    \item Use a supply-and-demand diagram to compare the initial equilibrium with the 
    new equilibrium and determine how price and quantity change.
\end{enumerate}
\end{tcolorbox}

This framework helps us predict how various events affect market outcomes.
\end{frame}

%==========================================================
\begin{frame}{Example: Change in Equilibrium Due to a Shift in Demand}

Suppose the summer is unusually hot. How does this affect the market for ice cream?  
We use the three-step method:

\begin{enumerate}
    \item \textbf{Determine which curve shifts.}  
    Hot weather changes consumers’ taste, increasing their desire for ice cream.  
    Thus, the \textbf{demand curve} shifts, while the supply curve remains unchanged.

    \item \textbf{Determine the direction of the shift.}  
    Because hot weather increases the amount consumers want to buy at any price,  
    the demand curve shifts \textbf{to the right} from $D_1$ to $D_2$.

    \item \textbf{Use the diagram to determine the new equilibrium.}  
    At the old price of \$2.00, there is now excess demand.  
    Firms respond by raising prices.  
    The new equilibrium price rises to \textbf{\$2.50},  
    and the equilibrium quantity rises from \textbf{7 to 10 cones}.
\end{enumerate}

\begin{tcolorbox}[colframe=orange!80!black, colback=white, title=\textbf{Result}]
Hot weather \textbf{increases demand}, which \textbf{raises both equilibrium price and quantity}.
\end{tcolorbox}

\end{frame}

%==========================================================
\begin{frame}{How an Increase in Demand Affects Equilibrium}

An increase in demand shifts the curve to the right from $D_1$ to $D_2$, raising the 
equilibrium price from \$2.00 to \$2.50 and the equilibrium quantity from 7 to 10 cones.

\begin{center}
\includegraphics[width=0.7\textwidth]{pictures/chap4/T23.png}
\end{center}

\end{frame}

%==========================================================
\begin{frame}{Shifts in Curves versus Movements along Curves}

When hot weather increases the demand for ice cream, the demand curve shifts to the right. 
This raises the equilibrium price, and because of the higher price, firms supply a larger 
quantity. Although quantity supplied rises, the \textbf{supply curve itself does not shift}.

\begin{tcolorbox}[colframe=purple!80!black, colback=white, title=\textbf{Key Distinction}]
\begin{itemize}
    \item \textbf{Supply} refers to the entire supply curve (its position).
    \item \textbf{Quantity supplied} refers to the amount sellers wish to sell at a given price.
\end{itemize}
\end{tcolorbox}

In this example, weather affects consumers’ desire for ice cream, so it shifts the 
\textbf{demand curve}. The higher price then causes a \textbf{movement along} the supply curve.

\end{frame}

%==========================================================
\begin{frame}{Summary}
    \begin{tcolorbox}[colframe=blue!60!black, colback=white, title=\textbf{Summary}]
    \begin{itemize}
        \item A \textbf{shift in supply} = a “change in supply.”
        \item A \textbf{shift in demand} = a “change in demand.”
        \item A movement \textbf{along} a fixed supply curve = a “change in quantity supplied.”
        \item A movement \textbf{along} a fixed demand curve = a “change in quantity demanded.”
    \end{itemize}
    \end{tcolorbox}
\end{frame}
%==========================================================
\begin{frame}{Example: Change in Equilibrium Due to a Shift in Supply}

Suppose a hurricane destroys part of the sugarcane crop, driving up the price of sugar. 

\begin{enumerate}
    \item \textbf{Identify which curve shifts.}  
    Sugar is an input. A higher input price raises production costs and reduces the amount 
    firms are willing to supply. Thus, the \textbf{supply curve shifts}, while demand remains unchanged.

    \item \textbf{Determine the direction of the shift.}  
    Because firms can produce less ice cream at every price, the supply curve shifts 
    \textbf{to the left}, from $S_1$ to $S_2$.

    \item \textbf{Compare the initial and new equilibrium.}  
    At the old price of \$2.00, there is now excess demand—a shortage. Firms raise prices.  
    The new equilibrium price rises to \textbf{\$2.50}, and the equilibrium quantity falls 
    from \textbf{7 to 4 cones}.
\end{enumerate}

\textbf{Result}

An increase in input prices decreases supply, raising the equilibrium price and lowering 
the equilibrium quantity.

\end{frame}

%==========================================================
\begin{frame}{How a Decrease in Supply Affects the Equilibrium}

A leftward shift of supply from $S_1$ to $S_2$ increases the equilibrium price from 
\$2.00 to \$2.50 and decreases the equilibrium quantity from 7 to 4 cones.

\begin{center}
\includegraphics[width=0.7\textwidth]{pictures/chap4/T24.png}
\end{center}

\end{frame}
%==========================================================
\begin{frame}{Example: Shifts in Both Supply and Demand}

Suppose that during the same summer there is both a heat wave (increasing demand) 
and a hurricane that raises sugar prices (decreasing supply). To analyze this, we 
follow the three-step method:

\begin{enumerate}
    \item \textbf{Identify which curves shift.}  
    Hot weather increases consumers’ desire for ice cream → \textbf{demand shifts right}.  
    Higher sugar prices raise production costs → \textbf{supply shifts left}.

    \item \textbf{Determine the direction of each shift.}  
    The demand curve shifts to the right (from $D_1$ to $D_2$).  
    The supply curve shifts to the left (from $S_1$ to $S_2$).

    \item \textbf{Compare the initial and new equilibrium.}  
    Because demand rises and supply falls, \textbf{equilibrium price rises for sure}.  
    The change in equilibrium quantity depends on the relative sizes of the shifts:
    \begin{itemize}
        \item If demand increases more than supply decreases → quantity rises.
        \item If supply decreases more than demand increases → quantity falls.
    \end{itemize}
\end{enumerate}

\end{frame}

%==========================================================
\begin{frame}{Simultaneous Shifts in Supply and Demand}

When demand rises and supply falls at the same time:
\begin{itemize}
    \item The equilibrium price rises.
    \item The equilibrium quantity may rise or fall depending on which shift is larger.
\end{itemize}

\begin{center}
\includegraphics[width=\textwidth]{pictures/chap4/T25.png}
\end{center}

\end{frame}
%==========================================================
\begin{frame}{Summary}
    \begin{center}
        \includegraphics[width=\textwidth]{pictures/chap4/T26.png}
    \end{center}
\end{frame}
%==========================================================
%==========================================================
%==========================================================
%==========================================================
%==========================================================
\end{document}