\documentclass{beamer}
\usepackage{graphicx} % Required for inserting images
\usepackage{amsmath}
\usepackage[most]{tcolorbox}
\usepackage{lmodern}
\usepackage{mathabx}

\usetheme{Madrid} % 可選其他主題:e.g., Warsaw, Berkeley, etc.
\usecolortheme{default}
\setbeamertemplate{caption}[numbered]% Number float-like environments
% Customize the caption
\setbeamerfont{caption}{size=\footnotesize}
% \setbeamercolor{caption}{fg=blue}
% \setbeamercolor{caption name}{fg=red}

% 每章節開始時自動產生章節頁
\AtBeginSection[]
{
  \begin{frame}
    \frametitle{Table of Contents}
    \tableofcontents[currentsection]
  \end{frame}
}


\title{Mankiw's Principles of Economics}
\subtitle{chap 26: Saving, Investment, and the Financial System}
\author{Hsu Chun-Wei}
\date{July 2025}

\begin{document}

\maketitle

% 目錄頁
\begin{frame}
  \frametitle{Table of Contents}
  \tableofcontents
\end{frame}

%==========================================================
\section{The Market for Loanable Funds}
%==========================================================
\begin{frame}{The Market for Loanable Funds}

After discussing financial institutions and saving and investment, we now build a model of financial markets.

\begin{tcolorbox}[colframe=red!80!black, colback=white, title=\textbf{Purpose of the Model}]
The loanable funds model explains how financial markets coordinate saving and investment.
\end{tcolorbox}

\begin{itemize}
    \item Explains how saving is allocated to investment.
    \item Provides a tool to analyze government policies.
    \item Focuses on the role of the interest rate.
\end{itemize}

\end{frame}

%==========================================================
\begin{frame}{The Market for Loanable Funds}

To keep the analysis simple, we assume the economy has a single financial market.

\begin{tcolorbox}[colframe=blue!70!black, colback=white, title=\textbf{Market for Loanable Funds}]
The market in which those who want to save supply funds and those who want to borrow to invest demand funds.
\end{tcolorbox}

\begin{itemize}
    \item Savers supply loanable funds.
    \item Borrowers demand loanable funds.
    \item All saving and borrowing take place in this market.
\end{itemize}

\end{frame}

%==========================================================
\begin{frame}{The Role of the Interest Rate}

In the market for loanable funds, there is a single interest rate.

\begin{tcolorbox}[colframe=green!70!black, colback=white, title=\textbf{Interest Rate}]
The interest rate is both:
\begin{itemize}
    \item The return to saving
    \item The cost of borrowing
\end{itemize}
\end{tcolorbox}

\begin{itemize}
    \item Higher interest rates encourage saving.
    \item Higher interest rates discourage borrowing and investment.
\end{itemize}

The interest rate adjusts to balance saving and investment.

\end{frame}

%==========================================================
\subsection{Supply and Demand for Loanable Funds}
%==========================================================
\begin{frame}{Supply of Loanable Funds}

The supply of loanable funds comes from people who have extra income they want to save.

\begin{tcolorbox}[colframe=blue!70!black, colback=white, title=\textbf{Supply of Loanable Funds}]
Saving is the source of the supply of loanable funds.
\end{tcolorbox}

\begin{itemize}
    \item Households can lend directly by buying bonds.
    \item Or lend indirectly by depositing money in banks.
    \item In both cases, saving provides funds to the market.
\end{itemize}

As the interest rate rises:
\begin{itemize}
    \item Saving becomes more attractive.
    \item Quantity of loanable funds supplied increases.
\end{itemize}

\end{frame}

%==========================================================
\begin{frame}{Demand for Loanable Funds}

The demand for loanable funds comes from households and firms that want to invest.

\begin{tcolorbox}[colframe=red!80!black, colback=white, title=\textbf{Demand of Loanable Funds}]
Investment is the source of the demand for loanable funds.
\end{tcolorbox}

\begin{itemize}
    \item Households borrow to buy new homes.
    \item Firms borrow to buy equipment or build factories.
\end{itemize}

As the interest rate rises:
\begin{itemize}
    \item Borrowing becomes more expensive.
    \item Quantity of loanable funds demanded decreases.
\end{itemize}

\end{frame}

%==========================================================
\begin{frame}{Interest Rate and Equilibrium}

The interest rate is the price of a loan.

\begin{tcolorbox}[colframe=green!70!black, colback=white, title=\textbf{Interest Rate}]
The interest rate is:
\begin{itemize}
    \item The return to saving
    \item The cost of borrowing
\end{itemize}
\end{tcolorbox}

\begin{itemize}
    \item Supply curve slopes upward.
    \item Demand curve slopes downward.
    \item The equilibrium interest rate balances saving and investment.
\end{itemize}

If the interest rate is too low:
\begin{itemize}
    \item Shortage of loanable funds
    \item Interest rate rises
\end{itemize}

If the interest rate is too high:
\begin{itemize}
    \item Surplus of loanable funds
    \item Interest rate falls
\end{itemize}

\end{frame}
%==========================================================
\begin{frame}{Interest Rate and Equilibrium}

\begin{center}
    \includegraphics[width=0.8\textwidth]{pictures/chap26/T1.png}
\end{center}

\end{frame}
%==========================================================
\begin{frame}{Real vs Nominal Interest Rate}

Economists distinguish between nominal and real interest rates.

\begin{tcolorbox}[colframe=purple!70!black, colback=white, title=\textbf{Interest Rates}]
\begin{itemize}
    \item Nominal interest rate: as reported in the news
    \item Real interest rate: nominal rate minus inflation
\end{itemize}
\end{tcolorbox}

\[
\text{Real interest rate} = \text{Nominal interest rate} - \text{Inflation rate}
\]

\begin{itemize}
    \item Inflation erodes the value of money over time.
    \item The real interest rate reflects the true cost of borrowing.
\end{itemize}

The loanable funds model uses the \textbf{real interest rate}.

\end{frame}

%==========================================================
\begin{frame}{Using the Loanable Funds Model}

The loanable funds model works like other supply-and-demand models.

\begin{tcolorbox}[colframe=orange!80!black, colback=white, title=\textbf{Policy Analysis Steps}]
\begin{enumerate}
    \item Determine whether the policy affects saving or investment.
    \item Decide whether supply or demand shifts.
    \item Use the diagram to analyze changes in interest rate and quantity.
\end{enumerate}
\end{tcolorbox}

When the interest rate adjusts:
\begin{itemize}
    \item Saving and investment are coordinated.
    \item The financial system directs saving to investment.
\end{itemize}

\end{frame}
%==========================================================
\subsection{policy impacts}
%==========================================================
\begin{frame}{Policy 1: Saving Incentives}

Many policymakers believe that the U.S. saving rate is too low.

\begin{tcolorbox}[colframe=red!80!black, colback=white, title=\textbf{Key Motivation}]
Higher saving leads to greater investment, higher productivity, and higher living standards.
\end{tcolorbox}

\begin{itemize}
    \item Saving is a key determinant of long-run economic growth.
    \item Tax laws may discourage saving by taxing interest and dividend income.
    \item Lower after-tax returns reduce the incentive to save.
\end{itemize}

\end{frame}

%==========================================================
\begin{frame}{Tax Incentives for Saving}

Governments can encourage saving by reducing taxes on interest income.

\begin{tcolorbox}[colframe=blue!70!black, colback=white, title=\textbf{Example Policy}]
Expanding tax-favored accounts, such as Individual Retirement Accounts (IRAs).
\end{tcolorbox}

\begin{itemize}
    \item Raises the after-tax return to saving.
    \item Makes saving more attractive at any given interest rate.
\end{itemize}

We now analyze this policy using the loanable funds model.

\end{frame}

%==========================================================
\begin{frame}{Saving Incentives and Loanable Funds}

\begin{tcolorbox}[colframe=green!70!black, colback=white, title=\textbf{Step 1: Which Curve Shifts?}]
Saving incentives affect the \textbf{supply of loanable funds}.
\end{tcolorbox}

\begin{itemize}
    \item Households save more at each interest rate.
    \item Supply of loanable funds increases.
\end{itemize}

\begin{tcolorbox}[colframe=orange!80!black, colback=white, title=\textbf{Step 2: Direction of Shift}]
The supply curve shifts to the \textbf{right} (from $S_1$ to $S_2$).
\end{tcolorbox}

The demand for loanable funds remains unchanged.

\end{frame}
%==========================================================
\begin{frame}{Effects of Saving Incentives}
\begin{center}
    \includegraphics[width=0.8\textwidth]{pictures/chap26/T2.png}
\end{center}
\end{frame}
%==========================================================
\begin{frame}{Effects of Saving Incentives}

\begin{tcolorbox}[colframe=purple!70!black, colback=white, title=\textbf{Step 3: New Equilibrium}]
An increase in saving lowers the interest rate and raises investment.
\end{tcolorbox}

\begin{itemize}
    \item Interest rate falls.
    \item Quantity of loanable funds increases.
    \item Investment rises due to lower borrowing costs.
\end{itemize}

\begin{tcolorbox}[colframe=red!70!black, colback=white, title=\textbf{Conclusion}]
Saving incentives lead to lower interest rates and greater investment.
\end{tcolorbox}

\end{frame}

%==========================================================
\begin{frame}{Policy 2: Investment Incentives}

Governments sometimes try to encourage firms to invest more in new capital.

\begin{tcolorbox}[colframe=red!80!black, colback=white, title=\textbf{Policy Tool}]
An investment tax credit makes investment more attractive to firms.
\end{tcolorbox}

\begin{itemize}
    \item Reduces the cost of building new factories or buying equipment.
    \item Increases the incentive to invest at any given interest rate.
\end{itemize}

We analyze this policy using the market for loanable funds.

\end{frame}

%==========================================================
\begin{frame}{Investment Incentives: Which Curve Shifts?}

\begin{tcolorbox}[colframe=blue!70!black, colback=white, title=\textbf{Step 1: Identify the Curve}]
Investment incentives affect the \textbf{demand for loanable funds}.
\end{tcolorbox}

\begin{itemize}
    \item Firms want to borrow more to invest.
    \item Investment rises at any given interest rate.
\end{itemize}

The supply of loanable funds remains unchanged because household saving is not directly affected.

\begin{tcolorbox}[colframe=green!70!black, colback=white, title=\textbf{Step 2: Direction}]
The demand curve for loanable funds shifts to the \textbf{right}.
\end{tcolorbox}

\begin{itemize}
    \item From $D_1$ to $D_2$
    \item Higher quantity of loanable funds demanded at every interest rate
\end{itemize}

\end{frame}

%==========================================================
\begin{frame}{Effects of Investment Incentives}
\begin{center}
    \includegraphics[width=0.8\textwidth]{pictures/chap26/T3.png}
\end{center}
\end{frame}
%==========================================================
\begin{frame}{Effects of Investment Incentives}

\begin{tcolorbox}[colframe=purple!70!black, colback=white, title=\textbf{Step 3: New Equilibrium}]
An increase in investment demand raises the interest rate.
\end{tcolorbox}

\begin{itemize}
    \item Real interest rate increases.
    \item Quantity of loanable funds increases.
    \item Higher interest rates induce households to save more.
\end{itemize}

\begin{tcolorbox}[colframe=orange!80!black, colback=white, title=\textbf{Conclusion}]
Investment incentives lead to higher interest rates and greater saving.
\end{tcolorbox}

\end{frame}

%==========================================================
\begin{frame}{Policy 3: Government Budget Deficits and Surpluses}

Government budgets are a central topic in economic policy debates.

\begin{tcolorbox}[colframe=red!80!black, colback=white, title=\textbf{Key Definitions}]
\begin{itemize}
    \item \textbf{Budget deficit}: government spending exceeds tax revenue
    \item \textbf{Budget surplus}: tax revenue exceeds government spending
    \item \textbf{Balanced budget}: spending equals tax revenue
\end{itemize}
\end{tcolorbox}

Governments finance budget deficits by borrowing in financial markets, which adds to government debt.

\end{frame}

%==========================================================
\begin{frame}{Budget Deficit: Which Curve Shifts?}

To analyze a budget deficit, we use the loanable funds model.

\begin{tcolorbox}[colframe=blue!70!black, colback=white, title=\textbf{Step 1: Identify the Curve}]
A change in the government budget affects \textbf{public saving}.
\end{tcolorbox}

\begin{itemize}
    \item National saving = private saving + public saving
    \item A budget deficit reduces public saving
\end{itemize}

Therefore, the budget deficit affects the \textbf{supply of loanable funds}, not demand.


\end{frame}

%==========================================================
\begin{frame}{Budget Deficit: Direction of Shift}

\begin{tcolorbox}[colframe=green!70!black, colback=white, title=\textbf{Step 2: Direction}]
A budget deficit reduces national saving.
\end{tcolorbox}

\begin{itemize}
    \item Supply of loanable funds decreases
    \item Supply curve shifts to the \textbf{left} (from $S_1$ to $S_2$)
\end{itemize}

The demand for loanable funds remains unchanged.

\end{frame}
%==========================================================
\begin{frame}{Effects of a Budget Deficit}

\begin{center}
    \includegraphics[width=0.8\textwidth]{pictures/chap26/T4.png}
\end{center}
\end{frame}

%==========================================================
\begin{frame}{Effects of a Budget Deficit}

\begin{tcolorbox}[colframe=orange!80!black, colback=white, title=\textbf{Step 3: New Equilibrium}]
A decrease in the supply of loanable funds raises the interest rate.
\end{tcolorbox}

\begin{itemize}
    \item Real interest rate increases
    \item Quantity of loanable funds decreases
    \item Investment falls
\end{itemize}

\begin{tcolorbox}[colframe=red!70!black, colback=white, title=\textbf{Crowding Out}]
Government borrowing raises interest rates and reduces private investment.
\end{tcolorbox}

\end{frame}

%==========================================================
\begin{frame}{Budget Surpluses}

A budget surplus has the opposite effects of a budget deficit.

\begin{tcolorbox}[colframe=purple!70!black, colback=white, title=\textbf{Budget Surplus}]
A budget surplus increases public saving and national saving.
\end{tcolorbox}

\begin{itemize}
    \item Supply of loanable funds increases
    \item Interest rate falls
    \item Investment rises
\end{itemize}

Higher investment leads to greater capital accumulation and faster long-run economic growth.

\end{frame}

%==========================================================
\begin{frame}{The History of U.S. Government Debt}

The level of U.S. government debt has varied substantially over time.

\begin{tcolorbox}[colframe=red!80!black, colback=white, title=\textbf{Debt-to-GDP Ratio}]
The debt-to-GDP ratio measures government debt relative to the economy’s ability to raise tax revenue.
\end{tcolorbox}

\begin{itemize}
    \item A falling debt-to-GDP ratio suggests the government is living within its means.
    \item A rising debt-to-GDP ratio suggests debt is growing faster than tax capacity.
\end{itemize}

Because GDP is a rough measure of the tax base, the debt-to-GDP ratio is a key indicator of fiscal sustainability.

\end{frame}

%==========================================================
\begin{frame}{The History of U.S. Government Debt}

\begin{figure}
    \centering
    \includegraphics[width=0.7\linewidth]{pictures/chap26/T5.png}
    \caption{The U.S. Government Debt}
\end{figure}

\end{frame}
%==========================================================
\begin{frame}{War as a Major Cause of Government Debt}

Throughout U.S. history, wars have been the primary cause of large increases in government debt.

\begin{tcolorbox}[colframe=blue!70!black, colback=white, title=\textbf{War and Deficits}]
Wars raise government spending sharply, while taxes typically rise by less.
\end{tcolorbox}

\begin{itemize}
    \item Budget deficits increase during wars.
    \item Debt-to-GDP rises.
    \item After wars, spending falls and the debt-to-GDP ratio declines.
\end{itemize}

Economists often view debt financing of wars as appropriate:
\begin{itemize}
    \item It smooths tax rates over time.
    \item It spreads the cost of war across generations.
\end{itemize}

\end{frame}
%==========================================================
\begin{frame}{Recent Debt Growth and Economic Implications}

The debt-to-GDP ratio rose sharply after 2008.

\begin{tcolorbox}[colframe=orange!80!black, colback=white, title=\textbf{Financial Crisis and Recession}]
The financial crisis and deep recession increased deficits through:
\begin{itemize}
    \item Lower tax revenue
    \item Higher government spending
\end{itemize}
\end{tcolorbox}

\begin{itemize}
    \item Large deficits reduce national saving.
    \item Reduced saving raises interest rates.
    \item Higher interest rates crowd out private investment.
\end{itemize}

\begin{tcolorbox}[colframe=red!70!black, colback=white, title=\textbf{Big Picture}]
Persistent budget deficits can slow long-run economic growth.
\end{tcolorbox}

\end{frame}

%==========================================================
\begin{frame}{Financial Crises}

In 2008 and 2009, the U.S. and many other economies experienced a major financial crisis.

\begin{tcolorbox}[colframe=red!80!black, colback=white, title=\textbf{Financial Crisis}]
A financial crisis is a disruption of the financial system that prevents it from allocating saving to investment efficiently.
\end{tcolorbox}

Financial crises often lead to:
\begin{itemize}
    \item Reduced lending
    \item Lower investment
    \item Economic downturns
\end{itemize}

\end{frame}

%==========================================================
\begin{frame}{Elements of a Financial Crisis}

\begin{tcolorbox}[colframe=blue!70!black, colback=white, title=\textbf{1. Decline in Asset Prices}]
Large drops in asset prices reduce the value of collateral and wealth.
\end{tcolorbox}

\begin{itemize}
    \item In 2008–2009, housing prices fell sharply.
\end{itemize}

\begin{tcolorbox}[colframe=green!70!black, colback=white, title=\textbf{2. Insolvency of Financial Institutions}]
Falling asset values lead to loan defaults and threaten banks with bankruptcy.
\end{tcolorbox}

\begin{itemize}
    \item Mortgage defaults weaken banks’ balance sheets.
\end{itemize}

\end{frame}

%==========================================================
\begin{frame}{Elements of a Financial Crisis}

\begin{tcolorbox}[colframe=orange!80!black, colback=white, title=\textbf{3. Loss of Confidence}]
Uncertainty about bank solvency reduces trust in financial institutions.
\end{tcolorbox}

\begin{itemize}
    \item Depositors and investors withdraw funds.
    \item Banks sell assets and reduce lending.
\end{itemize}

\begin{tcolorbox}[colframe=purple!70!black, colback=white, title=\textbf{4. Credit Crunch}]
Even creditworthy borrowers have trouble obtaining loans.
\end{tcolorbox}

\end{frame}

%==========================================================
\begin{frame}{Elements of a Financial Crisis (5 \& 6)}

\begin{tcolorbox}[colframe=red!70!black, colback=white, title=\textbf{5. Economic Downturn}]
Reduced lending lowers investment and aggregate demand.
\end{tcolorbox}

\begin{itemize}
    \item Output falls
    \item Unemployment rises
\end{itemize}

\begin{tcolorbox}[colframe=black!70!white, colback=white, title=\textbf{6. Vicious Cycle}]
Economic downturn further weakens firms and asset values.
\end{tcolorbox}

Financial problems and economic downturn reinforce each other.

\end{frame}

%==========================================================
\begin{frame}{Why Financial Crises Matter}

\begin{tcolorbox}[colframe=blue!80!black, colback=white, title=\textbf{Key Lesson}]
A well-functioning financial system is essential for economic stability and growth.
\end{tcolorbox}

\begin{itemize}
    \item Financial crises disrupt saving and investment.
    \item Government policy can help restore confidence.
    \item Financial systems usually recover, but the costs can be severe.
\end{itemize}

\end{frame}
    
%==========================================================
\begin{frame}{Summary}

\begin{tcolorbox}[colframe=red!80!black, colback=white, title=\textbf{Role of the Financial System}]
The financial system coordinates the economy’s saving and investment.
\end{tcolorbox}

\begin{itemize}
    \item \textbf{Savers} supply funds.
    \item \textbf{Borrowers} demand funds to finance investment.
    \item Financial markets and intermediaries match saving with investment.
\end{itemize}

\end{frame}
%==========================================================
\section{Conclusion}
%==========================================================
\begin{frame}{Summary}

\begin{tcolorbox}[colframe=blue!70!black, colback=white, title=\textbf{Key Institutions}]
\begin{itemize}
    \item \textbf{Financial Markets}: bond market (debt), stock market (equity)
    \item \textbf{Financial Intermediaries}: banks, mutual funds
\end{itemize}
\end{tcolorbox}

\begin{tcolorbox}[colframe=green!70!black, colback=white, title=\textbf{Key Accounting Identities}]
\[
Y = C + I + G \quad (\text{closed economy})
\]
\[
S = Y - C - G \quad \Rightarrow \quad S = I
\]
\end{tcolorbox}

\begin{itemize}
    \item Saving equals investment for the economy as a whole.
    \item $S = (Y - T - C) + (T - G)$
    \item National saving = private saving + public saving
\end{itemize}

\end{frame}

%==========================================================
\begin{frame}{Summary}

\begin{tcolorbox}[colframe=purple!70!black, colback=white, title=\textbf{Market for Loanable Funds}]
\begin{itemize}
    \item Supply of loanable funds = saving
    \item Demand for loanable funds = investment
    \item Price = \textbf{real interest rate}
\end{itemize}
\end{tcolorbox}

\begin{itemize}
    \item Interest rate adjusts to balance saving and investment.
    \item Higher interest rate: saving $\uparrow$, investment $\downarrow$
\end{itemize}

\end{frame}

%==========================================================
\begin{frame}{Summary}

\begin{tcolorbox}[colframe=orange!80!black, colback=white, title=\textbf{Policy Effects}]
\begin{itemize}
    \item \textbf{Saving incentives}: supply $\rightarrow$, interest rate $\downarrow$, investment $\uparrow$
    \item \textbf{Investment incentives}: demand $\rightarrow$, interest rate $\uparrow$, saving $\uparrow$
    \item \textbf{Budget deficit}: supply $\leftarrow$, interest rate $\uparrow$, investment $\downarrow$
\end{itemize}
\end{tcolorbox}

\begin{tcolorbox}[colframe=red!70!black, colback=white, title=\textbf{Crowding Out}]
Government budget deficits reduce national saving, raise interest rates, and crowd out private investment.
\end{tcolorbox}

\begin{itemize}
    \item Financial crises occur when the financial system fails to allocate saving to investment.
    \item A healthy financial system is essential for long-run economic growth.
\end{itemize}

\end{frame}

%==========================================================
%==========================================================

\end{document}