\documentclass{beamer}
\usepackage{graphicx} % Required for inserting images
\usepackage{amsmath}
\usepackage[most]{tcolorbox}
\usepackage{lmodern}
\usepackage{mathabx}

\usetheme{Madrid} % 可選其他主題:e.g., Warsaw, Berkeley, etc.
\usecolortheme{default}
\setbeamertemplate{caption}[numbered]% Number float-like environments
% Customize the caption
\setbeamerfont{caption}{size=\footnotesize}
% \setbeamercolor{caption}{fg=blue}
% \setbeamercolor{caption name}{fg=red}

% 每章節開始時自動產生章節頁
\AtBeginSection[]
{
  \begin{frame}
    \frametitle{Table of Contents}
    \tableofcontents[currentsection]
  \end{frame}
}


\title{Mankiw's Principles of Economics}
\subtitle{chap 18: The Markets for the Factors of Production}
\author{Hsu Chun-Wei}
\date{July 2025}

\begin{document}

\maketitle

% 目錄頁
\begin{frame}
  \frametitle{Table of Contents}
  \tableofcontents
\end{frame}

%==========================================================
\section{18-1 The Demand for Labor}
%==========================================================
\begin{frame}{The Demand for Labor}

Labor markets, like other markets in the economy, are governed by the forces of supply and demand. 
The same basic tools used to analyze goods markets can also be applied to labor markets.

\begin{tcolorbox}[colframe=red!80!black, colback=white, title=\textbf{Key Idea: Labor as a Derived Demand}]
Labor demand is a \textbf{derived demand}. Firms demand labor not for its own sake, but because labor is an input used to produce goods and services that consumers want.
\end{tcolorbox}

\end{frame}
%==========================================================
\begin{frame}{The Demand for Labor}

To understand the demand for labor, we must focus on firms:
\begin{itemize}
    \item Firms hire workers to produce goods for sale.
    \item The demand for labor depends on the demand for the goods being produced.
    \item Changes in product markets directly affect wages and employment.
\end{itemize}

\begin{tcolorbox}[colframe=blue!70!black, colback=white, title=\textbf{Goods Market vs. Labor Market}]
\begin{itemize}
    \item In the goods market, supply and demand determine the \textbf{price of goods}.
    \item In the labor market, supply and demand determine the \textbf{wage of workers}.
    \item The wage is the price of labor.
\end{itemize}
\end{tcolorbox}

\end{frame}

%==========================================================
\begin{frame}{The Demand for Labor}
    \begin{center}
        \includegraphics[width=\textwidth]{pictures/chap18/T1.png}
    \end{center}
\end{frame}
%==========================================================
\begin{frame}{The Competitive Profit-Maximizing Firm}

To understand labor demand, we examine how a typical firm decides how much labor to hire.
Consider an apple producer that must choose how many apple pickers to employ.

\begin{tcolorbox}[colframe=red!80!black, colback=white, title=\textbf{Key Assumption 1: Competitive Markets}]
The firm is \textbf{competitive} in both:
\begin{itemize}
    \item The output market (it sells apples)
    \item The labor market (it hires apple pickers)
\end{itemize}
As a result, the firm is a \textbf{price taker}: it takes the price of apples and the wage as given.
\end{tcolorbox}

Because many firms sell apples and hire workers, a single firm cannot influence market prices.
It must decide only how much output to produce and how many workers to hire.

\end{frame}
%==========================================================
\begin{frame}{The Competitive Profit-Maximizing Firm}

\begin{tcolorbox}[colframe=blue!70!black, colback=white, title=\textbf{Key Assumption 2: Profit Maximization}]
The firm’s goal is to \textbf{maximize profit}:
\[
\text{Profit} = \text{Total Revenue} - \text{Total Cost}
\]
The firm does not care directly about output or employment, only about profit.
\end{tcolorbox}

The firm’s supply of goods and its demand for labor are both derived from this profit-maximizing objective.

\end{frame}

%==========================================================
\begin{frame}{The Production Function}

To decide how many workers to hire, a firm must understand how labor affects output.
This relationship is described by the \textbf{production function}.

\begin{tcolorbox}[colframe=red!80!black, colback=white, title=\textbf{Production Function}]
A production function shows the relationship between:
\begin{itemize}
    \item \textbf{Inputs} (labor)
    \item \textbf{Output} (apples)
\end{itemize}
All other inputs (land, capital, technology) are held constant.
\end{tcolorbox}

\end{frame}

%==========================================================
\begin{frame}{The Marginal Product of Labor}


\begin{tcolorbox}[colframe=blue!70!black, colback=white, title=\textbf{Marginal Product of Labor (MPL)}]
\[
\text{MPL} = \frac{\Delta Q}{\Delta L}
\]
MPL measures the additional output produced by hiring one more worker.
\end{tcolorbox}

As more workers are hired, the MPL eventually falls due to diminishing marginal returns.

\end{frame}

%==========================================================
\begin{frame}{From Marginal Product to Hiring Decisions}

The firm sells output in a competitive market, so the price of apples is constant.

\begin{tcolorbox}[colframe=red!80!black, colback=white, title=\textbf{Value of the Marginal Product of Labor}]
\[
\text{VMPL} = P \times \text{MPL}
\]
VMPL measures the additional revenue generated by hiring one more worker.
\end{tcolorbox}

\end{frame}

%==========================================================
\begin{frame}{From Marginal Product to Hiring Decisions}

\begin{tcolorbox}[colframe=blue!70!black, colback=white, title=\textbf{Profit-Maximizing Rule}]
\begin{itemize}
    \item Hire another worker if $\text{VMPL} > W$
    \item Stop hiring when $\text{VMPL} = W$
    \item Do not hire if $\text{VMPL} < W$
\end{itemize}
\end{tcolorbox}

\textbf{Key insight:}  
The firm’s demand for labor is determined by comparing the value of what a worker produces to the wage.

\end{frame}
%==========================================================
\begin{frame}{The Hiring Decisions}
    \begin{center}
        \includegraphics[width=\textwidth]{pictures/chap18/T2.png}
    \end{center}
\end{frame}
%==========================================================
\begin{frame}{The Production Function and Diminishing Marginal Product}

Figure graphs the relationship between labor input and output.
The number of workers is on the horizontal axis, and total output is on the vertical axis.
\begin{center}
    \includegraphics[width=0.7\textwidth]{pictures/chap18/T3.png}
\end{center}

\end{frame}
%==========================================================
\begin{frame}{The Production Function and Diminishing Marginal Product}

\begin{tcolorbox}[colframe=blue!70!black, colback=white, title=\textbf{Marginal Product of Labor (MPL)}]
The marginal product of labor is the increase in output resulting from the hiring of one additional worker.
\end{tcolorbox}

As the number of workers increases, the marginal product of labor declines.
This pattern reflects \textbf{diminishing marginal product}:
early workers can perform the most productive tasks, while additional workers contribute less to total output.
As a result, the production function becomes flatter as labor increases.

\end{frame}
%==========================================================
\begin{frame}{The Value of the Marginal Product}

A profit-maximizing firm is concerned not with output itself, but with the \textbf{revenue} generated by production.
To decide how many workers to hire, the firm compares the benefit of an additional worker with the wage.

\begin{tcolorbox}[colframe=red!80!black, colback=white, title=\textbf{Value of the Marginal Product of Labor}]
The value of the marginal product of labor (VMPL) equals:
\[
\text{VMPL} = P \times \text{MPL}
\]
where $P$ is the market price of output.
\end{tcolorbox}

\end{frame}
%==========================================================
\begin{frame}{The Value of the Marginal Product}

VMPL measures the additional \textbf{revenue} generated by hiring one more worker.
Because output price is constant for a competitive firm and MPL diminishes as labor increases, VMPL also diminishes as more workers are hired.

\begin{tcolorbox}[colframe=blue!70!black, colback=white, title=\textbf{Marginal Revenue Product}]
Economists sometimes refer to VMPL as the \textbf{marginal revenue product} of labor:
the extra revenue earned from employing one additional unit of labor.
\end{tcolorbox}

\end{frame}

%==========================================================
\begin{frame}{VMPL and the Demand for Labor}

Suppose the firm faces a constant market wage.
Hiring decisions depend on a comparison between the value of what a worker produces and the wage.

\begin{tcolorbox}[colframe=red!80!black, colback=white, title=\textbf{Profit-Maximizing Hiring Rule}]
\begin{itemize}
    \item Hire labor if $\text{VMPL} > W$
    \item Stop hiring when $\text{VMPL} = W$
    \item Do not hire if $\text{VMPL} < W$
\end{itemize}
\end{tcolorbox}
\end{frame}

%==========================================================
\begin{frame}{VMPL and the Demand for Labor}

\begin{center}
    \includegraphics[width=0.8\textwidth]{pictures/chap18/T4.png}
\end{center}
\end{frame}

%==========================================================
\begin{frame}{VMPL and the Demand for Labor}

Graphically, the firm hires workers up to the point where the VMPL curve intersects the market wage.
Below this point, hiring increases profit; above it, hiring reduces profit.

\begin{tcolorbox}[colframe=blue!70!black, colback=white, title=\textbf{Labor Demand Curve}]
For a competitive, profit-maximizing firm, the \textbf{value-of-marginal-product curve is the labor-demand curve}.
\end{tcolorbox}

\end{frame}

%==========================================================
\begin{frame}{What Causes the Labor-Demand Curve to Shift?}

We have learned that a firm’s labor-demand curve reflects the
\textbf{value of the marginal product of labor (VMPL)}.

\begin{tcolorbox}[colframe=red!80!black, colback=white, title=\textbf{Key Insight}]
Any factor that changes the \textbf{value of the marginal product of labor}
will shift the labor-demand curve.
\end{tcolorbox}

Recall:
\[
\text{VMPL} = P \times \text{MPL}
\]

Therefore, changes in either:
\begin{itemize}
    \item The price of output ($P$), or
    \item The marginal product of labor (MPL)
\end{itemize}
will cause the labor-demand curve to shift.

\end{frame}

%==========================================================
\begin{frame}{Determinants of Labor Demand}

\begin{tcolorbox}[colframe=blue!70!black, colback=white, title=\textbf{1. Output Price}]
An increase in the price of output raises VMPL and shifts labor demand to the right.
A decrease in output price lowers VMPL and shifts labor demand to the left.
\end{tcolorbox}

\begin{tcolorbox}[colframe=blue!70!black, colback=white, title=\textbf{2. Technological Change}]
Technological progress usually increases MPL and labor demand
(\textbf{labor-augmenting technology}).
However, \textbf{labor-saving technology} can reduce MPL and shift labor demand left.
\end{tcolorbox}

\begin{tcolorbox}[colframe=blue!70!black, colback=white, title=\textbf{3. Supply of Other Factors}]
A change in the quantity of other inputs (such as capital or land)
affects MPL and therefore labor demand.
\end{tcolorbox}

\end{frame}
%==========================================================
\begin{frame}{Input Demand and Output Supply}

A competitive, profit-maximizing firm makes two key decisions:
how much output to produce and how much labor to hire.
These two decisions are closely linked.

\begin{tcolorbox}[colframe=red!80!black, colback=white, title=\textbf{Profit-Maximization Conditions}]
\begin{itemize}
    \item \textbf{Output decision:} Produce where $P = MC$
    \item \textbf{Labor decision:} Hire labor where $P \times MPL = W$
\end{itemize}
\end{tcolorbox}

From the labor-hiring rule:
\[
P \times MPL = W
\]

Dividing both sides by $MPL$, we obtain:
\[
P = \frac{W}{MPL}
\]

\end{frame}
%==========================================================
\begin{frame}{Input Demand and Output Supply}

\begin{tcolorbox}[colframe=blue!70!black, colback=white, title=\textbf{Key Link Between Labor and Output}]
The marginal cost of output is:
\[
MC = \frac{W}{MPL}
\]
Thus, the labor-hiring condition implies:
\[
P = MC
\]
\end{tcolorbox}

\textbf{Conclusion:}  
A firm that hires labor until the value of the marginal product equals the wage
will also produce output up to the point where price equals marginal cost.

\end{frame}
%==========================================================
\section{18-2 The Supply of Labor}
%==========================================================
\begin{frame}{The Supply of Labor}

Having analyzed labor demand, we now turn to the other side of the labor market:
\textbf{labor supply}.

\begin{tcolorbox}[colframe=red!80!black, colback=white, title=\textbf{Focus of Labor Supply}]
Labor supply reflects the decisions of individuals about:
\begin{itemize}
    \item How many hours to work
    \item How much leisure to enjoy
\end{itemize}
given the wage they can earn.
\end{tcolorbox}

In this section, we discuss the labor-supply decision informally.
A more formal model of household decision making is developed later.

\end{frame}

%==========================================================
\begin{frame}{The Trade-off between Work and Leisure}

For individuals, one of the most important trade-offs is between \textbf{work and leisure}.

\begin{tcolorbox}[colframe=red!80!black, colback=white, title=\textbf{Opportunity Cost of Leisure}]
The cost of an hour of leisure is the wage that could have been earned by working.
An increase in the wage raises the opportunity cost of leisure.
\end{tcolorbox}

An upward-sloping labor-supply curve reflects this trade-off:
as the wage increases, workers choose to supply more labor and enjoy less leisure.
\end{frame}

%==========================================================
\begin{frame}{What Causes the Labor-Supply Curve to Shift?}

The labor-supply curve shifts whenever people change the amount of labor
they are willing to supply at a given wage.

\begin{tcolorbox}[colframe=red!80!black, colback=white, title=\textbf{Key Insight}]
Any factor that affects individuals' preferences, opportunities, or ability to work
can shift the labor-supply curve.
\end{tcolorbox}

It is important to distinguish:
\begin{itemize}
    \item \textbf{Movements along the curve}: caused by a change in the wage
    \item \textbf{Shifts of the curve}: caused by factors other than the wage
\end{itemize}

\end{frame}

%==========================================================
\begin{frame}{Determinants of Labor Supply}

\begin{tcolorbox}[colframe=blue!70!black, colback=white, title=\textbf{1. Changes in Tastes}]
A change in attitudes toward work versus leisure
can increase or decrease the supply of labor.
\end{tcolorbox}

\begin{tcolorbox}[colframe=blue!70!black, colback=white, title=\textbf{2. Changes in Alternative Opportunities}]
If wages rise in another labor market,
workers may switch jobs, reducing labor supply in the original market.
\end{tcolorbox}

\begin{tcolorbox}[colframe=blue!70!black, colback=white, title=\textbf{3. Immigration}]
An inflow of workers into a country or region
increases the supply of labor in that labor market.
\end{tcolorbox}

\end{frame}

%==========================================================
\section{18-3 Equilibrium in the Labor Market}
%==========================================================
\begin{frame}{Equilibrium in the Labor Market}
    \begin{center}
        \includegraphics[width=0.8\textwidth]{pictures/chap18/T5.png}
    \end{center}
\end{frame}
%==========================================================
\begin{frame}{Equilibrium in the Labor Market}

\begin{tcolorbox}[colframe=blue!70!black, colback=white, title=\textbf{Labor Market Equilibrium}]
In equilibrium:
\begin{itemize}
    \item Firms hire labor until $\text{VMPL} = W$.
    \item Workers supply labor based on the wage.
    \item The wage that clears the market must equal the value of the marginal product of labor.
\end{itemize}
\end{tcolorbox}

\textbf{Note:}  
Any event that shifts labor supply or labor demand
must change both the equilibrium wage and the value of the marginal product of labor by the same amount.

\end{frame}

%==========================================================
\begin{frame}{Shifts in Labor Supply}

Suppose immigration increases the number of workers willing to work.
The labor-supply curve shifts to the right from $S_1$ to $S_2$.
\begin{center}
    \includegraphics[width=0.8\textwidth]{pictures/chap18/T6.png}
\end{center}


\end{frame}
%==========================================================
\begin{frame}{Shifts in Labor Supply}
\begin{tcolorbox}[colframe=red!80!black, colback=white, title=\textbf{Immediate Effect}]
At the initial wage $W_1$, the quantity of labor supplied exceeds the quantity demanded,
creating a surplus of labor.
\end{tcolorbox}

This surplus puts downward pressure on the wage.
As the wage falls from $W_1$ to $W_2$, firms find it profitable to hire more workers.

\begin{tcolorbox}[colframe=blue!70!black, colback=white, title=\textbf{New Equilibrium}]
\begin{itemize}
    \item Wage falls
    \item Employment rises
    \item The value of the marginal product of labor falls
\end{itemize}
\end{tcolorbox}

In equilibrium, the wage and the value of the marginal product of labor
both adjust to the same lower level.

\end{frame}

%==========================================================
\begin{frame}{Shifts in Labor Demand}

Suppose an increase in the popularity of apples raises the price of apples.
This price increase does not change the marginal product of labor,
but it raises the \textbf{value of the marginal product of labor}.

\begin{tcolorbox}[colframe=red!80!black, colback=white, title=\textbf{Key Mechanism}]
An increase in the output price increases VMPL at every level of employment,
shifting the labor-demand curve to the right.
\end{tcolorbox}

\end{frame}

%==========================================================
\begin{frame}{Shifts in Labor Demand}

\begin{center}
    \includegraphics[width=0.8\textwidth]{pictures/chap18/T7.png}
\end{center}

\end{frame}

%==========================================================
\begin{frame}{Shifts in Labor Demand}

At the initial wage $W_1$, firms now find it profitable to hire more workers.
As labor demand shifts from $D_1$ to $D_2$, the equilibrium wage rises from $W_1$ to $W_2$,
and employment increases from $L_1$ to $L_2$.

\begin{tcolorbox}[colframe=blue!70!black, colback=white, title=\textbf{New Equilibrium}]
\begin{itemize}
    \item Wage rises
    \item Employment rises
    \item The value of the marginal product of labor rises
\end{itemize}
\end{tcolorbox}

\end{frame}

%==========================================================
\begin{frame}{Labor Demand and Industry Prosperity}

Prosperity for firms in an industry is often linked to prosperity for workers.

\begin{tcolorbox}[colframe=red!80!black, colback=white, title=\textbf{Output Price and Wages}]
When the price of output rises:
\begin{itemize}
    \item Firms earn higher profits
    \item Labor demand increases
    \item Workers earn higher wages
\end{itemize}
\end{tcolorbox}

Conversely, when the price of output falls,
labor demand decreases and wages decline.

\textbf{Example:}  
Workers in industries with volatile output prices, such as oil,
often experience wages that fluctuate with world market prices.

\end{frame}

%==========================================================
\section{18-4 The Other Factors of Production: Land and Capital}
%==========================================================
\begin{frame}{The Other Factors of Production: Land and Capital}

So far, we have focused on labor as a factor of production.
However, firms must also decide how much \textbf{land} and \textbf{capital} to use.

\begin{tcolorbox}[colframe=red!80!black, colback=white, title=\textbf{Factors of Production}]
Economists classify inputs into three categories:
\begin{itemize}
    \item \textbf{Labor}: workers’ time and effort
    \item \textbf{Land}: natural resources
    \item \textbf{Capital}: equipment and structures used in production
\end{itemize}
\end{tcolorbox}

\end{frame}

%==========================================================
\begin{frame}{The Other Factors of Production: Land and Capital}
\begin{tcolorbox}[colframe=blue!70!black, colback=white, title=\textbf{What Is Capital?}]
Capital is the stock of equipment and structures
that are used to produce goods and services.
It represents goods produced in the past and used in the present.
\end{tcolorbox}

For an apple-producing firm, capital includes ladders, trucks, storage buildings,
and even the trees themselves.

\textbf{Key idea:}  
Firms choose the quantities of all inputs—labor, land, and capital—to maximize profit.

\end{frame}

%==========================================================
\begin{frame}{Equilibrium in the Markets for Land and Capital}

To understand how much owners of land and capital earn,
we must distinguish between two different prices.

\begin{tcolorbox}[colframe=red!80!black, colback=white, title=\textbf{Two Prices of a Factor of Production}]
\begin{itemize}
    \item \textbf{Purchase price}: the price paid to own a factor permanently
    \item \textbf{Rental price}: the price paid to use a factor for a limited time
\end{itemize}
\end{tcolorbox}

The wage is simply the rental price of labor.
Similarly, land and capital also earn rental prices determined by supply and demand.

\end{frame}

%==========================================================
\begin{frame}{Equilibrium in the Markets for Land and Capital}

\begin{tcolorbox}[colframe=blue!70!black, colback=white, title=\textbf{Factor Demand}]
For labor, land, and capital, a competitive firm hires each input
up to the point where:
\[
\text{Value of marginal product} = \text{Factor price}
\]
\end{tcolorbox}

Thus, the demand curve for land or capital reflects the marginal productivity of that factor.
\end{frame}

%==========================================================
\begin{frame}{Equilibrium in the Markets for Land and Capital}

\begin{center}
    \includegraphics[width=\textwidth]{pictures/chap18/T8.png}
\end{center}

\end{frame}

%==========================================================
\begin{frame}{Factor Income and Asset Prices}

As long as firms are competitive and profit-maximizing,
each factor of production earns the value of its marginal contribution.

\begin{tcolorbox}[colframe=red!80!black, colback=white, title=\textbf{Distribution of Income}]
\begin{itemize}
    \item Workers earn wages equal to VMPL
    \item Landowners earn rent equal to the value of land's marginal product
    \item Owners of capital earn rental income equal to the value of capital's marginal product
\end{itemize}
\end{tcolorbox}

\end{frame}

%==========================================================
\begin{frame}{Factor Income and Asset Prices}

The purchase price of land or capital depends on the income it can generate.

\begin{tcolorbox}[colframe=blue!70!black, colback=white, title=\textbf{Link Between Rental and Purchase Prices}]
A buyer is willing to pay more for an asset
if it provides a valuable stream of future rental income.
\end{tcolorbox}

\textbf{Key insight:}  
The equilibrium purchase price reflects both the current and expected future
value of the marginal product.

\end{frame}

%==========================================================
\begin{frame}{What Is Capital Income?}

Labor income is straightforward: it is the wage workers earn from working.
Capital income is less obvious.

\begin{tcolorbox}[colframe=red!80!black, colback=white, title=\textbf{Capital Income}]
Capital income is the income earned from owning capital,
the stock of equipment and structures used in production.
It reflects the rent paid for the use of capital.
\end{tcolorbox}

In practice, capital income reaches households in several forms:
\begin{itemize}
    \item \textbf{Interest} paid to bondholders and bank depositors
    \item \textbf{Dividends} paid to shareholders
    \item \textbf{Retained earnings}, which increase the value of firms and their stock
\end{itemize}

Regardless of how capital income is paid,
capital is ultimately compensated according to the
\textbf{value of its marginal product}.

\end{frame}

%==========================================================
\begin{frame}{Linkages among the Factors of Production}

We have seen that the price paid to any factor of production
equals the \textbf{value of its marginal product}.

\begin{tcolorbox}[colframe=red!80!black, colback=white, title=\textbf{Marginal Productivity and Factor Prices}]
\begin{itemize}
    \item Because of diminishing marginal product, an abundant factor has a low marginal product and a low price.
    \item A scarce factor has a high marginal product and a high price.
\end{itemize}
\end{tcolorbox}

However, factors of production are typically used together.
As a result, the marginal product of one factor depends on the quantities of other factors available.

\begin{tcolorbox}[colframe=blue!70!black, colback=white, title=\textbf{Key Insight}]
An event that changes the supply of one factor
will often affect the earnings of \textbf{all} factors of production.
\end{tcolorbox}

\end{frame}

%==========================================================
\begin{frame}{An Example: Changes in One Factor Affect Others}

Suppose a hurricane destroys many of the ladders used to pick apples.

\begin{tcolorbox}[colframe=red!80!black, colback=white, title=\textbf{Effect on the Ladder Market}]
With fewer ladders available:
\begin{itemize}
    \item The supply of ladders falls
    \item The rental price of ladders rises
\end{itemize}
\end{tcolorbox}
\end{frame}

%==========================================================
\begin{frame}{An Example: Changes in One Factor Affect Others}

With fewer ladders to work with, apple pickers become less productive.

\begin{tcolorbox}[colframe=blue!70!black, colback=white, title=\textbf{Effect on Labor}]
\begin{itemize}
    \item The marginal product of labor falls
    \item Labor demand decreases
    \item Wages fall
\end{itemize}
\end{tcolorbox}

\textbf{lesson:}  
Changes in the supply of one factor affect the value of the marginal product—and earnings—of other factors.

\end{frame}

%==========================================================
\begin{frame}{The Economics of the Black Death}

In 14th-century Europe, the Black Death wiped out about one-third of the population.
This tragic event provides a natural experiment for testing the theory of factor markets.

\begin{tcolorbox}[colframe=red!80!black, colback=white, title=\textbf{Shock to Factor Supplies}]
\begin{itemize}
    \item Labor supply fell sharply
    \item Land supply remained unchanged
\end{itemize}
\end{tcolorbox}

With fewer workers available:
\begin{itemize}
    \item The marginal product of labor increased
    \item Wages rose
\end{itemize}

Because land and labor are used together in production:
\begin{itemize}
    \item The marginal product of land fell
    \item Rents declined
\end{itemize}

\end{frame}

%==========================================================
\begin{frame}{The Economics of the Black Death}

\begin{tcolorbox}[colframe=blue!70!black, colback=white, title=\textbf{Historical Evidence}]
During this period, wages approximately doubled,
while rents fell by 50 percent or more.
\end{tcolorbox}

\textbf{Lesson:}  
A change in the supply of one factor of production
can raise the earnings of some factors while lowering the earnings of others.

\end{frame}

%==========================================================
%==========================================================
%==========================================================


\end{document}