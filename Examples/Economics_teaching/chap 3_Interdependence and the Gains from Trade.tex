\documentclass{beamer}
\usepackage{graphicx} % Required for inserting images
\usepackage{amsmath}
\usepackage[most]{tcolorbox}
\usepackage{lmodern}
\usepackage{mathabx}

\usetheme{Madrid} % 可選其他主題:e.g., Warsaw, Berkeley, etc.
\usecolortheme{default}
\setbeamertemplate{caption}[numbered]% Number float-like environments
% Customize the caption
\setbeamerfont{caption}{size=\footnotesize}
% \setbeamercolor{caption}{fg=blue}
% \setbeamercolor{caption name}{fg=red}

\title{Mankiw's Principles of Economics}
\subtitle{chap 3: Interdependence and the Gains from Trade}
\author{Hsu Chun-Wei}
\date{July 2025}

\begin{document}

\maketitle

%======================================================================
\begin{frame}{A Parable for the Modern Economy}

Consider a simple economy with two people—Rose (a cattle rancher) and Frank (a potato farmer)—and two goods: meat and potatoes.

\begin{tcolorbox}[colframe=red!80!black, colback=white, title=\textbf{Key Setting}]
\begin{itemize}
    \item Two goods: meat and potatoes.
    \item Two individuals: Rose (rancher) and Frank (farmer).
    \item Both desire to consume both goods.
\end{itemize}
\end{tcolorbox}

The gains from trade are clearest when Rose can produce only meat and Frank can produce only potatoes. Without trade, both would face limited variety; with trade, each can enjoy meals combining meat and potatoes.

\begin{tcolorbox}[colframe=blue!70!black, colback=white, title=\textbf{Why Trade Improves Well-Being}]
Trade allows both individuals to enjoy more variety and higher total consumption than if they were self-sufficient.
\end{tcolorbox}

\end{frame}

%======================================================================
\begin{frame}{Gains from Specialization}

Even if both Rose and Frank can produce both goods, their costs differ.  
Suppose Rose can grow potatoes but her land is poorly suited for it, and Frank can raise cattle but is not very efficient. Both still benefit from specializing.

\begin{tcolorbox}[colframe=blue!70!black, colback=white, title=\textbf{Why Specialization Helps}]
\begin{itemize}
    \item Individuals face different opportunity costs.
    \item Producing the good with lower cost increases total output.
    \item Trade enables consumption outside each person’s production possibilities.
\end{itemize}
\end{tcolorbox}

A key question arises: What if one person is better at producing \textit{every} good?  
Would Rose still gain from trading with Frank if she is more productive in both meat and potatoes?  
To answer this, we must examine what determines production decisions.

\end{frame}

%======================================================================
\begin{frame}{Production Possibilities}

Suppose Frank and Rose each work 8 hours per day and can allocate this time to growing potatoes or raising cattle.

\begin{tcolorbox}[colframe=red!80!black, colback=white, title=\textbf{Time Required per Ounce}]
\begin{itemize}
    \item Frank: 60 min/oz (meat), 15 min/oz (potatoes)
    \item Rose: 20 min/oz (meat), 10 min/oz (potatoes)
\end{itemize}
\end{tcolorbox}

From this, if each devotes all 8 hours to one good:
\begin{itemize}
    \item Frank: 8 oz meat or 32 oz potatoes
    \item Rose: 24 oz meat or 48 oz potatoes
\end{itemize}

\begin{center}
\includegraphics[width=0.7\textwidth]{pictures/chap3/T1.png}
\end{center}

\end{frame}

%======================================================================
\begin{frame}{Frank's Production Possibilities Frontier}

\textbf{Frank's PPF Key Points}
\begin{itemize}
    \item All time on potatoes: 32 oz potatoes, 0 meat.
    \item All time on meat: 8 oz meat, 0 potatoes.
    \item Split time (4 hours each): 16 oz potatoes, 4 oz meat.
\end{itemize}

This frontier illustrates the trade-off Frank faces: producing more potatoes requires giving up meat and vice versa.

\begin{center}
\includegraphics[width=0.5\textwidth]{pictures/chap3/HW2.png}
\end{center}

\end{frame}

%======================================================================
\begin{frame}{Rose's Production Possibilities Frontier}

\textbf{Rose's PPF Key Points}
\begin{itemize}
    \item All time on potatoes: 48 oz potatoes, 0 meat.
    \item All time on meat: 24 oz meat, 0 potatoes.
    \item Intermediate combinations lie along a straight line between these endpoints.
\end{itemize}

Rose, being more productive than Frank in both goods, still faces trade-offs when choosing between producing meat and potatoes.

\begin{center}
\includegraphics[width=0.5\textwidth]{pictures/chap3/HW3.png}
\end{center}

\end{frame}

%======================================================================
\begin{frame}{Linear Production Possibilities Frontier}
Here, We suppose that Frank’s technology allows him to switch between meat and potatoes at a constant rate.

\begin{tcolorbox}[colframe=blue!70!black, colback=white, title=\textbf{Constant Opportunity Cost}]
\begin{itemize}
    \item Reducing meat production by 1 ounce frees 1 hour.
    \item That hour increases potato production by 4 ounces.
    \item This trade-off is constant at all production levels.
\end{itemize}
\end{tcolorbox}

As a result, Frank’s production possibilities frontier is a straight line.

Similarly, Rose’s technology gives her constant trade-offs:
\begin{itemize}
    \item All potatoes: 48 oz; all meat: 24 oz.
    \item Equal split (4 hours each): 24 oz potatoes and 12 oz meat.
\end{itemize}

\end{frame}

%======================================================================
\begin{frame}{Production vs. Consumption Without Trade}

If Frank and Rose remain self-sufficient, their production possibilities frontier is also their consumption possibilities frontier. Without trade, they can consume only what they produce.

\begin{tcolorbox}[colframe=red!80!black, colback=white, title=\textbf{Self-Sufficiency}]
\begin{itemize}
    \item No trade means consumption = production.
    \item Each person must decide the mix of goods based on preferences and opportunity costs.
\end{itemize}
\end{tcolorbox}

Figure 1 identifies the combinations chosen by Frank and Rose when they do not trade:
\begin{itemize}
    \item \textbf{Frank (Point A):} 16 oz potatoes and 4 oz meat.
    \item \textbf{Rose (Point B):} 24 oz potatoes and 12 oz meat.
\end{itemize}

\end{frame}

%======================================================================
\begin{frame}{How Trade Expands Consumption Opportunities}

For Frank, trade moves his consumption from point A to point A*.

\begin{columns}

\column{0.5\textwidth}
\begin{tcolorbox}[colframe=blue!70!black, colback=white, title=\textbf{Frank's Gains from Trade}]
\begin{itemize}
    \item Without trade: produces and consumes 4 oz meat and 16 oz potatoes.
    \item With trade: specializes in potatoes, producing 32 oz.
    \item Trades 15 oz potatoes for 5 oz meat from Rose.
    \item Consumption after trade: 5 oz meat and 17 oz potatoes.
\end{itemize}
\end{tcolorbox}

\column{0.5\textwidth}
\begin{center}
\includegraphics[width=\textwidth]{pictures/chap3/T4.png}
\end{center}

\end{columns}

\end{frame}

%======================================================================
\begin{frame}{Rose's Gains from Trade and Summary}

Trade also allows Rose to move from point B to point B*, consuming more of both goods through specialization and exchange.

\begin{columns}

\column{0.5\textwidth}
\begin{tcolorbox}[colframe=blue!70!black, colback=white, title=\textbf{Rose's Gains from Trade}]
\begin{itemize}
    \item Without trade: produces and consumes 12 oz meat and 24 oz potatoes.
    \item With trade: specializes in meat, producing 18 oz.
    \item Trades 5 oz meat for 15 oz potatoes from Frank.
    \item Consumption after trade: 13 oz meat and 27 oz potatoes.
\end{itemize}
\end{tcolorbox}

\column{0.5\textwidth}
\begin{center}
\includegraphics[width=\textwidth]{pictures/chap3/T5.png}
\end{center}

\end{columns}

\end{frame}

%======================================================================
\begin{frame}{Gains from Trade: Summary}

\begin{tcolorbox}[colframe=red!80!black, colback=white, title=\textbf{Increase in Consumption}]
\begin{itemize}
    \item Frank: +1 oz meat, +1 oz potatoes
    \item Rose: +1 oz meat, +3 oz potatoes
\end{itemize}
\end{tcolorbox}

\begin{center}
\includegraphics[width=0.8\textwidth]{pictures/chap3/T6.png}
\end{center}

\end{frame}

%======================================================================
\begin{frame}{Comparative Advantage: The Driving Force of Specialization}

If Rose is better at raising cattle \emph{and} growing potatoes, how can Frank specialize in anything at all?  
To answer this puzzle, we turn to the principle of \textbf{comparative advantage}.

\begin{tcolorbox}[colframe=red!80!black, colback=white, title=\textbf{The Puzzle}]
Even if one person is better at producing every good, specialization and trade can still benefit both sides.
\end{tcolorbox}

A natural first step is to ask:  
\textbf{Who can produce potatoes at a lower cost—Frank or Rose?}  
This question points to the idea of opportunity cost, which lies at the heart of comparative advantage and explains the gains from trade.

\end{frame}

%======================================================================
\begin{frame}{Absolute Advantage}

\begin{tcolorbox}[colframe=red!80!black, colback=white, title=\textbf{Definition}]
A producer has an \textbf{absolute advantage} if they can produce a good using fewer inputs (such as time) than another producer.
\end{tcolorbox}

In our example, time is the only input. Rose requires less time than Frank to produce both goods:
\begin{itemize}
    \item Meat: Rose needs 20 minutes per ounce; Frank needs 60 minutes.
    \item Potatoes: Rose needs 10 minutes per ounce; Frank needs 15 minutes.
\end{itemize}

\begin{tcolorbox}[colframe=blue!70!black, colback=white, title=\textbf{Conclusion}]
Rose has an absolute advantage in producing both meat and potatoes because she uses fewer minutes per unit of output.
\end{tcolorbox}

\end{frame}

%======================================================================
\begin{frame}{Opportunity Cost and Comparative Advantage}

Another way to compare production costs is to examine \textbf{opportunity cost}, the value of the next-best alternative given up.  
Since Frank and Rose each work 8 hours per day, time spent producing one good reduces time available for the other.

\begin{tcolorbox}[colframe=red!80!black, colback=white, title=\textbf{Opportunity Cost}]
The opportunity cost of a good is what must be given up to obtain it.
\end{tcolorbox}

\begin{center}
\includegraphics[width=0.75\textwidth]{pictures/chap3/T7.png}
\end{center}

\end{frame}

%======================================================================
\begin{frame}{Opportunity Cost: Frank vs. Rose}

\textbf{Rose’s opportunity cost:}
\begin{itemize}
    \item Producing 1 oz of potatoes takes 10 minutes.
    \item Those 10 minutes could produce $\tfrac{1}{2}$ oz of meat (since meat takes 20 min/oz).
    \item Therefore: \textbf{OC\textsubscript{Rose}(1 oz potatoes) = 1/2 oz meat}.
\end{itemize}

\textbf{Frank’s opportunity cost:}
\begin{itemize}
    \item Producing 1 oz of potatoes takes 15 minutes.
    \item In 15 minutes, Frank could produce $\tfrac{1}{4}$ oz of meat (since meat takes 60 min/oz).
    \item Therefore: \textbf{OC\textsubscript{Frank}(1 oz potatoes) = 1/4 oz meat}.
\end{itemize}

\begin{tcolorbox}[colframe=blue!70!black, colback=white, title=\textbf{Key Insight}]
The opportunity cost of meat is the inverse of the opportunity cost of potatoes.
\end{tcolorbox}


\end{frame}

%======================================================================
\begin{frame}{Comparative Advantage}

\begin{tcolorbox}[colframe=red!80!black, colback=white, title=\textbf{Definition}]
Comparative advantage: the ability to produce a good at a lower opportunity cost than another producer.
\end{tcolorbox}

Applying this to our example:
\begin{itemize}
    \item \textbf{Frank’s opportunity cost of potatoes} = $\tfrac{1}{4}$ oz meat  
          (lower than Rose’s $\tfrac{1}{2}$ oz meat)  
          → Frank has a comparative advantage in \textbf{potatoes}.
    \item \textbf{Rose’s opportunity cost of meat} = 2 oz potatoes  
          (lower than Frank’s 4 oz potatoes)  
          → Rose has a comparative advantage in \textbf{meat}.
\end{itemize}

\begin{tcolorbox}[colframe=blue!70!black, colback=white, title=\textbf{Key Insight}]
Even if one person has an absolute advantage in both goods (as Rose does),  
it is impossible to have a comparative advantage in both goods.
\end{tcolorbox}

Comparative advantage reflects \textbf{relative} opportunity costs:  
If one good has a high opportunity cost for a person, the other good must have a low opportunity cost for that same person.

\end{frame}

%======================================================================
\begin{frame}{Comparative Advantage and Trade}

\begin{tcolorbox}[colframe=red!80!black, colback=white, title=\textbf{Key Principle}]
Specialization based on \textbf{comparative advantage} expands total production and can make everyone better off.
\end{tcolorbox}

In our example:
\begin{itemize}
    \item Frank specializes more in potatoes.
    \item Rose specializes more in meat.
\end{itemize}

As a result:
\begin{itemize}
    \item Potato output increases from 40 to 44 ounces.
    \item Meat output increases from 16 to 18 ounces.
\end{itemize}

Both Frank and Rose share the benefits of this increased production through trade.

\end{frame}

%======================================================================
\begin{frame}{Why Trade Makes Both Sides Better Off}

Because Frank and Rose have different opportunity costs, each can obtain a good through trade at a price lower than their own opportunity cost.

\begin{tcolorbox}[colframe=blue!70!black, colback=white, title=\textbf{Frank’s Viewpoint}]
Frank receives 5 oz of meat in exchange for 15 oz of potatoes.  
This means Frank buys meat at a price of 3 oz potatoes per ounce of meat.  
Since his opportunity cost is 4 oz potatoes per ounce of meat, Frank benefits from the trade.
\end{tcolorbox}

\begin{tcolorbox}[colframe=blue!70!black, colback=white, title=\textbf{Rose’s Viewpoint}]
Rose receives 15 oz of potatoes for 5 oz of meat.  
This means Rose buys potatoes at a price of $\tfrac{1}{3}$ oz meat per ounce of potatoes.  
Since her opportunity cost is $\tfrac{1}{2}$ oz meat per ounce of potatoes, Rose also benefits.
\end{tcolorbox}

\end{frame}

%======================================================================
\begin{frame}{The Price of the Trade}

The principle of comparative advantage implies gains from specialization and trade.  
But what determines the price at which trade occurs?

\begin{tcolorbox}[colframe=red!80!black, colback=white, 
title=\textbf{General Rule}]
For both parties to gain from trade, the trading price must lie \textbf{between the two opportunity costs}.
\end{tcolorbox}

In our example:
\begin{itemize}
    \item Rose’s opportunity cost of meat = 2 oz potatoes per oz meat.
    \item Frank’s opportunity cost of meat = 4 oz potatoes per oz meat.
    \item Any mutually beneficial price must lie between \textbf{2 and 4}.
\end{itemize}

Frank and Rose agree to trade at \textbf{3 oz potatoes per 1 oz meat}, which lies within this range.

\end{frame}

%======================================================================
\begin{frame}{Reason behind the price of trade}
    \begin{tcolorbox}[colframe=blue!70!black, colback=white, 
    title=\textbf{Why the Price Must Be in the Range}]
    \begin{itemize}
        \item If price $<$ 2 oz potatoes: both want to \emph{buy} meat → no one sells.
        \item If price $>$ 4 oz potatoes: both want to \emph{sell} meat → no one buys.
    \end{itemize}
    Only a price between 2 and 4 creates incentives for one person to buy and the other to sell.
    \end{tcolorbox}
    
    A price in this range allows each person to buy goods at a cost lower than his or her own opportunity cost, making both parties better off.
\end{frame}
%======================================================================
\begin{frame}{Should the United States Trade with Other Countries?}

Just as individuals gain from specialization and trade, so do nations.  
Many goods consumed in the United States are produced abroad, and many U.S. goods are sold overseas.

\begin{tcolorbox}[colframe=red!80!black, colback=white, title=\textbf{Key Terms}]
\begin{itemize}
    \item \textbf{Imports}: goods produced abroad and sold domestically.
    \item \textbf{Exports}: goods produced domestically and sold abroad.
\end{itemize}
\end{tcolorbox}

Specialization allows each country to focus on producing the goods for which it has a comparative advantage, increasing total world output and raising living standards.

\end{frame}

%======================================================================
\begin{frame}{Comparative Advantage in International Trade}

Consider two countries (the United States and Japan) and two goods (food and cars):

\begin{itemize}
    \item Both countries can produce cars equally well (1 car per worker per month).
    \item The United States is more efficient at producing food:
    \begin{itemize}
        \item U.S.: 2 tons of food per worker per month
        \item Japan: 1 ton of food per worker per month
    \end{itemize}
\end{itemize}

\textbf{Comparative Advantage Results}
\begin{itemize}
    \item \textbf{Japan} has a comparative advantage in producing cars  
          (lower opportunity cost: 1 ton of food).
    \item \textbf{The United States} has a comparative advantage in producing food  
          (lower opportunity cost: 0.5 cars).
\end{itemize}

Thus:
\begin{itemize}
    \item Japan should specialize in cars and export cars to the United States.
    \item The United States should specialize in food and export food to Japan.
    \item Through specialization and trade, both countries can enjoy more food \textit{and} more cars.
\end{itemize}

\end{frame}

%======================================================================
\begin{frame}{Real-World Trade: Who Wins and Who Loses?}

Although countries as a whole gain from trade, individual groups within a country may be affected differently.

\begin{tcolorbox}[colframe=red!80!black, colback=white, title=\textbf{Key Point}]
Trade can make a nation better off overall, even if some individuals are made worse off.
\end{tcolorbox}

Examples:
\begin{itemize}
    \item When the U.S. exports food and imports cars,  
          farmers may benefit while some autoworkers may face increased competition.
    \item Political debates often frame trade as if countries “win” or “lose,”  
          but trade is not a zero-sum game.
\end{itemize}

\textbf{Conclusion}:International trade allows all countries to increase prosperity by specializing according to comparative advantage.

\end{frame}

%======================================================================
\end{document}
