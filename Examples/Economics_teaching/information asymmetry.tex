\documentclass{beamer}
\usepackage{graphicx} % Required for inserting images
\usepackage{amsmath}
\usepackage[most]{tcolorbox}
\usepackage{lmodern}
\usepackage{mathabx}

\usetheme{Madrid} % 可選其他主題:e.g., Warsaw, Berkeley, etc.
\usecolortheme{default}
\setbeamertemplate{caption}[numbered]% Number float-like environments
% Customize the caption
\setbeamerfont{caption}{size=\footnotesize}
% \setbeamercolor{caption}{fg=blue}
% \setbeamercolor{caption name}{fg=red}

\title{Frontiers of microeconomics}
\subtitle{section22.1: Asymmetric Information}
\author{Hsu Chun-Wei}
\date{2025}

% 每章節開始時自動產生章節頁
\AtBeginSection[]
{
  \begin{frame}
    \frametitle{Table of Contents}
    \tableofcontents[currentsection]
  \end{frame}
}

\begin{document}

\maketitle

% 目錄頁
\begin{frame}
  \frametitle{Table of Contents}
  \tableofcontents
\end{frame}
%======================================================================
\section{Moral hazard}
%======================================================================
\begin{frame}{Moral Hazard: Principal-Agent Problem}

\textbf{Moral hazard} is a problem that arises when one person, called the \textbf{agent}, performs a task on behalf of another person, called the \textbf{principal}.

\vspace{0.5em}
If the principal cannot perfectly monitor the agent’s behavior, the agent may exert less effort than is optimal for the principal.

\begin{tcolorbox}[colframe=red!80!black, colback=white, title=\textbf{Definition of Moral Hazard}]
Moral hazard refers to the \textbf{risk}, or \textbf{"hazard"}, of inappropriate or otherwise \textbf{"immoral"} behavior by the agent due to asymmetric information.
\end{tcolorbox}

\vspace{0.5em}
In such a situation, the principal tries various mechanisms to motivate the agent to act more responsibly.

\end{frame}
%======================================================================
\begin{frame}{Moral Hazard in Employment Relationships}

The employment relationship is a classic example of the principal-agent problem.

\vspace{0.5em}
\begin{itemize}
    \item The \textbf{employer} is the \textbf{principal}.
    \item The \textbf{worker} is the \textbf{agent}.
    \item The \textbf{moral hazard} arises when workers, who are imperfectly monitored, are tempted to \textbf{shirk their responsibilities}.
\end{itemize}

Employers can respond to this problem in various ways:

\begin{itemize}
    \item \textbf{Better monitoring}: Use hidden cameras or other tools to observe workers’ behavior and catch irresponsible actions when supervisors are absent.
    
    \item \textbf{High wages}: Pay workers above market wage to reduce shirking. Losing a high-paying job discourages bad behavior.

    \item \textbf{Delayed payment}: Postpone part of the salary (e.g., year-end bonuses). Workers who shirk risk losing future income.
\end{itemize}


\end{frame}
%=====================================================================
\section{Adverse Selection}
%======================================================================
\begin{frame}{Adverse Selection and the Lemons Problem}

\textbf{Adverse selection} is a problem that occurs when one party in a transaction has more information than the other.

\vspace{0.5em}
In particular, sellers may know more about the quality of a good than buyers do. This can lead to the sale of low-quality goods, which is harmful to uninformed buyers.

\begin{tcolorbox}[colframe=red!80!black, colback=white, title=\textbf{Key Idea}]
The "selection" of goods in the market is \textbf{adverse} from the buyer’s perspective due to hidden characteristics.
\end{tcolorbox}

\end{frame}

%======================================================================
\begin{frame}{Adverse Selection}

\textbf{Used car markets} are a classic example of adverse selection.

\vspace{0.5em}
\begin{itemize}
    \item Sellers know more about their car’s defects than buyers.
    \item Owners of bad cars are more likely to sell them—buyers fear getting a \textbf{“lemon”}.
    \item This fear reduces buyer trust and participation in the used car market.
    \item Even slightly used cars may sell for much less than new ones, due to buyers’ suspicion.
\end{itemize}

\begin{tcolorbox}[colframe=red!80!black, colback=white, title=\textbf{The adverse selection}]
the tendency for the mix of unobserved attributes to become undesirable from the standpoint of an uninformed party
\end{tcolorbox}

\end{frame}
%=====================================================================
\section{Signaling}
%======================================================================
\begin{frame}{Signaling: A Response to Asymmetric Information}

\textbf{Signaling} refers to actions taken by an informed party to credibly reveal private information to uninformed parties.

\vspace{0.5em}
\begin{itemize}
    \item \textbf{Firms} may advertise to signal product quality to potential customers.
    \item \textbf{Students} may earn college degrees to signal high ability to employers—even if the degree doesn’t increase productivity.
\end{itemize}

\begin{tcolorbox}[colframe=red!80!black, colback=white, title=\textbf{Purpose of Signaling}]
To reduce the effects of asymmetric information by making hidden qualities observable.
\end{tcolorbox}

\end{frame}
%=====================================================================
\section{Screening}
%======================================================================
\begin{frame}{Screening: Inducing Information Revelation}

\textbf{Screening} is when an \textit{uninformed party} takes action to induce the informed party to reveal private information.

\vspace{0.5em}
\begin{itemize}
    \item \textbf{Contrast with signaling}: 
    \begin{itemize}
        \item \textbf{Signaling} — informed party reveals information.
        \item \textbf{Screening} — uninformed party draws out information.
    \end{itemize}

    \item \textbf{Example}: A buyer asks for a used car to be inspected by a mechanic.
    \begin{itemize}
        \item If the seller refuses, it may signal the car is a lemon.
        \item The buyer can then choose to walk away or offer a lower price.
    \end{itemize}
\end{itemize}

\begin{tcolorbox}[colframe=red!80!black, colback=white, title=\textbf{Key Idea}]
Screening helps buyers protect themselves from hidden risks in the presence of asymmetric information.
\end{tcolorbox}

\end{frame}
%======================================================================
\section{Conclusion}
%======================================================================
\begin{frame}{Asymmetric Information and Public Policy (1/2)}

\textbf{Asymmetric information} adds to the list of reasons why markets may fail.

\vspace{0.5em}
\begin{itemize}
    \item Classical economics emphasizes the efficiency of markets (e.g., supply and demand, the invisible hand).
    \item But market outcomes may be suboptimal due to:
    \begin{itemize}
        \item Externalities
        \item Public goods
        \item Imperfect competition
        \item \textbf{Asymmetric information}
    \end{itemize}

    \item Hidden information can prevent efficient resource allocation:
    \begin{itemize}
        \item High-quality goods may not sell (e.g., used cars).
        \item Healthy individuals may pay more for insurance due to pooling with hidden-risk individuals.
    \end{itemize}
\end{itemize}

\end{frame}

%======================================================================
\begin{frame}{Asymmetric Information and Public Policy (2/2)}

Government intervention may be helpful, but three issues complicate the case:

\vspace{0.5em}
\begin{enumerate}
    \item \textbf{Private solutions exist}: Markets use \textbf{signaling} and \textbf{screening} to reduce information gaps.
    
    \item \textbf{Government lacks superior information}: Often, private individuals know more than regulators.

    \item \textbf{Government is imperfect}: Intervention may lead to other inefficiencies or unintended consequences.
\end{enumerate}

\begin{tcolorbox}[colframe=red!80!black, colback=white, title=\textbf{Conclusion}]
Asymmetric information justifies potential policy action, but it must be weighed against the limits of government capability.
\end{tcolorbox}

\end{frame}

%======================================================================
%======================================================================
%======================================================================
%======================================================================
%======================================================================
\end{document}
