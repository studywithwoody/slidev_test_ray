\documentclass{beamer}
\usepackage{graphicx} % Required for inserting images
\usepackage{amsmath}
\usepackage[most]{tcolorbox}
\usepackage{lmodern}
\usepackage{mathabx}

\usetheme{Madrid} % 可選其他主題:e.g., Warsaw, Berkeley, etc.
\usecolortheme{default}
\setbeamertemplate{caption}[numbered]% Number float-like environments
% Customize the caption
\setbeamerfont{caption}{size=\footnotesize}
% \setbeamercolor{caption}{fg=blue}
% \setbeamercolor{caption name}{fg=red}

% 每章節開始時自動產生章節頁
\AtBeginSection[]
{
  \begin{frame}
    \frametitle{Table of Contents}
    \tableofcontents[currentsection]
  \end{frame}
}


\title{Mankiw's Principles of Economics}
\subtitle{chap 7: Consumers, Producers, and the Efficiency of Markets}
\author{Hsu Chun-Wei}
\date{July 2025}

\begin{document}

\maketitle

% 目錄頁
\begin{frame}
  \frametitle{Table of Contents}
  \tableofcontents
\end{frame}

%==========================================================
\begin{frame}{Welfare Economics and Market Outcomes}

Welfare economics studies how the allocation of resources affects overall economic well-being. 
In this chapter, we focus on the benefits that buyers and sellers receive from participating in market transactions.

\begin{tcolorbox}[colframe=blue!70!black, colback=white, title=\textbf{Why Market Prices Work}]
\begin{itemize}
    \item Market prices coordinate the actions of buyers and sellers.
    \item Individuals pursue their own interests, not social welfare.
    \item Yet market outcomes move the economy toward a welfare-maximizing result.
\end{itemize}
\end{tcolorbox}

As if guided by an invisible hand, no single consumer or producer intends to maximize total welfare, 
but their interaction through markets achieves this outcome.

\end{frame}

%==========================================================
\begin{frame}{Consumer Surplus: Willingness to Pay}

We begin the study of welfare economics by examining the benefits buyers receive from participating in a market.

\begin{tcolorbox}[colframe=red!80!black, colback=white, title=\textbf{Willingness to Pay}]
Willingness to pay is the maximum amount that a buyer is willing to pay for a good.
\end{tcolorbox}

A buyer compares the market price with his willingness to pay:
\begin{itemize}
    \item If the price is lower than willingness to pay, the buyer chooses to buy.
    \item If the price is higher, the buyer refuses to buy.
    \item If the price equals willingness to pay, the buyer is indifferent.
\end{itemize}

Willingness to pay measures how much a buyer values a good.

\end{frame}

%==========================================================
\begin{frame}{Consumer Surplus: One Album Auction}

Consider an auction for a mint-condition Elvis Presley album with four potential buyers:

\begin{itemize}
    \item John: willing to pay \$100
    \item Paul: willing to pay \$80
    \item George: willing to pay \$70
    \item Ringo: willing to pay \$50
\end{itemize}

The bidding starts at a low price and rises until it reaches \$80.
At this price, only John is willing to continue bidding.

\begin{tcolorbox}[colframe=blue!70!black, colback=white, title=\textbf{Consumer Surplus}]
Consumer surplus equals willingness to pay minus the price actually paid.
\end{tcolorbox}

John pays \$80 for a good he values at \$100, so his consumer surplus is \$20.

\end{frame}

%==========================================================
\begin{frame}{Consumer Surplus: Two Albums and Total Benefit}

Now suppose there are two identical albums for sale, and each buyer wants at most one album.
Both albums sell at the same price.

The bidding stops when John and Paul bid \$70.

\begin{itemize}
    \item John’s consumer surplus: \$100 - \$70 = \$30
    \item Paul’s consumer surplus: \$80 - \$70 = \$10
\end{itemize}

\begin{tcolorbox}[colframe=red!70!black, colback=white, title=\textbf{Total Consumer Surplus}]
Total consumer surplus is the sum of consumer surplus across all buyers in the market.
\end{tcolorbox}

In this market, total consumer surplus equals \$40.

\end{frame}

%==========================================================
\begin{frame}{Using the Demand Curve to Measure Consumer Surplus}

Consumer surplus is closely related to the demand curve for a product.
To see this relationship, we derive the demand curve from buyers’ willingness to pay.

\begin{tcolorbox}[colframe=red!80!black, colback=white, title=\textbf{Deriving the Demand Curve}]
The market demand schedule is constructed by ranking buyers according to their willingness to pay.
\end{tcolorbox}

As the price falls:
\begin{itemize}
    \item More buyers are willing to purchase the good.
    \item Quantity demanded increases step by step.
\end{itemize}

Thus, the demand curve reflects the willingness to pay of all potential buyers in the market.

\end{frame}

%==========================================================
\begin{frame}{The Demand Curve and the Marginal Buyer}

The demand curve shows how quantity demanded varies with price.
Its height at any quantity represents buyers’ willingness to pay.

\begin{tcolorbox}[colframe=blue!70!black, colback=white, title=\textbf{Marginal Buyer}]
The marginal buyer is the buyer who would leave the market first if the price rose any higher.
\end{tcolorbox}

At each quantity:
\begin{itemize}
    \item The price on the demand curve equals the marginal buyer’s willingness to pay.
    \item Higher prices exclude buyers with lower willingness to pay.
\end{itemize}

Therefore, the demand curve directly measures how much the marginal buyer values the good.

\end{frame}

%==========================================================
\begin{frame}{Consumer Surplus and the Demand Curve}

Because the demand curve reflects willingness to pay, it can be used to measure consumer surplus.

\begin{tcolorbox}[colframe=red!70!black, colback=white, title=\textbf{Consumer Surplus on a Graph}]
Consumer surplus equals the area below the demand curve and above the market price.
\end{tcolorbox}

At a given price:
\begin{itemize}
    \item Each buyer’s consumer surplus equals willingness to pay minus price.
    \item Total consumer surplus is the sum of consumer surplus across all buyers.
\end{itemize}

Thus, the area under the demand curve and above the price measures total consumer surplus in a market.

\end{frame}

%==========================================================
%==========================================================
%==========================================================
%==========================================================
%==========================================================
%==========================================================
%==========================================================
%==========================================================
%==========================================================
%==========================================================
%==========================================================
%==========================================================
%==========================================================
%==========================================================
%==========================================================
%==========================================================
%==========================================================
%==========================================================
\end{document}