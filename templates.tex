\documentclass{beamer}
\usepackage{graphicx} % Required for inserting images
\usepackage{amsmath}
\usepackage[most]{tcolorbox}
\usepackage{lmodern}
\usepackage{mathabx}

\usetheme{Madrid} % 可選其他主題:e.g., Warsaw, Berkeley, etc.
% \useoutertheme[subsection=False]{miniframes} 
% 確保外層主題啟用,才會有上方目錄列
\usecolortheme{default}
% \setbeamertemplate{caption}[numbered]% Number float-like environments
% % Customize the caption
% \setbeamerfont{caption}{size=\footnotesize}
% \setbeamercolor{caption}{fg=blue}
% \setbeamercolor{caption name}{fg=red}

% 每章節開始時自動產生章節頁
\AtBeginSection[]
{
  \begin{frame}
    \frametitle{Table of Contents}
    \tableofcontents[currentsection]
  \end{frame}
}


\title{AP Precalculus}
\subtitle{Chapter 1.2: EXPONENTS AND RADICALS}
\author{Hsu Chun-Wei}
\date{July 2025}

\begin{document}

\maketitle

% 目錄頁
\begin{frame}
  \frametitle{Table of Contents}
  \tableofcontents
\end{frame}
%======================================================================
% \begin{columns}
%     \begin{column}{0.5\textwidth}
        
%     \end{column}
%     \begin{column}{0.5\textwidth}
        
%     \end{column}
% \end{columns}
%======================================================================
\section{EXPONENTS AND RADICALS}

\begin{frame}{Exponents and Radicals}
    In this section, we give meaning to expressions such as $a^{m/n}$, where the exponent $m/n$ is a rational number.  
    To prepare for this, we first recall key ideas about integer exponents, radicals, and $n$th roots.
\end{frame}
%======================================================================
\begin{frame}{Integer Exponents}

\textbf{Integer Exponents}

A product of identical numbers is usually written using exponential notation.  
For example,
\[
5 \cdot 5 \cdot 5 = 5^{3}.
\]
In general, we have the following definition.

\begin{tcolorbox}[colframe=red!80!black, colback=white, title=\textbf{Exponential Notation}]
If $a$ is any real number and $n$ is a positive integer, then the $n$th power of $a$ is defined as
\[
a^{n} = \underbrace{a \cdot a \cdot \ldots \cdot a}_{n \text{ factors}}.
\]

\vspace{0.5em}
The number $a$ is called the \textbf{base}, and $n$ is called the \textbf{exponent}.
\end{tcolorbox}

\end{frame}
%======================================================================
\begin{frame}{Example: Exponential Notation}

\begin{tcolorbox}[colframe=red!80!black, colback=white, title=\textbf{Example 1}]
\begin{itemize}
  \item[(a)]
  \[
  \left(\frac{1}{2}\right)^{5}
  = \left(\frac{1}{2}\right)\left(\frac{1}{2}\right)\left(\frac{1}{2}\right)\left(\frac{1}{2}\right)\left(\frac{1}{2}\right)
  = \frac{1}{32}
  \]

  \item[(b)]
  \[
  (-3)^{4}
  = (-3)(-3)(-3)(-3)
  = 81
  \]

  \item[(c)]
  \[
  -3^{4}
  = -(3 \cdot 3 \cdot 3 \cdot 3)
  = -81
  \]
\end{itemize}
\end{tcolorbox}

\vspace{0.6em}
\textbf{Important Note:}  
\[
(-3)^4 \neq -3^4
\]
The exponent applies only to the base inside the parentheses.

\end{frame}

%======================================================================
\begin{frame}{Rules for Working with Exponents}

To discover the rule for multiplication, consider the following example:
\[
5^{4} \cdot 5^{2}
= (5 \cdot 5 \cdot 5 \cdot 5)(5 \cdot 5)
= 5 \cdot 5 \cdot 5 \cdot 5 \cdot 5 \cdot 5
= 5^{6}
= 5^{4+2}.
\]

\vspace{0.8em}
This suggests the following general rule.

\begin{tcolorbox}[colframe=red!80!black, colback=white, title=\textbf{Product Rule for Exponents}]
For any real number $a$ and positive integers $m$ and $n$,
\[
a^{m} \cdot a^{n} = a^{m+n}.
\]
\end{tcolorbox}

\vspace{0.5em}
That is, \textbf{to multiply two powers with the same base, add their exponents}.

\end{frame}

%======================================================================
\begin{frame}{Zero and Negative Exponents}

We would like the product rule
\(
a^{m} \cdot a^{n} = a^{m+n}
\)
to remain valid even when exponents are zero or negative.

\vspace{0.6em}
For example,
\[
2^{0} \cdot 2^{3} = 2^{0+3} = 2^{3},
\]
which is possible only if $2^{0} = 1$.

Similarly,
\[
5^{4} \cdot 5^{-4} = 5^{4+(-4)} = 5^{0} = 1,
\]
which implies $5^{-4} = \dfrac{1}{5^{4}}$.

\begin{tcolorbox}[colframe=red!80!black, colback=white, title=\textbf{Zero and Negative Exponents}]
If $a \neq 0$ is any real number and $n$ is a positive integer, then
\[
a^{0} = 1
\qquad \text{and} \qquad
a^{-n} = \frac{1}{a^{n}}.
\]
\end{tcolorbox}

\end{frame}

%======================================================================
\begin{frame}{Example: Zero and Negative Exponents}

\begin{tcolorbox}[colframe=red!80!black, colback=white, title=\textbf{Example 2}]
\begin{itemize}
  \item[(a)]
  \[
  \left(\frac{4}{7}\right)^{0} = 1
  \]

  \item[(b)]
  \[
  x^{-1} = \frac{1}{x^{1}} = \frac{1}{x}
  \]

  \item[(c)]
  \[
  (-2)^{-3}
  = \frac{1}{(-2)^{3}}
  = \frac{1}{-8}
  = -\frac{1}{8}
  \]
\end{itemize}
\end{tcolorbox}

\vspace{0.6em}
\textbf{Key Idea:}  
A zero exponent gives $1$, and a negative exponent represents a reciprocal.

\end{frame}

%======================================================================
\begin{frame}{Rules for Working with Exponents}

Familiarity with the following rules is essential for working with exponents and bases.

\vspace{0.5em}
In the rules below:
\begin{itemize}
  \item $a$ and $b$ are real numbers,
  \item $m$ and $n$ are integers,
  \item $a \neq 0$, $b \neq 0$ when division is involved.
\end{itemize}

\vspace{0.8em}
These laws allow us to simplify expressions involving exponents efficiently.

\end{frame}

%======================================================================
\begin{frame}{Laws of Exponents}

\begin{tcolorbox}[colframe=red!80!black, colback=white, title=\textbf{Laws of Exponents}]
\[
\begin{array}{lll}
\textbf{Law} & \textbf{Example} & \textbf{Description} \\[0.4em]

1.\; a^{m}a^{n} = a^{m+n}
& 3^{2}\cdot3^{5}=3^{7}
& \text{ \footnotesize Add exponents when multiplying.} \\[0.6em]
2.\; \dfrac{a^{m}}{a^{n}} = a^{m-n}
& \dfrac{3^{5}}{3^{2}}=3^{3}
& \text{ \footnotesize Subtract exponents when dividing.} \\[0.8em]

3.\; (a^{m})^{n} = a^{mn}
& (3^{2})^{5}=3^{10}
& \text{ \footnotesize Multiply exponents.} \\[0.8em]

4.\; (ab)^{n} = a^{n}b^{n}
& (3\cdot4)^{2}=3^{2}\cdot4^{2}
& \text{ \footnotesize Raise each factor to the power.} \\[0.8em]

5.\; \left(\dfrac{a}{b}\right)^{n} = \dfrac{a^{n}}{b^{n}}
& \left(\dfrac{3}{4}\right)^{2}=\dfrac{3^{2}}{4^{2}}
& \text{ \footnotesize Raise numerator and denominator.}
\end{array}
\]
\end{tcolorbox}

\end{frame}
%============================================================
\begin{frame}{Proof of Law 3: $(a^{m})^{n} = a^{mn}$}

If $m$ and $n$ are positive integers, we have
\[
(a^{m})^{n}
= (a \cdot a \cdot \ldots \cdot a)^{n},
\]
where there are $m$ factors of $a$.

\vspace{0.5em}
Raising this expression to the power $n$ gives
\[
(a \cdot a \cdot \ldots \cdot a)
(a \cdot a \cdot \ldots \cdot a)
\cdots
(a \cdot a \cdot \ldots \cdot a),
\]
which consists of $n$ groups, each with $m$ factors.

\begin{tcolorbox}[colframe=red!80!black, colback=white, title=\textbf{Conclusion}]
Altogether, there are $mn$ factors of $a$, so
\[
(a^{m})^{n} = a^{mn}.
\]
\end{tcolorbox}

\vspace{0.4em}
The cases for which $m \le 0$ or $n \le 0$ can be proved using the definition of negative exponents.

\end{frame}

%============================================================
\begin{frame}{Proof of Law 4: $(ab)^{n} = a^{n}b^{n}$}

If $n$ is a positive integer, then
\[
(ab)^{n} = (ab)(ab)\cdots(ab),
\]
which contains $n$ identical factors.

\vspace{0.5em}
Using the Commutative and Associative Properties, we regroup the factors:
\[
(ab)(ab)\cdots(ab)
= (a \cdot a \cdot \ldots \cdot a)(b \cdot b \cdot \ldots \cdot b).
\]

\begin{tcolorbox}[colframe=red!80!black, colback=white, title=\textbf{Conclusion}]
There are $n$ factors of $a$ and $n$ factors of $b$, so
\[
(ab)^{n} = a^{n}b^{n}.
\]
\end{tcolorbox}

\vspace{0.4em}
If $n \le 0$, this law can be proved using the definition of negative exponents.

\end{frame}

%============================================================
\begin{frame}{Example: Using Laws of Exponents}

\begin{itemize}
  \item[(a)]
  \[
  x^{4}x^{7} = x^{4+7} = x^{11}
  \qquad \text{(Law 1: } a^{m}a^{n}=a^{m+n}\text{)}
  \]

  \item[(b)]
  \[
  y^{4}y^{-7} = y^{4-7} = y^{-3} = \frac{1}{y^{3}}
  \qquad \text{(Law 1)}
  \]

  \item[(c)]
  \[
  \frac{c^{9}}{c^{5}} = c^{9-5} = c^{4}
  \qquad \text{(Law 2: } \frac{a^{m}}{a^{n}}=a^{m-n}\text{)}
  \]

  \item[(d)]
  \[
  (b^{4})^{5} = b^{4\cdot 5} = b^{20}
  \qquad \text{(Law 3: } (a^{m})^{n}=a^{mn}\text{)}
  \]

  \item[(e)]
  \[
  (3x)^{3} = 3^{3}x^{3} = 27x^{3}
  \qquad \text{(Law 4: } (ab)^{n}=a^{n}b^{n}\text{)}
  \]

  \item[(f)]
  \[
  \left(\frac{x}{2}\right)^{5}
  = \frac{x^{5}}{2^{5}}
  = \frac{x^{5}}{32}
  \qquad \text{(Law 5: } \left(\frac{a}{b}\right)^{n}=\frac{a^{n}}{b^{n}}\text{)}
  \]
\end{itemize}

\end{frame}
%============================================================
\begin{frame}{Example: Simplifying Expressions with Exponents}

\textbf{Simplify:}
\begin{itemize}
    \item \(
(2a^{3}b^{2})(3ab^{4})^{3}
\)
    \item \(
\left(\frac{x}{y}\right)^{3}
\left(\frac{y^{2}x}{z}\right)^{4}
\)
\end{itemize}
\end{frame}
%============================================================
\begin{frame}{Example: Simplifying Expressions with Exponents}

\textbf{Simplify:}

\begin{tcolorbox}[colframe=red!80!black, colback=white, title=\textbf{Example 4(a)}]
\[
(2a^{3}b^{2})(3ab^{4})^{3}
\]

\[
= (2a^{3}b^{2})\bigl[3^{3}a^{3}(b^{4})^{3}\bigr]
\qquad \text{(Law 4 and Law 3)}
\]

\[
= (2a^{3}b^{2})(27a^{3}b^{12})
\]

\[
= (2)(27)a^{3}a^{3}b^{2}b^{12}
\qquad \text{(Group like bases)}
\]

\[
= 54a^{6}b^{14}
\qquad \text{(Law 1)}
\]
\end{tcolorbox}

\end{frame}

%============================================================
\begin{frame}{Example: Simplifying Expressions with Exponents}

\textbf{Simplify:}

\begin{tcolorbox}[colframe=red!80!black, colback=white, title=\textbf{Example 4(b)}]
\[
\left(\frac{x}{y}\right)^{3}
\left(\frac{y^{2}x}{z}\right)^{4}
\]

\[
= \frac{x^{3}}{y^{3}} \cdot \frac{(y^{2})^{4}x^{4}}{z^{4}}
\qquad \text{(Laws 5 and 4)}
\]

\[
= \frac{x^{3}y^{8}x^{4}}{y^{3}z^{4}}
\qquad \text{(Law 3)}
\]

\[
= \frac{x^{7}y^{5}}{z^{4}}
\qquad \text{(Laws 1 and 2)}
\]
\end{tcolorbox}

\end{frame}

%============================================================
\begin{frame}{Laws of Exponents: Negative Exponents}

\begin{tcolorbox}[colframe=red!80!black, colback=white, title=\textbf{Laws of Exponents}]

\begin{center}
\begin{tabular}{lll}
\textbf{Law} & \textbf{Example} & \textbf{Description} \\[0.6em]

6.\;
$\left(\dfrac{a}{b}\right)^{-n} = \left(\dfrac{b}{a}\right)^{n}$
&
$\left(\dfrac{3}{4}\right)^{-2} = \left(\dfrac{4}{3}\right)^{2}$
&
\parbox[c]{6cm}{\footnotesize
Invert the fraction and change\\ the sign of the exponent.
}
\\[1.0em]

7.\;
$\dfrac{a^{-n}}{b^{-m}} = \dfrac{b^{m}}{a^{n}}$
&
$\dfrac{3^{-2}}{4^{-5}} = \dfrac{4^{5}}{3^{2}}$
&
\parbox[c]{6cm}{\footnotesize
Move factors between \\ numerator and denominator.
}
\end{tabular}
\end{center}

\end{tcolorbox}

\end{frame}

%============================================================
\begin{frame}{Proof of Law 7}

Using the definition of negative exponents, we have
\[
\dfrac{a^{-n}}{b^{-m}}
= \dfrac{\dfrac{1}{a^{n}}}{\dfrac{1}{b^{m}}}.
\]

\vspace{0.6em}
Dividing by a fraction is equivalent to multiplying by its reciprocal:
\[
\dfrac{\dfrac{1}{a^{n}}}{\dfrac{1}{b^{m}}}
= \dfrac{1}{a^{n}} \cdot \dfrac{b^{m}}{1}.
\]

\begin{tcolorbox}[colframe=red!80!black, colback=white, title=\textbf{Conclusion}]
\[
\dfrac{a^{-n}}{b^{-m}} = \dfrac{b^{m}}{a^{n}}.
\]
\end{tcolorbox}

\end{frame}
%============================================================
\begin{frame}{Example 5: Simplifying with Negative Exponents}

\textbf{Eliminate negative exponents and simplify each expression.}
    \begin{enumerate}
        \item \(
\frac{6s t^{-4}}{2s^{-2} t^{2}}
\)
        \item \(
\left(\frac{y}{3z^{3}}\right)^{-2}
\)
    \end{enumerate}
\end{frame}

%============================================================
\begin{frame}{Example 5: Simplifying with Negative Exponents}

\textbf{Eliminate negative exponents and simplify each expression.}

\begin{tcolorbox}[colframe=red!80!black, colback=white, title=\textbf{Example (1)}]
\[
\frac{6s t^{-4}}{2s^{-2} t^{2}}
\]

\vspace{0.4em}
Using \textbf{Law 7}, move factors with negative exponents across the fraction:
\[
= \frac{6s s^{2}}{2 t^{2} t^{4}}
\]

\vspace{0.4em}
Group like bases and simplify:
\[
= \frac{6s^{3}}{2t^{6}}
= \frac{3s^{3}}{t^{6}}
\qquad \text{(Law 1)}
\]
\end{tcolorbox}

\end{frame}

%============================================================
\begin{frame}{Example: Simplifying with Negative Exponents}

\textbf{Eliminate negative exponents and simplify each expression.}

\begin{tcolorbox}[colframe=red!80!black, colback=white, title=\textbf{Example (2)}]
\[
\left(\frac{y}{3z^{3}}\right)^{-2}
\]

\vspace{0.4em}
Using \textbf{Law 6}, invert the fraction and change the sign of the exponent:
\[
= \left(\frac{3z^{3}}{y}\right)^{2}
\]

\vspace{0.4em}
Apply the power to numerator and denominator:
\[
= \frac{9z^{6}}{y^{2}}
\qquad \text{(Laws 4 and 5)}
\]
\end{tcolorbox}

\end{frame}

%============================================================
\begin{frame}{Scientific Notation}

Scientists use exponential notation as a compact way to write very large
and very small numbers, which are otherwise difficult to read and write.

\begin{tcolorbox}[colframe=red!80!black, colback=white, title=\textbf{Scientific Notation}]
A positive number $x$ is written in \textbf{scientific notation} if it has the form
\[
x = a \times 10^{n},
\]
where
\[
1 \le a < 10 \quad \text{and} \quad n \text{ is an integer}.
\]
\end{tcolorbox}

\vspace{0.5em}
The number $a$ is called the \textbf{coefficient}, and $n$ is the \textbf{power of ten}.

\end{frame}

%============================================================
\begin{frame}{Scientific Notation: Interpreting the Exponent}

\textbf{Positive Exponent (Large Numbers)}

\[
4 \times 10^{13} = 40{,}000{,}000{,}000
\]
A positive exponent indicates that the decimal point is moved
\textbf{13 places to the right}.

\vspace{0.8em}
\textbf{Negative Exponent (Small Numbers)}

\[
1.66 \times 10^{-24}
= 0.00000000000000000000000166
\]
A negative exponent indicates that the decimal point is moved
\textbf{24 places to the left}.

\begin{tcolorbox}[colframe=red!80!black, colback=white, title=\textbf{Key Idea}]
\begin{itemize}
  \item Positive exponent $\rightarrow$ move the decimal point to the right.
  \item Negative exponent $\rightarrow$ move the decimal point to the left.
\end{itemize}
\end{tcolorbox}

\end{frame}

%============================================================
\begin{frame}{Example: Changing from Decimal to Scientific Notation}

\textbf{Write each number in scientific notation.}

\begin{tcolorbox}[colframe=red!80!black, colback=white, title=\textbf{Example}]
\begin{itemize}
  \item[(a)]
  \[
  56{,}920 = 5.692 \times 10^{4}
  \]
  The decimal point is moved \textbf{4 places to the left}.

  \vspace{0.6em}
  \item[(b)]
  \[
  0.000093 = 9.3 \times 10^{-5}
  \]
  The decimal point is moved \textbf{5 places to the right}.
\end{itemize}
\end{tcolorbox}

\vspace{0.5em}
\textbf{Reminder:}
\begin{itemize}
  \item Moving the decimal point to the left gives a \textbf{positive} exponent.
  \item Moving the decimal point to the right gives a \textbf{negative} exponent.
\end{itemize}

\end{frame}

%============================================================
\begin{frame}{Example: Calculating with Scientific Notation}

\textbf{Example:}

If
\[
a \approx 0.00046, \qquad
b \approx 1.697 \times 10^{22}, \qquad
c \approx 2.91 \times 10^{-18},
\]
use a calculator to approximate the quotient $\dfrac{ab}{c}$.

\begin{tcolorbox}[colframe=red!80!black, colback=white, title=\textbf{Solution}]
We rewrite all numbers in scientific notation and apply the laws of exponents:
\[
\frac{ab}{c}
\approx
\frac{(4.6 \times 10^{-4})(1.697 \times 10^{22})}{2.91 \times 10^{-18}}.
\]

\[
=
\frac{(4.6)(1.697)}{2.91}
\times 10^{-4+22+18}.
\]

\[
\approx 2.7 \times 10^{36}.
\]
\end{tcolorbox}

\vspace{0.4em}
\textbf{Note:}  
The answer is rounded to \textbf{two significant figures}, since the least accurate given value has two significant figures.

\end{frame}

%============================================================
\begin{frame}{Radicals: Square Roots}

To give meaning to powers with rational exponents, we first discuss radicals.

\vspace{0.5em}
The symbol $\sqrt{\ }$ means \textbf{the positive square root of}.

\begin{tcolorbox}[colframe=red!80!black, colback=white, title=\textbf{Square Root}]
\[
\sqrt{a} = b \quad \text{means} \quad b^{2} = a \ \text{and} \ b \ge 0.
\]
\end{tcolorbox}

\vspace{0.4em}
Since $a = b^{2} \ge 0$, the expression $\sqrt{a}$ is defined only when $a \ge 0$.

\[
\sqrt{9} = 3
\quad \text{because} \quad
3^{2} = 9 \ \text{and} \ 3 \ge 0.
\]

\end{frame}

%============================================================
\begin{frame}{Radicals: The $n$th Root}

Square roots are special cases of $n$th roots.

\begin{tcolorbox}[colframe=red!80!black, colback=white, title=\textbf{Definition of the $n$th Root}]
If $n$ is a positive integer, the \textbf{principal $n$th root} of $a$ is defined by
\[
\sqrt[n]{a} = b \quad \text{means} \quad b^{n} = a.
\]

If $n$ is even, we must have $a \ge 0$ and $b \ge 0$.
\end{tcolorbox}

\vspace{0.5em}
Examples:
\[
\sqrt[4]{81} = 3
\quad \text{because} \quad
3^{4} = 81,
\]
\[
\sqrt[3]{-8} = -2
\quad \text{because} \quad
(-2)^{3} = -8.
\]

\end{frame}

%============================================================
\begin{frame}{Radicals: Important Observations}

The expressions
\[
\sqrt{-8}, \quad \sqrt[4]{-8}, \quad \sqrt[6]{-8}
\]
are \textbf{not defined}, because even powers of real numbers are always nonnegative.

\vspace{0.6em}
Notice the difference:
\[
\sqrt{4^{2}} = \sqrt{16} = 4,
\qquad
\sqrt{(-4)^{2}} = \sqrt{16} = 4 = |{-4}|.
\]

\begin{tcolorbox}[colframe=red!80!black, colback=white, title=\textbf{Key Result}]
\[
\sqrt{a^{2}} = |a|.
\]
\end{tcolorbox}

\vspace{0.4em}
This result holds for square roots and, more generally, for any even root.

\end{frame}

%============================================================
\begin{frame}{Properties of the $n$th Roots}

\begin{tcolorbox}[colframe=red!80!black, colback=white, title=\textbf{Properties of $n$th Roots}]

\begin{center}
\begin{tabular}{lll}
\textbf{Property} & \textbf{Rule} & \textbf{Example} \\[0.6em]

1. &
$\sqrt[n]{ab} = \sqrt[n]{a}\sqrt[n]{b}$
&$\sqrt[3]{-8\cdot 27}
= \sqrt[3]{-8}\sqrt[3]{27}
= -6$
\\[0.9em]

2. &
$\sqrt[n]{\dfrac{a}{b}} = \dfrac{\sqrt[n]{a}}{\sqrt[n]{b}}$
&
$\sqrt[4]{\dfrac{16}{81}}
= \dfrac{\sqrt[4]{16}}{\sqrt[4]{81}}
= \dfrac{2}{3}$
\\[1.0em]

3. &
$\sqrt[m]{\sqrt[n]{a}} = \sqrt[mn]{a}$
&
$\sqrt{\sqrt[3]{729}}
= \sqrt[6]{729}
= 3$
\\[1.0em]

4. &
$\sqrt[n]{a^{n}} = a \quad \text{if $n$ is odd}$
&
$\sqrt[3]{(-5)^{3}} = -5,
\quad
\sqrt[5]{2^{5}} = 2$
\\[1.0em]

5. &
$\sqrt[n]{a^{n}} = |a| \quad \text{if $n$ is even}$
&
$\sqrt[4]{(-3)^{4}} = |-3| = 3$
\end{tabular}
\end{center}

\end{tcolorbox}


\vspace{0.4em}
\textbf{Note:}  
In each property, we assume that all given roots exist.

\end{frame}

%============================================================
\begin{frame}{Example: Simplifying Expressions Involving $n$th Roots}
    \textbf{Simplifying Expressions Involving $n$th Roots}
    \begin{enumerate}
        \item \(
\sqrt[3]{x^{4}}
\)
        \item \(
\sqrt[4]{81x^{8}y^{4}}
\)
    \end{enumerate}
\end{frame}
%============================================================
\begin{frame}{Example: Simplifying Expressions Involving $n$th Roots}

\begin{tcolorbox}[colframe=red!80!black, colback=white, title=\textbf{Example (1)}]
\[
\sqrt[3]{x^{4}}
\]

Factor out the largest perfect cube:
\[
= \sqrt[3]{x^{3}x}
\]

Apply the product property of roots:
\[
= \sqrt[3]{x^{3}}\sqrt[3]{x}
\qquad \text{(Property 1)}
\]

Since the index is odd:
\[
= x\sqrt[3]{x}
\qquad \text{(Property 4)}
\]
\end{tcolorbox}

\end{frame}

%============================================================
\begin{frame}{Example: Simplifying Expressions Involving $n$th Roots}

\begin{tcolorbox}[colframe=red!80!black, colback=white, title=\textbf{Example (2)}]
\[
\sqrt[4]{81x^{8}y^{4}}
\]

Rewrite each factor under the root:
\[
= \sqrt[4]{81}\sqrt[4]{x^{8}}\sqrt[4]{y^{4}}
\qquad \text{(Property 1)}
\]

Simplify each term:
\[
= 3\sqrt[4]{(x^{2})^{4}}\,|y|
\qquad \text{(Property 5)}
\]

Since $|x^{2}| = x^{2}$:
\[
= 3x^{2}|y|
\]
\end{tcolorbox}

\end{frame}

%============================================================
\begin{frame}{Example: Combining Radicals}

\begin{itemize}
  \item[(a)]
  \[
  \sqrt{32} + \sqrt{200}
  = \sqrt{16\cdot 2} + \sqrt{100\cdot 2}
  \]
  \[
  = \sqrt{16}\sqrt{2} + \sqrt{100}\sqrt{2}
  \qquad \text{(Property 1: } \sqrt{ab}=\sqrt{a}\sqrt{b}\text{)}
  \]
  \[
  = 4\sqrt{2} + 10\sqrt{2}
  = 14\sqrt{2}
  \qquad \text{(Distributive Property)}
  \]

  \vspace{0.8em}
  \item[(b)] If $b>0$, then
  \[
  \sqrt{25b} - \sqrt{b^{3}}
  = \sqrt{25}\sqrt{b} - \sqrt{b^{2}}\sqrt{b}
  \qquad \text{(Property 1)}
  \]
  \[
  = 5\sqrt{b} - b\sqrt{b}
  \qquad \text{(Property 5, } b>0\text{)}
  \]
  \[
  = (5-b)\sqrt{b}
  \qquad \text{(Distributive Property)}
  \]
\end{itemize}

\vspace{0.4em}
\textbf{Key Idea:}  
Radicals can be combined \textbf{only after} they have the same radical part.

\end{frame}

%============================================================
\begin{frame}{Rational Exponents}

To define a \textbf{rational exponent} (or fractional exponent) such as $a^{1/3}$,
we use radicals.

\vspace{0.6em}
To remain consistent with the Laws of Exponents, we require
\[
(a^{1/n})^{n} = a^{(1/n)n} = a^{1} = a.
\]

\vspace{0.6em}
By the definition of the $n$th root, this leads to

\begin{tcolorbox}[colframe=red!80!black, colback=white, title=\textbf{Key Relationship}]
\[
a^{1/n} = \sqrt[n]{a}.
\]
\end{tcolorbox}

\end{frame}

%============================================================
\begin{frame}{Definition of Rational Exponents}

\begin{tcolorbox}[colframe=red!80!black, colback=white, title=\textbf{Definition of Rational Exponents}]
For any rational exponent $\dfrac{m}{n}$ in lowest terms, where $m$ and $n$ are integers
and $n>0$, we define
\[
a^{m/n} = \left(\sqrt[n]{a}\right)^{m}
\quad \text{or equivalently} \quad
a^{m/n} = \sqrt[n]{a^{m}}.
\]

\vspace{0.4em}
If $n$ is even, we require that $a \ge 0$.
\end{tcolorbox}

\vspace{0.5em}
With this definition, the \textbf{Laws of Exponents} also hold for rational exponents.

\end{frame}

%============================================================
\begin{frame}{Example: Using the Definition of Rational Exponents}

\begin{itemize}
  \item[(a)]
  \[
  4^{1/2} = \sqrt{4} = 2
  \]

  \vspace{0.2em}
  \item[(b)]
  \[
  8^{2/3} = \left(\sqrt[3]{8}\right)^{2} = 2^{2} = 4
  \qquad \textit{or} \qquad 8^{2/3} = \sqrt[3]{8^{2}} = \sqrt[3]{64} = 4\] 
  
  \vspace{0.2em}
  \item[(c)]
  \[
  125^{-1/3}
  = \frac{1}{125^{1/3}}
  = \frac{1}{\sqrt[3]{125}}
  = \frac{1}{5}
  \]

  \vspace{0.2em}
  \item[(d)]
  \[
  \frac{1}{\sqrt[3]{x^{4}}}
  = \frac{1}{x^{4/3}}
  = x^{-4/3}
  \]
\end{itemize}

\vspace{0.2em}
\textbf{Key Idea:}  
Rational exponents can be rewritten using radicals, and negative exponents represent reciprocals.

\end{frame}

%===========================================================
\begin{frame}{Example: Writing Radicals as Rational Exponents}
\begin{itemize}
  \item[(a)]
  \[
  (2\sqrt{x})(3\sqrt[3]{x})
  = (2x^{1/2})(3x^{1/3})
  \qquad \text{(Definition of rational exponents)}
  \]
  \[
  = 6x^{1/2+1/3}
  = 6x^{5/6}
  \qquad \text{(Law 1)}
  \]

  \vspace{0.6em}
  \item[(b)]
  \[
  \sqrt{x\sqrt{x}}
  = (x x^{1/2})^{1/2}
  \qquad \text{(Definition of rational exponents)}
  \]
  \[
  = (x^{3/2})^{1/2}
  \qquad \text{(Law 1)}
  \]
  \[
  = x^{3/4}
  \qquad \text{(Law 3)}
  \]
\end{itemize}

\vspace{0.4em}
\textbf{Key Idea:}  
Rewrite radicals as rational exponents first, then apply the Laws of Exponents.

\end{frame}

%============================================================
\begin{frame}{Rationalizing the Denominator}

It is often useful to eliminate radicals in a denominator by multiplying
both the numerator and the denominator by an appropriate expression.
This process is called \textbf{rationalizing the denominator}.

\vspace{0.6em}
If the denominator is of the form $\sqrt{a}$, we multiply by $\sqrt{a}$:
\[
\frac{1}{\sqrt{a}}
= \frac{1}{\sqrt{a}} \cdot \frac{\sqrt{a}}{\sqrt{a}}
= \frac{\sqrt{a}}{a}.
\]
This multiplication does not change the value of the expression, since we
are multiplying by $1$.

\begin{tcolorbox}[colframe=red!80!black, colback=white, title=\textbf{General Method}]
If the denominator is of the form $\sqrt[n]{a^{m}}$ with $m<n$, then
multiplying numerator and denominator by $\sqrt[n]{a^{\,n-m}}$ will
rationalize the denominator (for $a>0$):
\[
\sqrt[n]{a^{m}}\,\sqrt[n]{a^{\,n-m}}
= \sqrt[n]{a^{\,n}}
= a.
\]
\end{tcolorbox}

\vspace{0.4em}
After rationalizing, the denominator contains \textbf{no radicals}.

\end{frame}

%============================================================
\begin{frame}{Example: Rationalizing Denominators}

\begin{itemize}
  \item[(a)]
  \[
  \frac{2}{\sqrt{3}}
  = \frac{2}{\sqrt{3}} \cdot \frac{\sqrt{3}}{\sqrt{3}}
  = \frac{2\sqrt{3}}{3}
  \]
  (Multiplying by $1$)

  \vspace{0.6em}
  \item[(b)]
  \[
  \frac{1}{\sqrt[3]{x^{2}}}
  = \frac{1}{\sqrt[3]{x^{2}}} \cdot \frac{\sqrt[3]{x}}{\sqrt[3]{x}}
  = \frac{\sqrt[3]{x}}{\sqrt[3]{x^{3}}}
  = \frac{\sqrt[3]{x}}{x}
  \]

  \vspace{0.6em}
  \item[(c)]
  \[
  \sqrt[7]{\frac{1}{a^{2}}}
  = \frac{1}{\sqrt[7]{a^{2}}}
  = \frac{1}{\sqrt[7]{a^{2}}} \cdot \frac{\sqrt[7]{a^{5}}}{\sqrt[7]{a^{5}}}
  = \frac{\sqrt[7]{a^{5}}}{\sqrt[7]{a^{7}}}
  = \frac{\sqrt[7]{a^{5}}}{a}
  \]
\end{itemize}

\vspace{0.4em}
\textbf{Key Idea:}  
Rationalize the denominator by multiplying by a form of $1$ so that the denominator contains no radicals.

\end{frame}

%============================================================
%============================================================
%============================================================
%============================================================
%============================================================
%============================================================
%============================================================


\end{document}
